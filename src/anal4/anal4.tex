\documentclass[fleqn,10pt,a4paper,titlepage]{article}
\includeonly{common,terstrukturak,diffszamitas,intszamitas}



%
% - ---- -- PACKAGES--------------------------
%

\usepackage{amssymb}
\usepackage{amsmath}
\usepackage[T1]{fontenc}
\usepackage[utf8]{inputenc}
\usepackage[magyar]{babel}
%\usepackage{amsthm}
\usepackage{theorem}
\usepackage{fancyhdr}
\usepackage{lastpage}
\usepackage{paralist}
\usepackage{enumerate}

%
% ---------- CODES --------------------------
%
\makeatletter
\gdef\th@magyar{\normalfont\slshape
  \def\@begintheorem##1##2{%
  \item[\hskip\labelsep \theorem@headerfont ##2.\ ##1.]}%
  \def\@opargbegintheorem##1##2##3{%
  \item[\hskip\labelsep \theorem@headerfont ##2. ##1.\ (##3)]}}
\makeatother


%
% ------------  N E W  C O M M A N D S --------
%


\theoremstyle{magyar}
\theoremheaderfont{\itshape\bfseries}
\newtheorem{de}{definíció}[section]
\newtheorem{te}{tétel}[section]
\newtheorem{bi}{bizonyítás}[section]
\newtheorem{ko}{következmény}[section]
\newtheorem{me}{megjegyzés}[section]
\newtheorem{al}{állítás}[section]

%
% ------- S E T T I N G S ----------------
%

\newcommand{\mktoc}{
  \pagenumbering{roman}
  \setcounter{page}{1}
  \lhead{\textbf{\thepage}}
  \cfoot{}
  \tableofcontents
  \newpage
  \lhead{\textbf{\thepage}}%/\pageref{LastPage}}
  \pagenumbering{arabic}
  \setcounter{page}{1}
}


\newcommand{\eqrho}{\stackrel{\varrho}{=}}
\newcommand{\eqrhon}[1]{\stackrel{\varrho_{#1}}{=}}
\newcommand{\MT}{\ensuremath{(M,\varrho)}\,}
\newcommand{\MTn}[1]{\ensuremath{(M,\varrho_{#1})}\,}
\newcommand{\sorozat}{\ensuremath{(a_n)\colon \N\n M}\,}
\newcommand{\sorozatn}[1]{\ensuremath{(a_{#1})\colon \N\n M}}

\renewcommand{\sectionmark}[1]{\markboth{\Roman{section}. fejezet\\#1}{}}
\newcommand{\hullam}{\widetilde}
\newcommand{\mr}[1]{(M_{#1},\,\ro_{#1})}
\newcommand{\fmm}{f\in M_1\n M_2}
\newcommand{\X}{\ensuremath{\mathcal{X}}}
\newcommand{\lp}{\ensuremath{l_p}}
\newcommand{\norma}[1]{\ensuremath{\left\Vert #1\right\Vert}}
\newcommand{\norman}[2]{\ensuremath{\norma{#1}^{(#2)}}}
\newcommand{\Norma}{\norma{\cdot}}
\newcommand{\Norman}[1]{\ensuremath{\Norma^{(#1)}}}
\newcommand{\opnorma}[1]{\ensuremath{|||#1|||}}
\newcommand{\NT}{\ensuremath{(\X,\,\Norma)}}
\newcommand{\skalar}[2]{\ensuremath{\langle#1,\,#2\rangle}}
\newcommand{\Skalar}{\skalar{.}{.}}
\newcommand{\skalarsz}{\Skalar}
\newcommand{\ET}{\ensuremath{(\X,\,\skalarsz)}}
\newcommand{\nullelem}{\mathsf{0}}
%\renewcommand{\square}{\blacksquare}
\newcommand{\RnRm}{\R^n\n\R^m}
\newcommand{\RnRn}{\R^n\n\R^n}
\newcommand{\RnR}{\R^n\n\R}
\newcommand{\RRm}{\R\n\R^m}
\newcommand{\Rnrm}{\RnRm}
\newcommand{\Rnrn}{\RnRn}
\newcommand{\Linearis}{\ensuremath{\mathcal{L}(\R^n,\,\R^m)}}
\DeclareMathOperator{\intD}{int\,D}
\DeclareMathOperator{\Folyt}{C}
\newcommand{\folyt}[1]{\Folyt\{#1\}}
\newcommand{\derivp}[1]{\D\{#1\}}
\newcommand{\dern}[2]{\D^{#1}\{#2\}}
\newcommand{\der}[1]{\derivp{#1}}
\newcommand{\vekt}[1]{\mathbf{#1}}
\newcommand{\Rmn}{\ensuremath{\R^{m\times n}}}
\DeclareMathOperator{\grad}{grad}
\newcommand{\vfi}{\varphi}
\newenvironment{spec}{\begin{trivlist}\item\relax\mbox{\textbf{Spec. esetek.\enskip}}\ignorespaces}{\end{trivlist}}
\newtheorem{lemma}{lemma}[section]
\newtheorem{PlSS}{PÉLDA}[subsubsection]
\DeclareMathOperator{\sgn}{sgn}
\DeclareMathOperator{\arctg}{arctg}
\newcommand{\Der}{\D}
\newcommand{\Oint}{\oint\limits}
\DeclareMathOperator{\Rint}{R}
\newcommand{\listazjbetu}{
  \renewcommand{\theenumi}{\alph{enumi}}
  \renewcommand{\labelenumi}{(\theenumi)}
}
\newcommand{\listazjromai}{
  \renewcommand{\theenumi}{\alph{enumi}}
  \renewcommand{\labelenumi}{(\theenumi)}
}
\newcommand{\listabetu}{
  \renewcommand{\theenumi}{\alph{enumi}}
  \renewcommand{\labelenumi}{\theenumi}
}
\newcommand{\listaszamkor}{
  \renewcommand{\theenumi}{\alph{enumi}}
  \renewcommand{\labelenumi}{\theenumi$^\circ$}
}
\newenvironment{enumzjromai}{\listazjromai\begin{enumerate}}{\end{enumerate}}
\newenvironment{enumzjbetu}{\listazjbetu\begin{enumerate}}{\end{enumerate}}

\newenvironment{enumzjr}{\begin{enumzjromai}}{\end{enumzjromai}}
\newenvironment{enumzjb}{\begin{enumzjbetu}}{\end{enumzjbetu}}


\DeclareRobustCommand{\tmspace}[3]{%
  \ifmmode\mskip#1#2\else\kern#1#3\fi\relax}
\providecommand*{\negmedspace}{\tmspace-\medmuskip{.2222em}}
%
% written by Claudio Beccari
% published in TUGboat, Volume 18 (1997), No. 1, 39--48
%
% slightly modified by FW
\makeatletter
\providecommand*{\diff}%
  {\@ifnextchar^{\DIfF}{\DIfF^{}}}
\def\DIfF^#1{%
  \mathop{\mathrm{d{}}}%%% original version: {\mathstrut d}}%
    \nolimits^{#1}\gobblespace}
\def\gobblespace{%
  \futurelet\diffarg\opspace}
\def\opspace{%
  \let\DiffSpace\negmedspace%%% original version: \,
  \ifx\diffarg(%
    \let\DiffSpace\relax
  \else
    \ifx\diffarg[%
      \let\DiffSpace\relax
    \else
      \ifx\diffarg\{%
        \let\DiffSpace\relax
      \fi\fi\fi\DiffSpace}
\makeatother

\providecommand*{\deriv}[3][]{%
    \frac{\diff^{#1}#2}{\diff #3^{#1}}}


\title{Analízis 4 előadás jegyzet (2004/2005)\\(Többváltozós analízis)}
\author{Tóth László Attila (panther@elte.hu)}
\date{}
\begin{document}
  \maketitle
  \mktoc
  \section{Metrikus terek}
\subsection{Környezetek, korlátos halmazok}

\begin{de}[Környezet]
  $(M,\varrho)$ MT, $a\in M,\ r>0$\\
  $K_r(a) := K_r^\varrho (a) := \{x\in M: \varrho(a,x) < r\}$
  az \emph{$a$ $r$ sugarú környezete} vagy \emph{$a$ középpontú
    r-sugarú gömb}.
\end{de}

Példa:\\
$M=\R^2,\ a=(0,0)\ r=1$.\\
$\di K_r^{\varrho_1}(a) :=\{(x,y)\in\R^2 : |x| + |y| < 1\}$\\
$K_r^{\varrho_2}(a) :=\{(x,y)\in\R^2 : \sqrt{x^2 + y^2} < 1\}$\\
$K_r^{\varrho_\infty}(a) :=\big\{(x,y)\in\R^2 : \max\{ |x| + |y|\} < 1\big\}$\\


\begin{de}
  $(M, \varrho)$ MT, $A\subset M$: korlátos, ha
  $\exists a \in M$ és $\exists r > 0: A\subset K_r^\varrho (a)$
\end{de}

\begin{te}
  $(M, \varrho)$ MT; $A\subseteq M$ korlátos $\Leftrightarrow
  \forall b \in M$-hez $\exists R>0\colon A\subseteq K_R^\varrho (b)
  $
\end{te}
\begin{biz}
  Lásd gyak
\end{biz}

\subsection{Ekvivalens metrikák}
\begin{de}[Ekvivalens metrikák]
  $(M,\varrho_1)$ és $(M,\varrho_2)$ MT.\\
  A $\varrho_1$ és $\varrho_2$ metrikák ekvivalensek, ha
  \[\exists c_1,c_2>0\colon
  c_1\varrho_2(x,y)\leq \varrho_1(x,y)\leq c_2\varrho_2(x,y)\qquad\qquad(x,y\in M)\]
  $Jel: \varrho_1\sim\varrho_2$\\Ill:
  \[\dfrac1{c_2}\varrho_1(x,y)\leq\varrho_2(x,y)\leq \dfrac1{c_1}\varrho_1(x,y)\]
\end{de}

\begin{te}
  A $\sim$ ekvivalenciareláció
\end{te}
\begin{biz}
  Trivi
\end{biz}

\begin{te}
  Ha $(M, \varrho_1),\,(M, \varrho_2)\ MT,\ \varrho_1\sim\varrho_2$
  AKKOR:  \begin{enumerate}[i)]
  \item $\forall a \in M\  \forall r_1>0\  \exists r_2>0\colon
    K_{r_2}^{\varrho_2}\subset K_{r_1}^{\varrho_1}$
  \item Ugyanez igaz fordítva is
  \end{enumerate}
\end{te}
\begin{biz}
  Legyen: $r_2:=\dfrac{r_1}{c_1}$
\end{biz}

\begin{te}
  Az $\R^n\ \ (n\in\N)$ halmazon értelmezett $\varrho_p\ \ (1\leq
  p\leq+\infty)$ metrikák ekvivalensek.
\end{te}
\begin{biz}
  Azt kell belátni, hogy $\ro_1\sim\ro_2\colon c_1\ro_1\leq\ro_2\leq c_2\ro_1$.
  Elég megmutatni, hogy $\ro_p\sim\ro_\infty\quad\forall p\in[1,+\infty)$, mivel $\sim$ tranzitív.
    \begin{gather*}
      \ro_p=\left(\sum_{k=1}^n\vert x_k-y_k\vert^p\right)^{\frac1p}\quad p\in[1,+\infty)\\
	\ro_\infty(x,y)=\max_{1\leq k\leq n}\vert x_k-y_k\vert\\
	\ro_\infty(x,y)\leq\ro_p(x,y)\text{ trivi, hiszen a legnagyobbat hagyjuk csak meg.}\\
	\ro_p(x,y)\leq\ro_\infty(x,y)\cdot\left(\sum_{k=1}^n1^p\right)^{\frac1p}=n\ro_\infty(x,y)
    \end{gather*}
\end{biz}

\begin{te}
  A $C[a,b]$-ben a $\varrho_1$ és $\varrho_\infty$ metrikák
  \emph{nem} ekvivalensek.   
\end{te}

\begin{biz}
  Ld gyak, F24.
\end{biz}

\subsection{Konvergens sorozatok MT-ekben. Teljes MT-ek}
Eml:  $\R$-beli konvergens sorozat.

\begin{de}[Konvergens sorozat]
  Az $(M, \varrho)$ MT egy $(a,n)\colon \N\n M$ sorozata konvergens,
  ha $\exists \alpha \in M\!\colon \forall \epsilon>0$-hoz $\exists
  n_0\in \N \ \forall n\geq n_0\colon\ a_n \in
  K_\epsilon^\varrho(\alpha)$\\
  Egy sorozat \emph{divergens}, ha nem konvegens.
\end{de}

\begin{Megj}
\item $(\R,\varrho)$ a szokásos metrikával: a ``régi'' definíciót
  kapjuk.
\item A definíció igen általános.
\item     $\alpha$ az $(a,n)$ \emph{határéréke}. Jel: $\lim (a_n)
  \eqrho\alpha$; $\underset{(n\n\infty)}{a_n\xrightarrow{\varrho} \alpha}$
\end{Megj}

\begin{te}
  Ha $\exists$ ilyen $\alpha\in M$, akkor az egyértelmű.
\end{te}
\begin{biz}
  Mint \R-ben: Indirekt feltevés: nem egyértelmű,
  $\alpha,\overline{\alpha}\in M$ ilyen.\\
  Legyen $0<r<\dfrac{\varrho(\alpha,\overline{\alpha})}2$.
  $K_r(\alpha)\cap K_r(\overline{\alpha})=\ures $ (ok: F15)
\end{biz}

\begin{te}(A konvergencia átfogalmazása)
  \MT MT.\\
  $\alpha \eqrho \lim(a_n)$ ekvivalens a
  következőkkel (bármelyikkel) :
  \begin{enumerate}[\ i)]
  \item $\forall \epsilon >0$-hoz $\exists n_0\in \N,\ \forall
    n>n_0\colon \varrho(\alpha,a_n)<\epsilon$
  \item $\forall \epsilon>0$ esetén $\{\,n\in\N\colon a_n\not\in
    K_\epsilon(\alpha)\,\}$ véges halmaz.
  \item $\di \lim_{n\n +\infty} \varrho(a_n,\alpha) = 0$
  \end{enumerate}
\end{te}
\begin{biz} Trivi. Hf.
\end{biz}

\begin{te}(Konvergens sorozatok tulajdonságai) \MT MT\\
  $(a_n)\colon \N\n M$ konvergens; $\alpha \eqrho \lim(a_n)$. EKKOR
  \begin{korlista}
  \item Az $(a_n)$ korlátos sorozat, azaz az $R_{(a_n)}$ halmaz
    korlátos.
  \item $\forall (\nu_n)\colon \N\n\N\ \uparrow$ (indexsorozat)
    esetén az $(a_{\nu_n})$ részsorozat konvergens és $\alpha$ a
    határértéke.
  \item Ha $(a_n)$-nek van két különböző $M$-beli értékhez tartozó
    résszorozata \nn $(a_n)$ divergens.
  \end{korlista}
\end{te}

\begin{biz}Mint \R-ben.
\end{biz}
\begin{Megj}
\item MT-ben nincsen rendezés, nincsen művelet, így nincs pl
  közrefogási elv sem.
\item A határérték, konvergencia metrikafüggő!
\end{Megj}

\begin{te}[Ekvivalens metrikák \nn konvergens sorozatok]\MTn1, \MTn2
  MT, \\$\varrho_1\,\sim\,\varrho_2$. Ekkor\\
  $\forall\,(a_n): \N\n M$ sorozatra: $\di\lim_{n\n+\infty}(a_n)
  \stackrel{\varrho_1}{=} \alpha \Leftrightarrow
  \di\lim_{n\n+\infty}(a_n) \stackrel{\varrho_2}{=} \alpha$\\
  Azaz ekvivalens metrikákban ugyanazok a konvergens sorozatok   
\end{te}

\begin{biz}
  $c_1 \ro_2 \leq \varrho_1\leq c_2\ro_2$ és
  $\underset{(n\n+\infty)}{a_n\overset{\di\ro_1}{\n}\alpha}$\\
  \nn\ $\di\lim_{n\n+\infty}\ro_1(a_n,\alpha)=0$ nullasorozat\\
  \nn\ $\ro_2(a_n,\alpha)\leq \dfrac1{c_1}\ro_1(a_n,\alpha)
  \nn \underset{(n\n+\infty)}{a_n\overset{\di\ro_2}{\n}\alpha}$ 
\end{biz}

Példa: Fordítva nem igaz:\\
Ha \MTn1, \MTn2 MT-ben ugyanazok a konvergens sorozatok $\not\nn\,
\ro_1\sim\ro_2$.\\
Pl: $M:=\{\,(a_n):\N\n\N\,\}$; $\ro_1$ diszkrét, $\ro_\infty$ a max.\\

%\underline{Teljesség}\\
\subsubsection{Teljesség}
Eml: \R-beli Cauchy-konvergencia-kritérium  
\begin{de}[Cauchy-sorozat]
  \MT MT; az $(a_n): \N\n M$ C-sorozat, ha 
  \[\forall \epsilon >0\, \exists n_0\in \N\, \forall n,m\geq
  n_0\colon \ro(a_n,a_m)<\epsilon\]
\end{de}
\begin{megj}
  \R-ben mondtuk: a nagy indexű tagok közel vannak egymáshoz.
\end{megj}

\begin{te}
  \MT MT, \sorozat
  \begin{korlista}
  \item Ha $(a_n)$ konvergens \nn\ $(a_n)$ Cauchy-sorozat
  \item visszafele \underline{nem} igaz
  \end{korlista}
  (azaz  MT-ben a Cauchy-kritérium nem igaz)
\end{te}

\begin{Biz}
\item $\lim(a_n)\eqrho\alpha\ \nn \ \forall \epsilon >0 \ \exists
  N\in\N\ \forall n>N\colon \ro(a_n,\alpha)<\epsilon$\\
  $\ro(a_n,a_m)\leq\ro(a_n,\alpha) +
  \ro(\alpha,a_m)<2\epsilon\,\nn\,(a_n)$ Cauchy-sorozat
\item Igazolni kell: $\exists \MT$ MT, $\exists (a_n)$ Cauchy-sorozat,
  ami nem konvergens.\\
  Pl $\MT := (\Q,\ro)$ ahol $\ro$ a szokásos metrika.\\
  $a_0 := 2;\ a_{n+1}:= \dfrac12(a_n+\dfrac2{a_n})\quad(n\in\N)$
  Ez egy racionális
  Cauchy-sorozat, ami nem konvergens.
\end{Biz}
\begin{de}[Teljesség]
  Az \MT MT teljes, ha teljesül  a Cauchy konvergencia-kritérium
\end{de}

\subsubsection{Konvergencia a ``nevezetes'' MT-ekben.}

1) Diszkrét MT: \MT\\
\sorozat konv és $\lim(a_n)\eqrho\alpha \Leftrightarrow \exists
N\in\N \ \forall n\geq N\colon a_n=\alpha $ (``kvázikonstans''
sorozat)

\begin{biz}
  $\forall \epsilon \in (0,1)\colon K_\epsilon(\alpha)={\alpha}$
\end{biz}

\begin{te}
  A diszkrét metrikus tér teljes.
\end{te}
\begin{biz}Hf
\end{biz}


2) $(\R^n,\ro_p)\quad 1\leq p\leq+\infty$

\begin{te}
  $n\in\N,1\leq p\leq+\infty$. Tekintsük a $(\R^n,\ro_p)$ MT-ben az
  \sorozatn{k}\ sorozatot, ahol
  $a_k=(a_k^{(1)},a_k^{(2)},\ldots,a_k^{(n)})$. Ekkor:
  \[(a_k) \text{ konvergens és}\]
  \[\di\lim_{k\n+\infty}a_k\stackrel{\ro_p}{=}\alpha =
  (\alpha^{(1)},\,\alpha^{(2)},\ldots,\,\alpha^{(n)})\in\R^n
  \Longleftrightarrow\]
  \[\forall i=1..n\ (a_k^{(i)})_{k\in\N}
  \text{ valós sorozat (i. koordinátasorozat) konvergens és
  } \lim_{k\n+\infty}a_k^{(i)}\stackrel{\ro_p}{=}\alpha^{(i)}\]
\end{te}

\begin{biz}
  Mivel a $\ro_p$-k ekvivalensek, ezért elegendő pl
  $\ro_{\infty}$-re $(p=+\infty)$.\\
  $\underline{\nn:}\ \ i=1..n;\
  \underset{\underset{0}{\downarrow}}{|a_k^{(i)} - \alpha^{(i)}}|
  \leq \underset{k\n+\infty}{\ro_\infty
    (a_k,}\underset{\underset0\downarrow}{\alpha)} \quad(\forall k\in\N)$\\
  $\underline{\Leftarrow:}\ $ Ha $\forall i=1..n\
  \lim(a_k^{(i)})=\alpha^{(i)} \nn \forall \epsilon >0\ \exists k_i\in
  \N\colon \ |a_k^{(i)}-\alpha^{(i)}|<\epsilon \quad \forall k\geq k_i$\\
  Ekkor $\di\max_{1\leq i \leq n} |a_k^{(i)} - \alpha^{(i)}| =
  \ro_\infty(a_k,\alpha) < \epsilon $. Ha $k\geq k_0:= max\{k_1,\ldots
  k_n\} \nn$ az állítás.
\end{biz}

\begin{te}
  $n\in\N;\ 1\leq p\leq +\infty$ esetén $(\R^n,\ro_p)$ MT teljes.
\end{te}
\begin{biz}
  Ld. gyak
\end{biz}

\subsubsection[Konvergens függvénysorozatok]{$(C[a,b],\ro_p)$ MT
  konvergens  sorozatai\\ (Függvénysorozatok I.)}
Megj: várható, hogy különböző metrikákban különböznek a konvergens
sorozatok.
\begin{de}\ 
  \begin{enumerate}[\quad(a)]
  \item Az $(f_n)\colon \N\n C[a,b]$ függvénysorozat konvergens a
    $(C[a,b],\ro_p)\\(1\leq p\leq\infty)$ MT-ben, ha  $\exists f\in
    \di C[a,b]\colon \lim_{n\n+\infty}\ro_p(f_n,f)=0$
  \item $p=+\infty$ esetén azt mondjuk, hogy az $(f_n)$ a
    \underline{maximum-metrikában} (v. \underline{egyenletesen az }
    \underline{$[a,b]$-n tart} az $f\in C[a,b]$ fv-hez, azaz:\\
    \[f_n \xrightarrow[n\n\infty]{\ro_p} f \ekv \forall\epsilon>0\
    \exists n_0\in\N\colon \forall n>n_0\  \forall  x\in[a,b]\colon
    |f_n(x)-f(x)| < \epsilon\]
  \item $p=1$ esetén azt mondjuk, hogy az $(f_n)$ az (egyes)
    \emph{integrál-metrikában} tart az $f$-hez, azaz:
    \[f_n \xrightarrow[n\n\infty]{\ro_1} f \ekv \forall\epsilon>0\
    \exists n_0\in\N\colon \forall n>n_0\ \colon \di\int\limits_a^b
    |f_n-f| < \eps\]

  \end{enumerate} 
\end{de}

\begin{megj}
  Sejthető , hogy  a kettő különböző.
\end{megj}
\begin{te}
  Ha  $(f_n)$ egyenletesen tart $f$-hez $[a,b]$-n, akkor $f_n$
  pontonként is tart az \\$\fcab$ folytonos függvényhez.
\end{te}
\begin{biz}
  Trivi.
\end{biz}
\begin{pl}
  $f_n(x) := x^n\quad x\in[0,1]\ n\in \N$
  pontonként tart a konstans nulla fv-hez. $f\not\in C[0,1] \nn f$
  nem egyenletesen konvergens azaz nem konvergens a
  $(C[a,b],\ro_\infty)$-ben, DE! $(C[0,1],\ro_1)$-ben tart a 0-hoz.
\end{pl}

\begin{te}
  Adott $(f_n)\colon \N\n C[a,b]$ sorozat:\\
  $f_n\xrightarrow[n\n\infty]{\ro_\infty} f\nn\ f_n
  \xrightarrow[n\n\infty]{\ro_1} f$\\
  DE fordítva NEM igaz.
\end{te}
\begin{biz}
  $\di\int_a^b|f_n-f| \leq (\max|f_n-f|)\int_a^b1 =
  (b-a)\max|f_n-f|$\\
  Az ellekező irányra ellenpélda lásd fenn, vagy:\\\\
  \unitlength 1mm
  \begin{picture}(30,30)(-5,-5)
    \thinlines
    \put(-2,0){\vector(1,0){30}}
    \put(0,-2){\vector(0,1){30}}
    \thicklines
    \put(10,0){\line(-2,5){10}}
    \put(10,0){\line(1,0){15}}
    \put(-5,24){$n$}
    \put(8,-4){$\frac1{n^2}$}
    \put(23,-3){$1$}
  \end{picture}
\end{biz}

\begin{te}\ \\
  (a) $C([a,b],\ro_\infty)$ teljes MT\\
  (a) $C([a,b],\ro_1)$ nem teljes MT
\end{te}
\begin{biz}
  Ld. gyak.
\end{biz}
\begin{te}[Bolzano-Weiserstrass-féle kiválasztási tétel] (azaz
  $\forall$ korlátos sorozatnak $\exists$ konvergens részsorozata)
  \begin{enumerate}[\quad(a)]
  \item $(\R^n,\ro_p)\ (n\in\N,\,1\leq p \leq +\infty)$ igaz
  \item \MT tetszőleges MT-ben azonban nem igaz
  \end{enumerate}
\end{te}
\begin{biz}
  (a) (vázlat) koordinátasorozatokkal\\
  (b) $\MT := (\N,\ro_d)$ diszkrét MT, $a_n := n\ (n\in\N)$ korlátos
  sorozat, de nincs konvergens részsorozata.
\end{biz}

\subsection{Topológiai fogalmak MT-ekben}
\begin{de}[Torlódási pont]
  \MT MT; $A\subset M$; az $a\in M$ az A torlódási pontja, ha
  $\forall \epsilon >0\  \forall K(a)$-ra$\colon (K(a)\setminus\{a\})
  \cap A \neq \ures$\\
  Jel: $A'\colon$ torlódási pontok halmaza
\end{de}

\begin{te}
  \MT MT, $A\subset M$,\\
  $a\in A' \ekviv \forall K(a)$-ra: $K(a)\cap A$ végtelen
  halmaz\\
  $a\in A' \ekviv \exists (a_n)\colon \N\n A$ injektív sorozat, hogy
  $\lim(a_n)=a$
\end{te}
\begin{de} \MT MT, $A\subset M$
  \begin{enumerate}
  \item $a\in A$ az $A$ \emph{belső pont}ja, ha $\exists K(a)\colon
    K(a)\in A$
  \item $a\in A$ az $A$ \emph{izolált pont}ja, ha $\exists
    K(a)\colon  (K(A)\setminus\{a\}) \cap A = \ures$
  \item $a\in M$ az $A$ \emph{határpont}ja, ha $\forall K(a)$
    esetén $K(a)\cap A \neq \ures$ és $K(a)\cap (M\setminus
    A) \neq \ures$
  \end{enumerate}
\end{de}

\begin{de} \MT MT, az $A\subset M$ halmaz
  \begin{enumerate}
  \item \emph{nyílt halmaz} az \MT-ban,ha $\forall$ pontja
    belső pont.
  \item \emph{zárt halmaz} az \MT-ban,ha $M\setminus A$ nyílt
  \item $\overline{\!A}=A\cap A'$ az \emph{$A$ lezárása}
  \end{enumerate}
\end{de}
\newpage%%%%%%%%%%%%%%%%%%%%%%%%%%%%%%%%%%%%%%%%%%%%%%%%%%%%%%%%%%%%%%%%%%%%%%
Például:
\begin{itemize}
\item $M,\ \ures$ nyílt és zárt
\item $K(a)$ nyílt halmaz
\item Véges sok pontból álló halmaz zárt
\item \MT-ban $\exists A$: sem nem nyílt, sem nem zárt
\end{itemize}

\begin{te}[Zárt halmazok jellemzése]\MT MT; $A\subset M$\\
  A következő állítások ekvivalensek:%
  \begin{enumerate}
  \item Az $A$ zárt \MT MT-ben
  \item $A'\subset A$
  \item $\forall (a_n):\N\n A$ konvergens sorozat esetén
    $\lim(a_n)\in A$  
  \end{enumerate}
\end{te}
\begin{biz} Mint \R-ben
\end{biz}

\begin{megj}
  Zártság-nyíltság relatív fogalmak: függnek az \emph{altér}től
  (részhalmaz\ldots), Pl\
  $A := (0,1) \in \R$ nyílt halmaz\\
  $A := (0,1) \in \R^2$ nem nyílt halmaz\\
\end{megj}

\begin{te}
  \MT MT; $A_i\subset M\quad (i\in\Gamma)$ nyílt halmazok
  \begin{enumerate}
  \item $\di\bigcup_{i\in\Gamma} A_i$ nyílt halmaz
  \item Ha $\Gamma_0$ véges $\nn \di\bigcap_{i\in\Gamma_0} A_i$ nyílt.
  \end{enumerate}
\end{te}

\begin{te}
  \MT MT; $A_i\subset M\quad (i\in\Gamma)$ zárt halmazok
  \begin{enumerate}
  \item $\di\bigcap_{i\in\Gamma} A_i$ zárt halmaz
  \item Ha $\Gamma_0$ véges $\nn \di\bigcup_{i\in\Gamma_0} A_i$ is
    zárt.
  \end{enumerate}
\end{te}

\begin{megj}A végesség lényeges.
\end{megj}

\subsubsection{Kompakt halmazok}

\begin{de}\MT MT. Az $A\subset M$ kompakt, ha $\forall (a_n)\colon
  \N\n A$ sorozathoz\\ $\exists (\nu_n)\colon \N\n\N\ \uparrow$
  indexsorozat, hogy: $a_{\nu_n}$ részsorozat konvergens és
  $\lim(a_{\nu_n})\in A$
\end{de}

\begin{megj}
  vö zártsággal $\nn$ (trivi) $\forall$ kompakt halmaz zárt
\end{megj}

\begin{de}\MT MT. Az $A\subset M$ halmaz \emph{nyílt lefedése}:\\
  $\{G_i\subset M \mid  i\in \Gamma ($tetsz. indexhalmaz$,\ \ G_i
  \neq \ures\ G_i$ nyílt \}\\
  amire: $\di A\subset \bigcap_{i\in\Gamma}G_i$
\end{de}

\newpage %%%%%%%%%%%%%%%%%%%%%%%%%%%%%%%%%%%%%%%%%%%%%%%%%%%%%%%%%%%%%%%%%%%
\begin{te}[A kompaktság ekvivalens jellemzései]\MT MT-ben a
  következő állítások ekvivalensek:
  \begin{enumerate}
  \item $A\subset M$ kompakt
  \item BOREL-féle lefedési tétel:\\
    Az $A$ \underline{minden} nyílt lefedéséből kiválasztható
    egy véges lefedőrendszer, azaz: $\forall{G_i : i\in \Gamma}$
    nyílt lefedése esetén $\exists \Gamma_0\subset\Gamma$ véges:
    $\di A\subset\bigcup_{i\in\Gamma_0}G_i$
  \item Az $A$ minden  végtelen részhalmazának van torlódási pontja
  \end{enumerate}
\end{te}

\begin{megj}
  A kompaktság abszolút fogalom, nem függ az altértől. \MT MT,
  $\hullam{M}\subset M$,
  $(\hullam{M},\ro_{|\hullam{M}\times\hullam{M}})$ az \MT egy
  \underline{altere}. $A\subset \hullam{M}$.\\
  $A$ kompakt \MT-ban $\ekviv\ A$ kompakt
  $(\hullam{M},\ro_{|\hullam{M}\times\hullam{M}})$-ban.
\end{megj}

\begin{te}[Kompakt halmazok tulajdonságai] \MT MT, az $A\subset M$
  kompakt. Ekkor:
  \begin{enumerate}
  \item az $A$ zárt \MT-ban
  \item az $A$ korlátos \MT-ban
  \item Ha $A$ korlátos és zárt  $\not\nn$ $A$ kompakt
  \end{enumerate}    
\end{te}
\begin{biz}
  1) trivi; 2) indirekt.\\
  3) Pl $\MT := (l_2,\,\ro_2)\quad A := \{ e^{(k)} \in l_2\, |\, e^{(k)} :=
  (0,\,\ldots\,,\,0,\, \overset{k}{\check{1}},\,0,\,\ldots\,,0),\
  k\in \N\,\}\\\nn\ \ro_2(e^{(k)},\, e^{(l)}) = \sqrt2$\quad Ekkor $A$
  korlátos és zárt, de nem kompakt - a definíció alapján
\end{biz}

\subsubsection{$\R^n$ kompakt részhalmazai}
$\R^n$-ben zártság és korlátosság a kompaktságnak nem csak
szükséges, de elégséges feltétele is.

\begin{te}
  $n\in\N,\ 1\leq p\leq +\infty$. Ekkor\\
  $A\in\R^n$ kompakt $\ekviv\ A$ korlátos és zárt
\end{te}

\begin{biz}
  \underline{\nn:} általában is igaz;\\
  \underline{$\Leftarrow$:} Bolzano-Weierstrass-kiválasztási tétel
  használható
\end{biz}

\subsection{Folytonosság (Metrikus terek közötti függvények)}
\begin{de}
  $(M_1,\ro_1),\,(M_2,\ro_2)$ MT-ek; $f\in M_1\n M_2$; $a\in D_f$.\\
  Az $f$ folytonos az $a\in D_f$ pontban (jel: $f\in C\{a\}$), ha
  \[\forall \epsilon>0\ \exists\delta>0\ \forall x\in D_f:\
  \ro_1(a,x)<\delta:\ \ro_2(f(x),f(a))<\epsilon\]
  másképp: $\ldots x\in K_\delta^{\ro_1}(a)\cap D_f:\ f(x)\in
  K_\epsilon^{\ro_2}(f(a))$
\end{de}

\begin{megj} $\R\n\R;\ f\in C\{a\}, a\in D_f$\\
  Megmarad az $\R\n\R$-beli lényeg: az ``$a$-hoz közeli pontokban a
  fv-értékek $f(a)$-hoz vannak közel'' (``közelség'' a megfelelő
  metrikában)
\end{megj}

\begin{te}[Átviteli elv]
  $(M_1,\ro_1),\,(M_2,\ro_2)$ MT-ek; $f\in M_1\n M_2$; 
  $a\in D_f$\\
  $f\in C\{a\} \ekviv \forall (x_n)\colon  \N\n D_f,\ \lim(x_n)
  \eqrhon1 a$ esetén $\lim( f(x_n))\eqrhon2 f(a)$.\\
  
\end{te}
\begin{biz} Mint $\R\n\R$ esetén.
\end{biz}

\begin{de}[Határérték]
  $(M_1,\ro_1),\,(M_2,\ro_2)$ MT-ek; $f\in M_1\n M_2$; $a\in
  D_f'$.\\
  Az $f$ fv-nek az $a\in D_f'$ pontban van határértéke, ha
  $\exists  A \in M_2$ melyre\\
  $\tilde{f}(x) = \Big\{\begin{array}{cc}
  f(x) & x\in D_f\setminus\{a\}\\
  A & x=a\end{array}$ folytonos $a\in M_1$-ben\\
  Jel: $\di\lim_{x\xrightarrow{\ro_1}a}f(x) \stackrel{\ro_2}{=} A$
  vagy $\di\lim_af=A$    
\end{de}


\begin{te} $(M_1,\ro_1),\,(M_2,\ro_2)$ MT-ek; $f\in M_1\n M_2$; $a\in
  D_f'$.\\Ha $\exists (x_n^{(1)}),\,(x_n^{(2)})\colon \N\n
  D_f\setminus\{a\}:  \lim(x_n^{(1)})=\lim(x_n^{(2)}) = a$,\\ de
  $\lim(f(x_n^{(1)}))=\lim(f(x_n^{(2)})) \nn \nexists \di\lim_af$
\end{te}
\begin{biz}
  Átviteli elv + indirekt.
\end{biz}

\begin{te}[Folytonosság, ekvivalens metrikák]
  $(M_1,\ro_1),(M_1,\tilde{\ro}_1)$ MT-ek, $\ro_1\sim\tilde{\ro}_1$ és
  $(M_2,\ro_2),(M_2,\tilde{\ro}_2)$ MT-ek,
  $\ro_2\sim\tilde{\ro}_2$. Ekkor
  $f\in (M_1,\ro_1)\n)(M_2,\ro_2)$ folytonos $a\in D_f$-ben $\ekviv
  f\in (M_1,\ro_1)\n)(M_2,\ro_2)$ folytonos $a\in D_f$-ben
\end{te}
\begin{biz}
  trivi
\end{biz}

\begin{te}[Kompozíció folytonos] $(M_i,\ro_i)\ i=1..3$ MT-ek,
  $f\in M_1\n M_2,\\g\in M_3\n M_2 \ (\,f\circ g\in M_3\n M_2\,)$,
  $g\in C\{a\}$ és $f\in C\{g(a)\}\nn f\circ g\in C\{a\}$    
\end{te}

\subsubsection{$\R^n\n \R^n$ függvények folytonossága}
\begin{de}
  $f\in \R^n\n \R^m$ fv folytonos az $a\in D_f$ pontban, ha\\ $f\in
  (\R^n,\ro^{(1)})\n(\R^m,\ro^{(2)}) $ MT-ek közötti leképezés
  folytonos az $a\in D_f$ pontban és $\ro^{(1)}$ tetszőleges $\ro_p$
  metrika $\R^n$-ben, ill $\ro^{(2)}$ $\R^m$-ben.
\end{de}
\begin{te}
  $f\in \R^n\n \R^m$, $f = \left(
  \begin{array}{c}
    f_1\\\vdots\\f_n\end{array}\right)$  ahol $f_i\in
    \R^n\n\R^1\quad (i=1..m)$ koordinátafüggvények.\\
    $f\in C\{a\} \ekviv \forall i=1..m$ $f_i\in C\{a\}$\\
    azaz elég a koordinátafüggvények folytonossága
\end{te}
\begin{te}[Műveletek]
  \begin{enumerate}
  \item Kompozíció
  \item $f,g\in \R^n\n\R^n,\ a\in D_f\cap D_g$. Ha $f,g\in C\{a\}$,
    akkor $f+g,\, \lambda f\in C\{a\}$
  \item Ha $f,g\in \R^n\n\R^1,\ a\in D_f\cap D_g,\ f,g\in
    C\{a\}\nn f+g,\,\lambda f,\, fg,\, \frac f g\in C\{a\}$ -- hányados
    esetén ha $g(a) \neq 0$
  \end{enumerate}
\end{te}


\subsubsection{Halmazon vett folytonosság}
\begin{de}
  $(M_1,\ro_1),\,(M_2,\ro_2)$ MT-ek, $f\in M_1\n M_2$, $A\subset
  D_f$.\\
  Az $f$ fv folytonosaz $A\subset D_f$ halmazon, ha $\forall a\in
  A\colon f\in C\{a\}$\\
  Jel: $f\in C(A)$. (globálos folytonosság)
\end{de}

\begin{te}[Globális folytonosság jellemzése nyílt halmazokkal]
  $(M_1,\ro_1),\,(M_2,\ro_2)$ MT-ek és $f\in M_1 \underline{=D_f}\n
  M_2$.\\
  $f$ folytonos $M_1$-en $\ekviv$ $\forall B\subset M_2$ nyílt
  halmaz esetén $f^{-1}[B]\subset M_1$ is nyílt halmaz\\
  (a $B$ halmaz $f$ által létesített ősképe)
\end{te}
\begin{biz}
  \underline{\nn:} Tfh. $f$ folytonos $M_1\,(=D_f)$-en; $B\subset
  M_2$ nyílt halmaz. Legyen  $a\in f^{-1}[B]$ (ha $f^{-1}[B]=\ures$
  akkor kész)
  \nn $f(a) \in B$; $B$ nyílt $\nn \exists \epsilon>0\
  K_\epsilon^{\ro_2} \left(f(a)\right)\subset B$\\
  DE! $f\in C\{a\}\nn \epsilon>0$-hoz $\exists\delta>0\colon x\in
  K_\delta^{\ro_1}(a): f(x) \in K_\epsilon^{\ro_2}
  \left(f(a)\right)$\\
  azaz $K_\delta^{\ro_1}(a)\subset f^{-1}[B]\nn
  f^{-1}[B]$ nyílt.\\
  $\underline{\Leftarrow:}$ Igazolni kell: $\forall a \in M_1$
  folytonos: $\forall \epsilon > 0\ \exists \delta > 0\colon\
  \forall x\in  K_\delta^{\ro_1}(a)\cap M_1$ esetén\\
  $f(x)\in K_\epsilon^{\ro_2}\left(f\left(a\right)\right)$.\\
  Legyen: $a\in M_1$ rögzített; $\epsilon > 0$; tekintsük:
  $K_\epsilon^{\ro_2}(f(a))\subset M_2$ nyílt halmaz, 
  $\stackrel{\text{feltétel}}{\nn}\\f^{-1}[K_\epsilon^{\ro_2}(f(a))]
  \subset M_1$ nyílt halmaz $\nn \exists K_\delta^{\ro_1}(a)\colon
  K_\delta^{\ro_1}(a)\subset f^{-1}[K_\epsilon^{\ro_2}(f(a))] \nn\\
  f[K_\delta^{\ro_1}(a)]\subset K_\epsilon^{\ro_2}(f(a)) $
\end{biz}

\begin{megj}
  Nem elég: $B\subset R_f$ nyílt halmazokat venni, ui: $f\in\R\n\R; 
  \ f := sgn;\ R_f = \{-1,\, 0,\, 1 \};\ B\subset R_f$ nyílt $\nn
  B=\ures$ (csak ez lehet). $f^{-1}[\ures] =\ures $ nyílt, de $f$
  nem folytonos
\end{megj}

\begin{te}[Általánosítás]
  $(M_1,\ro_1),\,(M_2,\ro_2)$ MT-ek és $f\in A\n M_2$, $A\subsetneqq
  M_1$.\\ 
  $f$ folytonos $A$-n $\ekviv$ $\forall B\subset M_2$ nyílt
  halmaz esetén $\exists G\subset M_1$ nyílt: $f^{-1}[B]= A\cap G$ 
\end{te}

\begin{megj}
  $F$ nyílt halmaz az $(A, \ro_{|A\times A}$ altéren $\ekviv \exists
  G\subset M_1$ nyílt halmaz az $(M_1,\ro_1)$ MT-ben: $F = A \cap G$ 
\end{megj}
\subsubsection{Kompakt halmazon folytonos függvények}

\begin{te}
  $(M_1,\ro_1),\,(M_2,\ro_2)$ MT-ek, tfh.
  \begin{enumerate}
  \item $A\subset M_1 kompakt$
  \item $f\colon A\n M_2$ folytonos A-n
  \end{enumerate}
  Ekkor
  \begin{enumerate}
  \item $R_f$ kompakt
  \item Ha $M_2=\R$ (valós értékű), akkor f felveszi a minimumát és a
    maximumát (Weiserstrass)
  \item Ha $f$ injektív, akkor $f^{-1}$ is folytonos
  \end{enumerate}
\end{te}
\begin{biz}
  mint $\R\n\R$-ben
\end{biz}

\begin{de}
  $(M_1,\ro_1),\,(M_2,\ro_2)$ MT-ek, $f\in M_1\n M_2$;
  $f$ egyenletesen folytonos az $A\subset D_f$ halmazon, ha $\forall
  \epsilon >0\ \exists \delta >0\ \forall x,y\in A \ro_1(x,y) < \delta
  \colon \ro_2(f(x), f(y)) < \epsilon$ 
\end{de}

\begin{te}
  $(M_1, \ro_1)\,(M_2,\ro_2)$ MT-ek, $f\in M_1\n M_2$
  \begin{enumerate}
  \item Ha $f$ egyenletesen folytonos $A\subset D_f$-en \nn $f$
    folytonos $A$-n.			
  \item Ha $A\subset D_f$ kompakt és $f$ folytonos $D_f$-en \nn $f$
    egyenletesen folytonos (Heine)					
  \end{enumerate}
\end{te}

\begin{de}
  $(M,\ro)$ MT,\begin{enumerate}
  \item  Az $A\subset M$ halmaz nem összefüggő, ha $\exists G_1,
    G_2\subset M$ nyílt halmaz, hogy\\
    $G_1\cap G_2=\ures;\
    G_1,G_2\neq\ures;\ (A\cap G_1) \cap (A\cap G_2) = \ures$ és
    $(A\cap G_1) \cup (A\cap G_2) = A$\\
    és $A\cap G_i \neq \ures\quad i=1,2$
  \item Az $A\subset M$ \emph{összefüggő}, ha az előbbi nem teljesül
  \end{enumerate}
\end{de}
\begin{pl}
  \begin{enumerate}[\quad(1)]
  \item $\R$-ben minden intervallum összefüggő
  \item (ábra)
  \item $(M,\ro)$-ban $K(a)$ összefüggő
  \item (ábra)
  \end{enumerate}
\end{pl}

\begin{te} 
  $\mr1,\,\mr2$ MT-ek, $\fmm$ folytonos $D_f$-en; $A\subset D_f$
  összefüggő.
  Ekkor $f[A]\subset M_2$ is összefüggő.\quad
  (öf halmaz folytonos képe is öf)
\end{te}

\begin{te}[spec: Bolzano-tétel]\MT MT, $f\in A\n\R$, $A\subset M$\\
  $\begin{array}{rl}
    (i) & A\subset M\ \text{öf}\\
    (ii) & f\ \text{folytonos}\ A\text{-n}
  \end{array} \Big\} \nn \forall c \in (f(a), f(b))(
  $ha $f(a) < f(b))\ \exists \xi\in A\colon f(\xi) = c$   
\end{te}

\begin{te}[Banach-féle fixpont-tétel]
  tfh. $\MT$ \underline{teljes} MT; $f\in M\n M$ kontrakció, azaz
  $\exists \alpha \in [0,1)\colon \ro( f(x),\, f(y)) \leq \alpha
    \ro(x,y) \quad (\forall x,y\in M)$.\\
    EKKOR \begin{enumerate}
    \item$\exists!\, x^*\in M\colon f(x^*) = x^*$ az $f$ fixpontja
    \item $x_0\in M\colon x_{n+1} = f(x_n)\ n\in \N$ iterációs
      sorozat konvergens és $\lim(x_n) = x^*$
    \item Hibabecslés: $\ro(x^*,x_n) \leq
      \dfrac{\alpha^n}{1-\alpha}\ro(x_0,x_1)\quad n\in \N$
    \end{enumerate}
\end{te}
\begin{megj}
  Fontos a teljesség és az, hogy $0\leqq \alpha < 1$
\end{megj}
\begin{biz}
  \begin{enumerate}
    \item $f$ kontrakció, ezért $f$ folytonos is, ugyanis
      \[\lim_{y\to x}f(y)=f(x),\text{ mivel }\ro( f(x),\, f(y)) \leq \alpha \ro(x,y),\ y\to x\]
    \item Igazoljuk, hogy $(x_n)$ Cauchy-sorozat:
      \begin{gather*}
	\ro(x_{n+1}-x_n)=\ro(f(x_n)-f(x_{n-1}))\leq\alpha\ro(x_n-x_{n-1})=\\
	\hspace*{2em}=\alpha \ro(f(x_{n-1})-f(x_{n-2})) \leq \alpha^2
	\ro(x_{n-1}-x_{n-2})\leq\dotsb\leq\alpha^n\ro(x_1-x_0)
	\intertext{Ebből}
	\ro(x_{n+m}-x_n)\leq \ro(x_{n+m}-x_{n+m-1})+ \ro(x_{n+m-1}-x_{n+m-2})+ \dotsb\\
	\hspace*{2em}\dotsb+\ro(x_{n+1}-x_n+)\leq\alpha^n\left(\alpha^{m-1}+\alpha^{m-2}+\ldots+\alpha^0\right)
	\ro (x_1-x_0)\leq\\\hspace*{2em}\underset{\alpha<1}{\leq} \dfrac{\alpha^n}{1-\alpha}\ro(x_0-x_1)
	\intertext{Ebből már következik, hogy $(x_n)$ Cauchy-sorozat, ui}
	\alpha^n\to0\quad(n\to+\infty)
      \end{gather*}
    \item $(x_n)$ Cauchy sorozat, ezért $(x_n)$ konvergens
      \begin{gather*}
	x^*=\lim(x_n)\\
	\begin{array}{c@{ }c@{ }c}x_{n+1}& =&f(x_n)\\\downarrow&&\downarrow\\
	  x^*&= &f(x^*)\end{array}
      \end{gather*}
      ui $f$ folytonos + átviteli elv (ezért $x^* = f(x^*)$) $\nn x^*$  fixpont.
    \item Egyértelműség: $x^*,x^{**}$ legyenek fixpontok.
      \[\ro(x^*-x^{**})=\ro(f(x^*)-f(x^{**}))\overset{f \text{ kontrakció}}{\leq}\alpha\ro(x^*-x^{**})
      \overset{\alpha<1}{<}\ro(x^*-x^{**})\]
      Tehát $x^*=x^{**}$.
    \item Hibabecslés: a 2. pontban ha $n=1,2,\dotsc\quad n\to+\infty$ határértéket vesszük
  \end{enumerate}

\end{biz}




\newpage
\section{Normált-, Banach-, Euklideszi-, Hilbert-terek}
\subsection{Lineáris terek (LT)}
1) Lineáris terek vagy vektorterek, ld. linalg.\\
$(\X, +, \lambda\cdot,\R)$ valós  LT, ezek lesznek csak\\
$(\X, +, \lambda\cdot,\C)$ komplex LT ($\C$ számtest)\\
2) Lineáris függőség, lineáris függetlenség, altér.

\begin{de}[Lineáris burok]
  $H \subset \X$ halmaz \emph{lineáris burka} (altér, legszűkebb H-t
  tartalmazó halmaz):
  $\di L(H) := \bigcap_{\underset{H\subset X_0}{X_0 < \X \text{altér}}} X_0$
\end{de}
3) Dimenzió, bázis

\begin{de}
  \begin{enumerate}[\quad(a)]
  \item Az $\X$ LT véges dimenziós, ha $\exists  e_1,\ldots,e_n\in \X$
    linárisan független és $L(\{e_1,\ldots,e_n\})=\X$; $\dim(\X) = n$, tehát
    $\{e_1,\ldots,e_n\}$ bázis $\X$-en
  \item Az $\X$ LT végtelen dimenziós, ha nem véges dimenziós: $\dim\X
    = +\infty$
  \end{enumerate}
\end{de}

\begin{Megj}
\item $\dim\X=+\infty \ekviv \forall n\in\N\ \exists
  e_1,\ldots,e_n$ lineárisan független elem \X-ben.
\item Linalg: $\dim \X < +\infty$
\end{Megj}

\begin{pl}
  \begin{enumerate}
  \item $\R^n$ a ``szokásos'' műveletekkel LT, $\dim \R^n=n$
  \item $C[a,b]$ pontonkénti műveletekkel LT, $\dim C[a,b] = +
    \infty$  ugyanis $\forall n\in \N\colon 1,x,x^2,\ldots,x^n,
    \ldots$ függvények lineárisan függetlenek.
  \item $l_p;\ (x_n),(y_n)\in l_p\colon (x_n)+(y_n) = (x_n+y_n);\
    \lambda(x_n)=(\lambda x_n)$ LT, $\dim l_p=+\infty$ ui:
    $e_n=(0,\ldots,\overset{n-1}{\breve{0}},\overset{n}{\breve{1}},
    \overset{n+1}{\breve{0}},\ldots)$   
  \end{enumerate}
\end{pl}

\subsection{Normált terek}
\begin{de}
  Az $\NT$ normált tér (NT, ha)
  \begin{enumerate}
  \item $\X$ LT $\R$ felett
  \item $\Vert.\Vert\colon \X\n\R$ olyan fv, melyre
    \begin{enumerate}
    \item $\Vert x \Vert \geq 0\quad \forall x\in\X$\\
      $\Vert x \Vert = 0 \ekviv x = 0$ ($\X$ nulleleme)
    \item $\Vert\lambda x\Vert = \vert\lambda\vert\cdot \Vert
      x\Vert\quad \forall x\in \X,\,\forall \lambda\in\R$
    \item $\Vert x+y\Vert\leq\Vert x\Vert+\Vert y\Vert \quad
      \forall x,y\in \X$ (3szög-egyenlőtlenség)
    \end{enumerate}
  \end{enumerate}
\end{de}
\begin{te}
  legyen $\NT$ NT. Ekkor a 
  \[\ro(x,y) := \Vert x - y \Vert\qquad (x,y\in \X)\]
  fv metrika az \X-en, ez a $\Vert.\Vert$ norma által indulkált
  metrika,
  azaz $\forall$NT egyúttal MT is
\end{te}
\newpage                                 %%%%%%%%%%%%%%%%%%%%%%%%%%%%%%%%%%%%%%%%%%%%%%
\begin{Biz}
\item $\ro(x,y) \geq 0$ és $\ro(x,y)=0 \ekviv x = y$
\item $\ro(x,y) = \ro(y,x)$
\item 3szög-egyenlőség: $\Vert x-y\Vert = \Vert( x-z ) + ( z-y)\Vert
  \leq \Vert x-z \Vert + \Vert z-y\Vert$
\end{Biz}

\begin{megj} NT $\subsetneqq$ MT, azaz $\exists$ MT, melyben nincs
  olyan norma, ami a metrikát indukálná
\end{megj}

\begin{Pl}
\item $(\R^n,\Vert.\Vert_p)\qquad n\in \N,\, 1\leq p \leq
  +\infty$
  \[\di\begin{array}{clc}
  1\leq p < +\infty\colon & \Vert x \Vert_p := \left(\sum\limits_{i=1}^n
  |x_k|^p\right)^\frac1{p} & x=(x_1,\ldots,x_n)\in \R^n\\
  p=+\infty\colon &\Vert x \Vert_\infty := \max\limits_{1\leq
    k\leq n} |x_k|& \text{maximum-norma}\end{array}\]
  NT és $\Vert.\Vert_p$ a $\ro_p$ metrikát indukálja.

\item  $(C[a,b],\, \Vert.\Vert_p)\qquad 1\leq p\leq+\infty$
  \[\di\begin{array}{clc}
  1\leq p < +\infty\colon & \Vert f \Vert_p := \left(\di\int\limits_a^b
  \vert f\vert^p\right)^\frac1{p} & f \in C[a,b]\\
  p=+\infty\colon &\Vert x \Vert_\infty := \max\limits_{x\in
    [a,b]} \vert f(x)\vert & \end{array}\]
  NT; $\Vert.\Vert_p$ a $\ro_p$ metrikát indukálja.

\item $(l_p,\Vert.\Vert_p)\qquad 1\leq p \leq
  +\infty$
  \[\di\begin{array}{clc}
  1\leq p < +\infty\colon & \Vert x \Vert_p := \left(\sum\limits_{i=1}^\infty
  |x_k|^p\right)^\frac1{p} & x=(x_n)\in l_p\\
  p=+\infty\colon &\Vert x \Vert_\infty := \sup\limits_{k\in\N} 
  |x_k|& \end{array}\]
  NT és $\Vert.\Vert_p$ a $\ro_p$ metrikát indukálja.
\item Mátrixok: $\R^{n\times m}$ LT, különböző normák
  értelmezhetőek, ld mátrixnormák (numanal)
\end{Pl}

\subsubsection{Konvergencia és teljesség NT-ekben}
\begin{de}\NT\ NT, $(x_n)\colon \N\n\X$ konvergens, ha  
  \[\exists \xi  \in \X\colon \lim_{n\n\infty} \Vert \xi-x \Vert=0\]
  Jel: $\lim(x_n)\stackrel{\Vert.\Vert}=\xi$; \quad $x_n  \xrightarrow[n\n\infty]{\Vert.\Vert} \xi$      
\end{de}

\begin{megj}
  Műveletek konvergens sorozatokkal: összeg, számszoros
\end{megj}


\begin{de}[Banach-tér]
  Az $\NT$ NT Banach-tér (BT), ha ha norma által indukált metrikával nyert MT teljes.
\end{de}
\begin{Pl}
\item $(\R^n,\,\Norma_p)$ BT $\quad(n\in\N; 1\leq p\leq \infty)$
\item $(l_p,\,\Norma_p)$ BT $\quad(n\in\N; 1\leq p\leq \infty)$
\item $(C[a_b],\,\Norma_\infty)$ BT 
\item $(C[a_b],\,\Norma_p)$ NEM BT $\quad(n\in\N; 1\leq p< \infty)$
\end{Pl}

\subsubsection{Ekvivalens normák}
\begin{de}
  $(\X,\Norman1),\ (\X,\Norman2)$ NT. A két norma ekvivalens 
  \[ \text{(jel:) } \Norman1\sim\Norman2\text{, ha }\exists m,M>0:m\Norman2\leq \Norman1\leq M\Norman2\]    
\end{de}
\begin{megj}
  Az ekvivalens normák szerepe: lásd ekvivalens metrikák
\end{megj}


\begin{te}Az $\R^n$ téren bármely két norma ekvivalens.\end{te}
\begin{biz}  Elég belátni, hogy $\Norma$ tetszőleges norma $\R^n$-en ekvivalens $\Norma_\infty$-nel, ui $\sim$
  ekvivalenciareláció, azaz $\exists m,M>0$, hogy
  \begin{gather}
    \norma x\leq M\cdot \norma x_\infty\label{eqg:1}\\
    m\norma x_\infty\leq \norma x\label{eqg:2}
  \end{gather}\begin{gather*}
  \intertext{(\underline{\ref{eqg:1}) bizonyítása}}%%%%%%%%%%%%%%%%%%%%%
  e_1,\dotsc,e_n\in\R^n\text{ szokásos bázis: } e_1=(0,\dotsc,0,\overset{i}{\breve{1}},0,\dotsc,0)\\
  x\in\R^n,\ x=\sum_{k=1}^n x_k e_k\\
  \norma x=\Vert{\sum_{k=1}^nx_ke_k}\Vert \leq \sum_{k=1}^n\norma{x_ke_k}=\sum_{k=1}^n \vert x\vert \norma{e_k} \leq
  \left(\sum_{k=1}^n\norma{e_k}\right) \cdot \underbrace{\max_{1\leq k\leq n}\vert x_k\vert}_{\norma{x}_\infty}=
  M\cdot\norma{x}_\infty  
  \intertext{\underline{(\ref{eqg:2}) bizonyítása} indirekt módon.}%%%%%%%%%%%%%%%%%%%%%%%%
  \text{tfh:}\forall k\in\N\ \exists x_k\in\R^n\colon \norma{x_k}_\infty>k\norma{x_k}\\
  y_k:=\dfrac{x_k}{\norma{x_k}_\infty}\quad(k\in\N)\ \nn\ \norma{y_k}=\norma{\dfrac{x_k}{\norma{x_k}_\infty}}=
  \dfrac{\norma{x_k}}{\norma{x_k}_\infty}<  \dfrac{\norma{x_k}}{k\norma{x_k}}=\dfrac1k\\
  \nn \lim_{k\to+\infty}\norma{y_k}=0\text{ így:}\\
  y_k\xrightarrow[k\to+\infty]{\Norma}\nullelem\in\R^n\tag{3}\label{eqg:3}\\
  \norma{y_k}_\infty= \norma{\dfrac{x_k}{\norma{x_k}_\infty}}_\infty= \dfrac{\norma{x_k}_\infty}{\norma{x_k}_\infty} = 1
  \quad \forall k\in\N\nn (y_k)\text{ korlátos } (\R^n,\,\Norma_\infty)\text{ NT-ben}
  \intertext{A Bolzano-Weierstrass-féle kiválasztási tétel alapján $\exists (y_{k_i})$ konvergens részsorozata}
  y_{k_i}\xrightarrow[k_i\to+\infty]{\Norma_\infty}y\\
  (\ref{eqg:3})\ \nn\ y_{k_i}\xrightarrow[k\to+\infty]{\Norma}\nullelem\tag{4}\label{eqg:4}\\
  \intertext{De}
  \norma{y_{k_i}-y}\overset{(\ref{eqg:1})}{\leq} M\norma{y_{k_i}-y}_\infty\to0\ \nn\ %
  y_{k_i}\xrightarrow[k_i\to+\infty]{\Norma}y \ \overset{(\ref{eqg:4})}{\nn}\ y=\nullelem\text{ ui a határérték}\\
   \hspace*{1em}\text{ egyértelmű } \nn\   y_{k_i}\xrightarrow[k\to+\infty]{\Norma_\infty}\nullelem \nn
  \lim_{k_i\to+\infty}\norma{y_{k_i}}_\infty=0
  \end{gather*}
  Ez pedig ellentmondás $\norma{y_{k_i}}_\infty=1$ miatt.
\end{biz}


\subsection{Euklideszi terek}
\begin{de}
  Az $\ET$ rendezett párt (valós) euklideszi térnek (ET) nevezzük, ha
  \begin{enumerate}
  \item $\X$ LT $\R$ felett
  \item $\Skalar\colon\X\times\X\n\R$ olyan fv, melyre
    \begin{enumerate}
    \item $\skalar x y \,=\,\skalar y x\qquad \forall x,y\in\X$
    \item $\skalar{\lambda x} y \,=\,\lambda\skalar x y\qquad \forall x\in\X,\ \forall\lambda\in\R$
    \item $\skalar{x_1+x_2} y \,=\,\skalar{x_1} x + \skalar {x_2} y \qquad (x_1,x_2,y\in\X)$
    \item $\forall x\in\X\ \skalar x x \geqq0 \text{ és } =0\ekviv x=0$
    \end{enumerate}
  \end{enumerate}
\end{de}

\begin{te}
  $\ET$ ET, ekkor
  \[ \Vert x\Vert := \sqrt{\skalar x x}\qquad(x\in\X)\]
  norma az \X-en, ez a \Skalar\ skaláris szorzat által indukált norma. $\NT$ NT, azaz minden ET egyúttal NT is.
\end{te}

\begin{biz}
  \begin{lemma}[Cauchy-Bunyakovszkij-egyenlőtlenség]\ \\
    $\ET$ ET, \Norma\ indukált norma. Ekkor
    \[\left|\skalar{x}{y}\right| \leq \norma{x}\cdot\norma{y}\qquad \forall x,y\in\X\] 
  \end{lemma}
  \textbf{Lemma bizonyítása}:\\
  $x,y\in\X$ rögzített, $\lambda \in R$
  \[
  \begin{split}
    0 \leq \skalar{\lambda x+y}{\lambda x+y} &\overset{\text{3. tul}}=
    \skalar{\lambda x}{\lambda x +y} + \skalar{y}{\lambda x +y} = \dotsb ={}\\
    &=  \lambda^2 \underbrace{\skalar{x}{ x}}_{\ \norma{x}^2} + 2\lambda\skalar{x}{y} +
    \underbrace{\skalar y y}_{\ \norma{y}^2}
  \end{split}
  \]
  Ha $\norma{x} = 0$ akkor kész\\
  Ha $\norma{x} \ne 0$ akkor a $\lambda$-val másodfokú egyenlet diszkriminánsa 0. \hfill$\blacktriangle$\\

  \textbf{A tétel bizonyítása}
  \begin{enumerate}[\qquad(i)]
  \item $\norma x \geq 0$ és $\norma x =0 \ekviv x = \nullelem$ - teljesül
  \item $\norma{\lambda x} = |\lambda| \cdot \norma{x}\qquad(\forall x \in N, \forall \lambda\in \R$ - teljesül
  \item Háromszög-egyenlőtlenség:\\
    \[ \begin{split}
      \norma{x+y}^2 &= \skalar{x+y}{x+y} = \underbrace{\skalar{x}{x}}_{\norma{x}^2} + 2\skalar{x}{y} + \underbrace{
	\skalar{y}{y}}_{\norma{y}^2} \overset{C-B}{\leqq} {}\\
      & \leqq \norma{x}^2 + 2\norma{x}\,\norma{y} + \norma{y}^2 = (\norma{x}+\norma{y})^2 \text { teljesül}
    \end{split} \]
  \end{enumerate}     
\end{biz}

\begin{megj}
  ET $\subsetneqq$ NT $\subsetneqq$ MT
\end{megj}
\newpage
\begin{de}[Elemek szöge] $\ET$ ET;  $\forall x,y\in \X\setminus\left\{\nullelem\right\}$-hoz
  \begin{enumerate}
  \item $\exists!\varphi\in[0,\pi]\quad \cos\varphi=\dfrac{\skalar{x}{y}}{\norma{x}\,\norma{y}}$\\
    $\varphi$ az $x$ és $y$ által bezárt szög
  \item $x$ és $y$ merőlegesek vagy ortogonálisak, ha $\skalar x y =0$       
  \end{enumerate}
\end{de}

\begin{Pl}
\item $(\R^n,\Skalar)\quad n\in \N\qquad x=(x_1,\dotsc,x_n)\in \R^n$
  \[ \skalar x y = \sum_{i=1}^n x_i y_i\qquad(x,y \in \R^n) \] 
  skaláris szorzat $\R^n$-n. Ez a skaláris szorzat a $\Norma_2$-t indukálja.
\item $(C[a,b],\,\Skalar)$ ET; $\di\skalar f g := \int_a^b fg \quad (f,g\in C[a,b])$\\
  Ez a skaláris szorzat a $\Norma_2$-t indukálja: $\di\sqrt{\skalar f g} = \sqrt{\int^b_a|f|^2} = \norma{f}_2$
\item $(l_2,\Skalar)\quad x=(x_n)\in l_2$
  \[\skalar x y := \sum_{n=1}^\infty x_n y_n \text{ skaláris szorzat}\]
  Ez a skaláris szorzat a $\Norma_2$-t indukálja.
\end{Pl}

\begin{te}
  Adott egy $\NT$ NT. Ekkor:
  \begin{align*}
    \exists \Skalar \text{ skaláris szorzat, ami a }\Norma \text{ normát induklálja}
    \intertext{\quad akkor és csak akkor, ha a normára az alábbi feltétel teljesül:}
    \norma{x+y}^2 + \norma{x-y}^2 = 2 (\norma{x}^2 + \norma{y}^2)\quad \forall x,y\in \X
  \end{align*}
\end{te}
\begin{megj}
  Ez a paralelogramma-szabály: $e^2 + f^2 = 2(a^2 + b^2)$
\end{megj}

\begin{biz} Ld gyak, kell
\end{biz}

\begin{te}
  $\R^n,\ C[a,b],\ l_p$-beli $p$-normák közül csak $p=2$ teljesíti a paralelogramma-szabályt, azaz csak a $p=2$ norma
  származtatható skaláris szorzatból.
\end{te}

\begin{biz} Ld gyak, kell
\end{biz}

\begin{de}[Hilbert-tér]
  Az $\ET$ ET Hilbert-tér (HT), ha a skaláris szorzat által indukált normával kapott NT teljes.     
\end{de}


\begin{Pl}
\item $(\R^n,\,\Skalar)$ HT
\item $(l_2,\,\Skalar)$ HT
\item $(C[a,b],\,\Skalar)$ NEM Hilbert-tér
\end{Pl}



% Local Variables:
% fill-column: 120
% TeX-master: t
% End:

  \newpage
  \section{Differenciálszámítás}

\begin{megj} Csak $\RnRm$ típusú fv-ek; $n,m\in\N$\end{megj}

\subsection{$\RnRm$ típusú leképezések}
\begin{de}
  $L: \RnRm$ lineáris, ha
{\listazjromai
\begin{enumerate}
  \item $L(x + y) = L(x) + L(y)\qquad \forall x,y\in \R^n$ (additivitás)
  \item $L(\lambda x) = \lambda L(x)\qquad \forall x\in\R^n,\ \lambda\in\R$ (homogenitás)  
\end{enumerate}
}
\end{de}

\begin{Megj}
\item Ha $L$ lineáris, akkor $\di L\left(\sum_{i=1}^s \lambda_i y_i\right) = \sum_{i=1}^s\lambda_i L(y_i)$
\item $L\in\R\to\R$ lineáris $\ekviv \exists!c\in \R\ L(x)\quad  x\in\R$\\
  $L$ és $c$ azonosítható egymással.
\item  Jelölés: $\Linearis :=  \{\,L\in R^n\to\R^m\,|\,L\text{ lineáris}\,\}$
\item  Szokásos műveletek: $+$, $\lambda\cdot$ pontonként.
\end{Megj}

\begin{te}
  \Linearis a szokásos műveletekkel LT az $\R$ felett
\end{te}

\subsubsection[Lineáris leképezés mátrix reprezentációja]{Lineáris leképezés mátrix reprezentációja (adott bázisban)}
$\R^n$-ben $e_1,\,e_2,\dotsc,\,e_n$ egy bázis (pl a kanonikus bázis: $e_i=(0,\dotsc,0, \overset{i}{\breve{1}},
0,\dotsc,0)$)\\
$\R^m$-ben $f_1,\,f_2,\dotsc,\,f_n$ egy bázis (pl a kanonikus bázis)\\
$L\in\Linearis,\ x\in \R^n,\di x = \sum_{j=1}^n x_j e_j\quad x_j$ az $x$ vektor $e$ bázisra vonatkozó
koordinátája
\begin{gather*}
  \di  L(x) = L\left(\sum_{j=1}^n x_j e_j\right) = \sum_{j=1}^n x_j L(e_j)\\
  \R^m \owns L(e_j) = \sum_{i=1}^m a_{ij} f_i\qquad\quad\text{$a_{ij}$ az $L(e_j)$ $i$. koordinátája az
  $(f_1,\dotsc,f_n)$ bázisban}\\
  A = \begin{pmatrix}
    a_{11} & a_{12} & \dots & a_{1n}\\
    a_{21} & a_{22} & \dots & a_{2n}\\
    \vdots & \vdots & \ddots & \vdots\\
    a_{m1} & a_{m2} & \dots & a_{mn}
  \end{pmatrix}\in \R^{m\times n}\\
  L(x) = \sum_{j=1}^n x_j L(e_j) = \sum_{j=1}^n x_j \left(\sum_{i=1}^m a_{ij} f_i\right) = \sum_{i=1}^n
  \left(\sum_{j=1}^n a_{ij} x_j\right) f_i\\
  \text{A } \sum_{j=1}^n a_{ij} x_j \text{ mátrixszorzással felírva:}\\
  \begin{pmatrix}
    a_{11} & \dots & a_{1n}\\
    \vdots & \ddots & \vdots\\
    \mathbf{a_{i1}} &  \dots & \mathbf{a_{in}}\\ 
    \vdots & \ddots & \vdots\\
    a_{m1}  & \dots & a_{mn}
  \end{pmatrix}
  \begin{pmatrix}x_1\\x_2\\\vdots\\x_n\end{pmatrix} = (Ax)_i
\end{gather*}

Tehát egyértelműen megfeleltethető e kettő egymásnak:\\
$L\in \Linearis\leftrightarrow A\in \R^{m\times n}$

\begin{te}
  Az $\Linearis$ LT izomorf  az $\R^{m\times n}$ LT-rel, azaz: 
  \begin{gather*}
    \exists \varphi: \Linearis\n\R^{m\times n} \text{ bijekció, melyre}\\
    \varphi(\lambda_1 L_1+\lambda_2 L_2) = \lambda_1\varphi(L_1) + \lambda_2 \varphi(L_2)
    \qquad \forall L_1,\,L_2\in\Linearis;\ \forall \lambda_1,\,\lambda_2\in\R
  \end{gather*}
\end{te}

\begin{megj}
  Az $\RnRm$ leképezés azonosítható egy $A\ m\times n$-es mátrix-szal (egy adott bázisban).
\end{megj}

\subsubsection{Norma értelmezése az $\Linearis$ LT-n}

\begin{te}
  \begin{enumerate}
    \item Ha $L\in\Linearis$, akkor 
      \[\di \opnorma {L} := \sup_{h\in\R^n}\{\,\underbrace{\norman{L(h)}{1}}_{\text{$\R^m$-beli}} : h\in\R^n,
      \underbrace{\norman{h}{2}}_{\text{$\R^n$-beli}}<1\} < \infty\]
      ún. \emph{operátornorma} norma az \Linearis\ téren.
      \item Ha $L\in\Linearis$, akkor
	\[ \norman{L(h)}1 \leqq \opnorma{L}\cdot\norman{h}2\qquad\forall h\in\R^n 
	\]
  \end{enumerate}
\end{te}

\begin{biz}?\end{biz}

\begin{megj}
  $f\in\R\n\R,\  f\in\D\{a\},\ a\in\intD_f$
  \begin{gather*}\di
    f\in\D\{a\} \ \ekviv\  \exists \lim_{h\n0} \dfrac {f(a+h)-f(a)}{h} < \infty \overset{\text{lineáris}}{\underset
      {\text{közelíés}}{\ekviv}}\\\ekviv \exists A\in \R \text{ és } \exists \epsilon \in \R\n\R,\, \lim_0\epsilon=0
    \colon f(a+h) -f(a) = Ah + \epsilon(h)\cdot h \ \ekviv\\
    \lim_{h\n0}\dfrac{|f(a+h)-f(a)-L(h)|}{|h|} = 0
  \end{gather*}
\end{megj}



\subsection{Totális deriválhatóság}

\begin{de}[Totális deriválhatóság]
  $f\in\RnRm,\ a\in\intD_f$. Az $f$ fv totálisan deriválható az $a\in\intD_f$ pontban, ha
  \[\exists L\in\Linearis\colon\quad \lim_{h\n\nullelem}\dfrac{\norman{f(a+h)-f(a)-L(h)}1}{\norman{h}2} = 0,\]
  ahol \Norman1\ egy $R^m$-beli tetszőleges, \Norman2 egy $\R^n$-beli tetszőleges norma.\\
  Az $f$ fv $a$-beli deriváltja: $f'(a) := L(a)$\\
  Jel: $f\in\D\{a\}$
\end{de}

\begin{te}
  Ha $f\in\derivp{a},\ f\in\RnRm$, akkor $f'(a)$ egyértelmű.
\end{te}

\begin{biz}
  Tegyük fel, hogy $\exists L,R\in\Linearis$, amire a definíció teljesül, továbbá 
  \begin{gather*}
     L-R=:S\in\Linearis\\
     \dfrac{\norman{S(h)}1}{\norman{h}2} = \dfrac{\norman{L(h)-R(h)}1}{\norman{h}2} \leqq\\
     \leqq\dfrac{\norman{f(a+h)-f(a)-L(h)}1}{\norman{h}2} + \dfrac{\norman{f(a+h)-f(a)-R(h)}1}{\norman{h}2} \xrightarrow
     [h\n\nullelem]{}0\\
     \nn \lim_{h\n\nullelem} \dfrac{\norman{S(h)}1}{\norman{h}2} = 0.\\\text{Legyen spec: }h:=\lambda e,\
     e\in\R^n,\ \norman{e}2 = 1,\ \lambda \n 0\, \nn\, h\n\nullelem. \\
     \dfrac{\norman{S(h)}1}{\norman{h}2} = \dfrac{\norman{S(\lambda e)}1}{\norman{\lambda e }2} =
     \dfrac{|\lambda|\,\norman{S(e)}1}{|\lambda|\,\norman{e}2}\, \nn\, S(e) = \nullelem \,\nn\, S \equiv 0 \,\nn\, L
     \equiv R
  \end{gather*}
\end{biz}

\begin{te}
  A deriválhatóság ténye és a derivált független attól, hogy $\R^n$-ben és $\R^m$-ben melyik normát választjuk.
\end{te}
\begin{biz}
  $\R^n$-ben a normák ekvivalensek.
\end{biz}

\begin{te}[Ekvivalens átfogalmazások]\ 
  \begin{enumerate}
    \item $\di f\in\der{a} \ekviv  \exists A\in\Rmn\colon  \lim_{h\n\nullelem} \dfrac{\norman{f(a+h) -f(a) -Ah}1}{
    \norman{h}2} = 0$\\
      Az $A$ az ún. \emph{deriváltmátrix}.
    \item $f\in\der{a}  \ekviv \left\{\begin{array}{l}\exists A\in\Rmn \text{  és } \di\exists \epsilon \in\RnRm\
    \lim_0\epsilon=\nullelem :\\f(a+h) -f(a)= Ah + \epsilon(h)\,\norman{h}2\qquad a,a+h\in \D_f\end{array}\right.$\\
      (lineáris fv-nyel való jó közelítés)
  \end{enumerate}
\end{te}
\begin{biz}Trivi\end{biz}
  

\textbf{Spec esetek:}
\begin{enumerate}
  \item $f\in\RnRm;\ f'(a)$ egy $m\times n$-es mátrix ($\in\Rmn)$
  \item $f\in\RnR;\ f'(a)$ egy sorvektor  ($\in R^{1\times n})$
  \item $f\in\RRm;\ f'(a)$ egy oszlopvektor ($\in R^{m\times1})$   
\end{enumerate}


\begin{Pl}
\item $f(x):=c\quad(x\in\R^n);\ c\in\R^m$ rögzített $\nn \forall a\in\R^n:\ f\in\der{a};\ f'(a)=0\in\Rmn$
\item $f=L\in\Linearis,\ \forall a\in\R^n\colon L\in\der a$ és $L'=L$
\end{Pl}

\begin{te}[Folytonosság-deriválhatóság]
  $f\in\RnRm$, $a\in\intD_f$
  \begin{enumerate}
    \item Ha $f\in\der a\nn f\in\folyt a$
    \item Visszafelé \emph{nem} igaz
  \end{enumerate}
\end{te}
\begin{biz}
  \[\nn: f(a+h) -f(a) = Ah + \epsilon(h)\,\norman{h}2\xrightarrow[h\n\nullelem]{}0 \nn f\in\folyt{a}\]
  \[\not\Leftarrow: n=m=1 \text{ és pl } f := \mathrm{abs}\]
\end{biz}

\begin{te}
  $f\in\RnRm, f=\begin{bmatrix}f_1\\\vdots\\f_m\end{bmatrix};\ f_i\in\RnR\ (i=1,\dotsc,m)$ az $f$
  koordinátafüggvényei\\
Ekkor
\[f\in\der{a} \ekviv \forall i = 1,2,\dotsc,m\text{ és } f_i\in\der{a} \text{ és } f'(a)= \begin{bmatrix}f_1'(a) \\
  f_2'(a) \\
  \vdots\\ f_m'(a)\end{bmatrix} = \begin{bmatrix}\hspace{0.8em}\dots\hspace{0.8em}\\\dots\\\vdots\\\dots
\end{bmatrix}\in\Rmn\]
\end{te}
\begin{biz}Trivi
\end{biz}

\begin{megj} $f\in\RnRm\ \nn\ \RnR$ elég vizsgálni.
\end{megj}
\subsubsection{Műveletek}
\begin{te}$f,g\in\RnRm,\ a\in(\intD_f)\cap(\intD_g),\ f,g\in\der a$\\
  $\begin{array}{@{\quad}rcl}
    1^\circ &  f+g\in\der a & (f+g)'(a) = f'(a) + g'(a)\\
    2^\circ &  \lambda g\in\der a\quad \forall \lambda\in\R & (\lambda f)'(a) = \lambda f'(a)
  \end{array}$
\end{te}
\begin{biz}
  trivi
\end{biz}

\begin{te}[Kompozíció, láncszabály]
  $g\in\RnRm,\ a\in\intD_g, g\in\der a$;\\ $f\in\R^m\n\R^r,\ R_g\subset D_f,\ f\in \der{g(a)}$.\\
  Ekkor $f\circ g\in\R^n\n\R^r$ deriválható az $a$-ban és $(f\circ g)'(a) = f'(g(a) \circ f'(a)$
\end{te}
\begin{biz} nem kell (ld. $\R\n\R$ eset)
\end{biz}

\begin{megj}\ \\
  \begin{tabular}{r@{$\,\in\,$}l@{$\ \nn\ $}l}
    $f\circ g$ & $\R^n\n\R^r$ &  $C:=(f\circ g)'(a)\in\R^{r\times n}$\\
    $g$ & $\R^n\n\R^m$ &  $A:=g'(a)\in\R^{m\times n}$\\
    $f$ & $\R^m\n\R^r$ &  $B:=f'(a)\in\R^{r\times m}$\\
  \end{tabular}$ C = BA$\\
Lineáris leképezések kompozíciója $\equiv$ mátrixreprezentációk szorzata
\end{megj}


\subsection{Parciális deriválhatóság}

\begin{de}[Parciális derivált]
  $f\in\RnRm,\ a\in\intD_f,\ e_1,e_2,\dotsc,e_n\in\R^n$ kanonikus bázis, azaz $e_i = (0,\dotsc,0,
  \overset{i}{\breve{1}}, 0,\dotsc,)0 $\\
  \emph{Az $f$-nek $\exists$ az $i$. változó szerinti parciális deriváltja az $a\in\intD_f$ pontban}, ha az\\
  $F: K(0)\owns t \mapsto f(a+t e_i)\quad (F: \R\n\R^m)$\\
  fv deriválható a $0$ pontban.\\
  Az $F'(0)$ oszlopvektor az $f$ $i$. változó szerinti parciális deriváltja az $a$-ban.\\
  Jel: $\partial_i f(a) := F'(0);\quad \dfrac {\partial f}{\partial {x_i}}(a)$ 
\end{de}

\begin{Pl}
  \item $f(x,\,y) = x^3 y^2\quad (x,y)\in\R^2$\\
    $\di\partial_1 f(x_0,y_0) = (x\n f(x,y_0)'_{x=x_0} = 3{x_0}^2{y_0}^2$\\
    $\di\partial_2 f(x_0,y_0) = (y\n f(x_0,y)'_{y=y_0} = 2{x_0}^3y_0$
  \item $f(x,y,z) = x^3 y^2 z$\\
    $\partial_2 f(x_0,\,y_0,\,z_0)= 2{x_0}^3y_0z_0$
\end{Pl}


\begin{te}
  $f\in\RnRm,\ a\in\intD_f$\\
  Ha $f\in\der{a}$, akkor $\forall i=1,2,\dotsc,n\colon  \partial_if(a)$ létezik
\end{te}

\begin{biz}
  \begin{gather*}
  F(t) := f(a+te_i),\qquad F:=f\circ g,\qquad g(t)=a+te_i.\\
  g\in\der 0,\ g'(a)=e_i\nn F=f\circ g\in\der 0\nn \exists.\\
  F'(0) = \partial_i f(a) = f'(a)g'(a) = f'(a)\cdot(e_1) = \underbrace{f'(a)}_{\text{mátrix}} \cdot
  \underbrace{e_1}_{\text{vektor}}
  \end{gather*}
\end{biz}

\begin{te}[A deriváltmátrix előállítása]
  $f\in\RnRm,\ a\in\intD_f,\\ f=\begin{bmatrix}f_1\\\vdots\\f_n\end{bmatrix};\ f_i\in\RnR\ (i=1,\dotsc,m)$\\
  \vspace{.1em}
  Ha $f\in\der a\nn\\$
  \[\Rmn\owns f'(a) = \begin{bmatrix}
    \partial_1 f_1(a) & \partial_2 f_1(a) & \dots & \partial_n f_1(a)\\
    \partial_1 f_2(a) & \partial_2 f_2(a) & \dots & \partial_n f_2(a)\\
    \vdots & \vdots & \ddots & \vdots \\
    \partial_1 f_1(a) & \partial_2 f_1(a) & \dots & \partial_n f_1(a)
    \end{bmatrix}\]
  deriváltmátrix vagy \emph{Jacobi-mátrix}
\end{te}

\begin{biz}
    $f\in\der a \nn \exists f'(a) = A = \left[ a_{ij}\right] \in \Rmn$
    \[\di\lim_{h\n\nullelem} \dfrac{\norman{f(a+h)-f(a) -Ah}1}{\norman h2}=0\]
    $\nn \forall i = 1,2,\dotsc,m \text{ esetén}$
    \[\di\lim_{h\n\nullelem} \dfrac{\left|f_i(a+h)- f_i(a) - \sum\limits_{k=1}^n a_{ik} h_k\right|}{\norman h2}=0\]
    Legyen spec: $h := te_j\quad (t\in\R)\quad (e_j = (0,\dotsc,0,\overset{j}{\breve{1}},0,\dotsc,0))\quad h\n\nullelem
    \ekviv t\n0$

    \[\di\nn \lim_{t\n0}\dfrac{\left|f_i(a+te_j)-f_i(a) - a_{ij}t\right|}{|t|\,\norman{e_j}2} = 0\]
    $\overset{\text{parc.der.}}{\nn} a_{ij} = \partial_j f_i(a)$
\end{biz}

\begin{Megj}
\item \underline{Spec. esetek}\\
\begin{enumerate}
  \item $f\in\RnR,\ f\in\der a\\ f'(a) = \begin{bmatrix}\partial_1 f(a) & \partial_2 f(a) & \dots & \partial_n
    f(a)\end{bmatrix}$
    az $f$ gradiens vektora, $\grad f(a)$
  \item $f\in\RRm\quad f=\begin{pmatrix}f_1\\\vdots\\f_m\end{pmatrix}\quad f_i\in\R\n\R$\\
    $f\in\der a\nn f'(a) = \begin{bmatrix}f_1'(a)\\\vdots\\f_m'(a)\end{bmatrix}$
\end{enumerate}
\item Totális és parciális derivált kapcsolata\\
  Tudjuk: $f\in\der a \nn \forall $ parciális derivált létezik.\\
  Várható: visszafele NEM igaz.\\
  Pl: $f(x,y) := \sqrt{|xy|}\quad(x,y)\in\R^2$.\\
  Ekkor $\partial_1 f(0,0) = 0 = \partial_2 f(0,0)$, de $f\not\in\der{(0,0)}$\\
  Hf, gyak
\item Meglepő: elégséges feltétel adható
\item $f\in\RnRm\qquad f=\begin{bmatrix}f_1\\\vdots\\f_m\end{bmatrix}$\\
  $f\in\der a \ekviv \forall i=1,\dotsc,m\ f_i\in\der a$\\
  azaz: elég az $m=1$ esetet vizsgálni  
\item Jelölés:\\
  $\varphi\in\RnR\\
  f\in\RnRm$
\end{Megj}

\begin{te}[Elégséges feltétel a deriválhatóságra]
  Legyen $\varphi\in\RnR,\ a\in\intD_\varphi,\\\varphi\in K(a)\n\R$.
  Tfh. $\forall i=1,\dotsc,n$-re
  \begin{enumerate}
    \item a $\partial_i\vfi$ parciális deriváltak léteznek $\forall x\in K(a)$-ra.
    \item $\partial_i\vfi(x)\colon K(a)\n\R,\ x\n\partial_i \vfi(x)$ parciális derivált függvények folytonosak az
    $a$-ban: $\partial_i \vfi\in\folyt a$      
  \end{enumerate}
Ekkor  $\vfi\in\der a$ (totálisan deriválható)
\end{te}

\begin{biz} $n=2$-re ($n>2$-re hasonlóan):
  \begin {gather*}
    \di\vfi\colon \R^2\n\R,\ a=(a_1,a_2),\ h=(h_1,h_2)\\
    \vfi\in\der a \overset{\text{def}}{\ekviv} \begin{array}{l}
      \exists A,B\in\R,\ \exists \epsilon\in\R^2\n\R,\ \di\lim_\nullelem\epsilon=0\text{, melyre:}\\
      \lim\limits_{h\n\nullelem}\dfrac{\left|\vfi(a+h)-\vfi(a)-(A,\,B)\begin{pmatrix}h_1\\h_2\end{pmatrix}\right|}{\norma h}=0
    \end{array}\\
    \vfi(a+h) - \vfi(a) = \vfi(a_1+h_1,\,a_2+h_2) - \vfi(a_1,\,a_2) =\\
    = \vfi(a_1+h_1,\,a_2+h_2) - \vfi(a_1+h_1,\,a_2) +\vfi(a_1+h_1,\,a_2) - \vfi(a_1,\,a_2) = \star
  \end{gather*}
  A valós-valós Lagrange-középértéktételt felhasználva legyen: $\nu_1\in(0,1)$; $a_2$ rögzített:
  \[\vfi(a_1+h_1,\,a_2) - \vfi(a_1,\,a_2) = \partial_1\vfi(a_1+\nu_1h_1,\,a_2)\cdot h_1\]
  Hasonlóan legyen $\nu_2\in(0,1)$; $a_1+h_1$ rögzített:
  \[\vfi(a_1+h_1,\, a_2+h_2) - \vfi (a_1+h_1, a_2) = \partial_2\vfi(a_1+h_1,a_2+\nu h_2)\cdot h_2\]
  Behelyettesítve:
  \[ \star=\partial_1\vfi(a_2+\nu_1h_1,a_2)\cdot h_1 + \partial_2\vfi(a_1+h_1,a_2+\nu_2 h_2)\cdot h_2 = \sharp\]
  De!
  \begin{gather*}\begin{array}{l}\partial_1\vfi \in\folyt{(a_1,a_2)}\\
      \partial_2\vfi \in\folyt{(a_1,a_2)}\end{array} \nn 
    \partial_1(a_1+\nu_1 h_1, a_2) = \partial_1\vfi(a_1,a_2)+\epsilon_1(h)\\
    \text{ahol a folytonosság miatt} \lim_{h\n\nullelem}\epsilon_1(h)=0 \text{, illetve:}\\
    \partial_2(a_1+h_1, a_2+\nu_2 h_2) = \partial_2\vfi(a_1,a_2)+\epsilon_2(h),\qquad\lim_{h\n\nullelem}\epsilon_2(h)=0    
  \end{gather*}
  Így:
  \begin{gather*}\sharp = \vfi(a+h)-\vfi(a) = \big[\underbrace{\partial_1\vfi(a_1,\,a_2)}_{A}+\epsilon_1(h)\big]h_1 + 
    \big[\underbrace{\partial_2\vfi(a_1,\,a_2)}_{B}+\epsilon_2(h)\big]h_2.\\
    \dfrac{\left|\vfi(a+h) - \vfi(a) - \begin{pmatrix}A & B\end{pmatrix} \begin{pmatrix}h_1\\h_2\end{pmatrix}\right|}
    {\norma h} = \dfrac{\left|\epsilon_1(h)h_1 + \epsilon_2(h)h_2\right|}{\norma h} \leq |\epsilon_1(h) + \epsilon_2(h)| 
    \xrightarrow[h\n\nullelem]{} 0 \\\nn \vfi\in\der{a}
  \end{gather*}
\end{biz}

\subsection{Iránymenti deriválhatóság}

\begin{de}[$e$ irány szerinti iránymenti derivált]
  $f\in\RnRm,\\a\in\intD_f,\ e\in\R^n $ egyésgvektor $(\norma{e}_2 = 1)$.\\
  Az $f$ fv-nek az $a\in\intD_f$ pontban az $e$  irány szerinti iránymenti deriváltja létezik, ha az
  \[F: K(a)\owns t \mapsto f(a+te) \in \R^n\]
  fv a $0$ pontban deriválható. Az $F'(0)\in\R^n$ az $e$ irányban vett iránymenti deriváltja $a$-ban.\\
  Jel: $\partial_ef(a) := F'(0)$
\end{de}

\begin{te}[Az iránymenti derivált kiszámolása]
  Ha $f\in\RnRm,\ a\in\intD_f$,\\
  Ekkor $\forall e\in\R^n$ egységvektor $(\norma{e}_2 = 1)$ esetén $\exists\partial_ef(a)$ és
\[\partial_ef(a) = f'(a)\cdot e\]
\end{te}
\begin{megj} $\partial_ef(a)\in\R^m;\ f'(a) \in\RnRm;\ e\in\R^n$ - mátrixszorzás
\end{megj}
\underline{Spec. esetek:}
\begin{enumerate}
  \item Iránymenti derivált: parciális derivált általánosítása. Ha $e=e_i$, akkor $\partial_ef(a) = \partial_if(a)$
  \item $m=1\colon \ \vfi\colon\RnR^1; e\in\R^n; \norma{e}_2=\di\sqrt{\sum_{k=1}^n{e_k}^n} = 1$
    \[\di \partial_e\vfi(a) = \skalar{\grad f(a)}{e} = \sum_{k=1}^n \partial_kf(a)e_k\]
\end{enumerate}
\begin{Megj}
\item Lényeges: $\norma{e}_2=1$
\item Totális derivált és iránymenti derivált kapcsolata:\\
  Tudjuk: $f\in\der{a}\nn\ \forall$ irányban deriválható\\
  Visszafele NEM igaz!
\end{Megj}

\subsection{Középértéktétel}
\begin{te}[Lagrange-középértéktétel]
  $n\geq1,~U\subset\R^n$ nyílt és $\forall a\in U,\,a+h\in U$.\\
  Szakasz: $[a,\,a+h] := \{\,a+th:t\in(0,1)\,\}\subset U$
  {\listazjbetu
    \begin{enumerate}
    \item Ha $\vfi\colon U\n\R,\ \vfi\in\D(U)$: $U$ minden pontjában deriválható\\
      akkor $\exists \nu\in(0,1):\vfi(a+h)-\vfi(a)=\vfi'(a+\nu h)\cdot h=\skalar{\grad\vfi(a+\nu h)}{h}$
    \item Ha $f\colon U\n\R^m,\ m\geq2,\ f\in\der{x}\ (x\in U)$, akkor
      \[ \norma{f(a+h)-f(a)}_\infty\leq \di\sup_{0\leq\nu\leq1} f'(a+\nu h)(h) \leq
      \sup_{0\leq\nu\leq1}\opnorma{f'(a+\nu h)}\cdot \norma{h}_\infty\]
    \end{enumerate}  
  }
\end{te}
\begin{biz} nem kell
\end{biz}

\subsection{Többször deriválható függvények}
\begin{de}
  $\vfi\in\RnR,\ a\in\intD_f$. A $\vfi$ kétszer deriválható az $a\in\intD_f$, ha
  {\listazjromai
    \begin{enumerate}
    \item $\exists K(a)\colon \forall x\in K(a)$-ban $\vfi$ deriválható
    \item $\forall i=1,\dotsc,n\ \partial_i\vfi$ parciális függvények deriválhatóak $a$-ban: $\partial_i\in\der{a}$
    \end{enumerate}
  }
  Jel: $\vfi \in\dern2a$
\end{de}
\begin{megj}
  $(i) \ekviv \exists\vfi'=(\partial_1\vfi,\dotsc,\partial_n\vfi)\in\R^n\n\R^n$ függvény\\
  $(ii)\ \partial_i\vfi\in\der{a}\ (i=1,2,\dotsc,n) \ekviv \vfi'\in\der{a}$\\\\
  $\vfi'' = (\vfi')'$, de $\vfi'\in\R^n\n\R^n,\ \vfi''\in\R^n\n\R^{n\times n}$.\\
  \[\vfi(a)'' = \begin{bmatrix}
  \partial_1\partial_1\vfi &\partial_2\partial_1\vfi & \cdots & \partial_n\partial_1\vfi \\
  \vdots & \vdots & \ddots & \vdots \\
  \partial_1\partial_n\vfi &\partial_2\partial_n\vfi & \cdots & \partial_n\partial_n\vfi 
  \end{bmatrix}\in\R^{n\times n}\]
  az ún. \emph{Hesse-féle mátrix}  
\end{megj}

\begin{de}
  $\vfi \in\RnR,\ a\in\intD_\vfi$. \\A $\vfi$ $s$-szer $(s\geq2)$ deriválható az $a$-ban $(\vfi \in\dern n a)$, ha
  {\listazjromai
    \begin{enumerate}
    \item $\exists K(a)\subset D_\vfi$, hogy $\vfi$ (s-1)-szer deriválható a $K(a)\ \forall$ pontjában.
    \item Az összes $\partial_{i_1}\partial_{i_2}\ldots\partial_{i_{s-1}}\vfi\quad 1\leq i_1, i_2,\dotsc,i_{s-1} \leq n$
      \quad (s-1)-edrendű parciális derivált függvény deriválható az $a$-ban.
    \end{enumerate}
  }
\end{de}
\begin{te}[Young-tétel]$\vfi\in\RnR,\ a\in\intD_\vfi$
  \[\vfi \in\dern2a\nn\forall i,j=1..n\quad \partial_j(\partial_i\vfi) = \partial_i(\partial_j\vfi) \]
\end{te}
\begin{te}\label{te:youngkov}(Következmény!!) $\vfi\in\RnR,\ a\in\intD_\vfi, \vfi\in\dern s a\ (s\geq2)$
  \[ (\partial_{i_1}\partial_{i_2}\ldots\partial_{i_s}\vfi)(a) = 
  (\partial_{\sigma_1}\partial_{\sigma_2}\ldots\partial_{\sigma_s}\vfi)(a)\]
  ahol $1\leq i_1, i_2,\dotsc,i_s \leq n$ és a $\sigma_1, \sigma_2,\dotsc,\sigma_s$ az $i_1, i_2,\dotsc,i_s$ egy
  permutációja  
\end{te}

\subsubsection{Taylor-formula}
\begin{te}(Emlékeztető)
  $f\in\R\n\R;\ m\in\N,\ f\in\D^{m+1}\left(K\left(a\right)\right);\ a,a+h\in\D_f;\\\exists \nu\in(0,1):$
  \[\di f(a+h) = f(a) + \sum_{k=1}^m\dfrac{f^{(h)}(a)}{k!}h^k + \dfrac{f^{(m+1)}(a+\nu h)}{(m+1)!}h^{m+1}\]  
\end{te}

\begin{de}[Multiindex] $n\geq 1$ rögzített, $i$ multiindex, $i:=(i_1,\dotsc,i_n),\ i_k\geq 0$ egészek.\\
  $|i| := i_1 + i_2 + \ldots + i_n$ a multiindex rendje\\
  $i!~ := i_1! \cdot i_2! \dotsm i_n!$\\
  $x=(x_1,\dotsc,x_n)\in\R^n\colon\quad x^i := x_1^{i_1}\cdot x_2^{i_2}\dotsm x_n^{i_n}$\\
  $\partial^i\vfi := \partial_1^{i_1}\partial_2^{i_2}\dotsb\partial_n^{i_n}\vfi$ vagyis az első változó szerint
  $i_1$-szer, stb.\\
  $\partial^0\vfi :=  \vfi$
\end{de}

\begin{de}[Homogén $n$ változós $m$-edfokú polinom] \ \\$n=1,2,\dotsc$; $m=0,1,2,\dotsc$; $i$ multiindex: $|i|=m$
  \[\R^n\owns x\mapsto \sum_{|i|=m}a_ix^i\quad \text{ahol }a_i\in\R\]  
\end{de}

\begin{spec}{\listazjromai\begin{enumerate}
  \item $n=1;\ m=0,1,2,\dotsc\colon\quad \R\owns x\mapsto ax^m$
  \item $n=2;\ m=1\colon\quad i\colon (0,1) \text { v. }(1,0)\colon\quad \R^2\owns(x_1,x_2)\mapsto a_1x_1+a_2x_2$
  \item $n=2;\ m=2\colon\quad i\colon (2,0),\ (1,1),\ (0,2)\colon\quad \R^2\owns(x_1,x_2)\mapsto a{x_1}^2+bx_1x_2 +
    c{x_2}^2$    
  \end{enumerate} }
\end{spec}

\begin{te}[Taylor-formula a Lagrange-maradéktaggal]Tegyük fel, hogy
  {\listazjbetu \begin{enumerate}
    \item $\vfi\colon U\n\R,\ U\subset\R^n$ nyílt halmaz
    \item $a\in U,\ h\in\R^n\colon\ [a,a+h] := \{a+th:t\in(0,1)\}\subset U$
    \item $\vfi\in\D^{m+1}([a,a+h])\quad(m=0,1,2,\dotsc\text{rögzített})$
  \end{enumerate} }
  Ekkor $\exists \nu\in(0,1)$
  \[ \di\vfi(a+h) = \vfi(a) + \underbrace{\sum_{k=1}^m\left(\sum_{|i|=k}\dfrac{\partial^i\vfi(a)}{i!}h^i\right)}
  _{\text{Taylor-polinom}} + \underbrace{\sum_{|i|=m+1}\dfrac{\partial^i\vfi(a+\nu h)}{i!}h^i}
  _{\text{Lagrange-féle maradéktag}}\]  
\end{te}

\begin{biz}Visszavezethető $\R\n\R$-re\\
  \[ F(t) := \vfi(a+th)\qquad(t\in[0,1])\]
  Az $F\in\R\n\R$ függvényre a Taylor-formula alkalmazható a $[0,1]$ intervallumon (a feltételek teljesülnek).\\
  \[\di\exists \nu\in(0,1)\colon F(1) = F(0) + \sum_{k=1}^m \dfrac{F^{(k)}(0)}{k!} (1-0)^k + 
  \dfrac{F^{(m+1)}(\nu)} {(m+1)!}\]
  
  A tétel állítása a következő lemma felhasználásával adódik.
  \begin{lemma} A fenti $F$ függvény esetén ($\vfi$ $s$-szer deriválható $[a,a+h]$-n)
    \[\di\dfrac{F^{(k)}(t)}{k!}= \sum_{|i|=k}\dfrac{\partial^i\vfi(a+th)}{i!}h^i\qquad k=0,1,2,\dotsc,s\]
  \end{lemma}
  \textbf{A lemma bizonyítása} $k$-ra vonatkozó teljes indukcióval.\\
  $k=1$ esetén $F$ definiciója és az összetett függvény deriválási szabálya alapján
  \[ F'(t) = \skalar{\grad \vfi(a+th)}{h} = \sum_{|i|=1} \partial^i \vfi(a+th)\cdot h^i\qquad(t\in[0,1])\]
  Tegyük fel, hogy $k\in\{1,\dotsc,s-1\}$ esetén igaz az állítás. Így $k+1$-re:
  \begin{align*}
    \dfrac1{(k+1)!}F^{(k+1)}(t) &= \dfrac1{(k+1)!}(F^{(k)})'(t)\\
    &=\dfrac1{k+1}\sum_{|i|=k} \dfrac1{i!}
    (\partial_1\partial^i\vfi(a+th)h^ih_1+\ldots + \partial_n\partial^i\vfi(a+th)h^ih_n) ={}\\
    &\stackrel{\text{\ref{te:youngkov} alapján}}{=}
    \sum_{|i|=k+1}\dfrac{\partial^i\vfi(a+th)}{i!}h^i\qquad(t\in[0,1])
  \end{align*}
\end{biz}

\subsection{Inverz függvények ($\RnRn$)}
\underline{Globális, $\R\n\R$-beli tétel:}\\
\[f\colon (a,b)\n\R,\ f\in\D,\ f'>0\ (a,b)$-n, ekkor $\exists f^{-1}$ inverz, ui $f\uparrow\]
Ez nem általánosítható, de a lokális változata igen.:\\\\
\underline{Lokális, $\R\n\R$-beli tétel:}\\
$f: I\n \R, I\subset\R$ intervallum\\
$f$ folytonosan deriválható\\
$a\in I$-ben $f'(a)\neq 0$\\
EKKOR $\exists U := K(a)$ és $\exists V:= K(f(a))$: $f_{|U}\colon U\n V$ bijekció $\nn \exists$ inverz\\
és az $f^{-1}$ inverz differincálható és $\left(f^{-1}\right)'(x) = \dfrac1{f'(f^{-1}(x))}\quad\forall x\in V$


\begin{te}[Inverz függvény tétel] $\Omega \subset \R^n$ nyílt, $a\in\Omega,\ f:\Omega\n\R^n$. Tfh:
  {\listazjromai \begin{enumerate}
  \item $f$ folytonosan deriválható $\Omega$-n
  \item $\det f'(a) \neq 0$
  \end{enumerate} }
  Ekkor
  { \listazjbetu \begin{enumerate}
  \item $\exists U := K(a)$ és $\exists V:= K(f(a))$ $f_{|U}\colon U\n V$ bijekció (azaz $\exists f^{-1}$ inverz)
  \item $f^{-1}$ deriválható, $(f^{-1})'(x) = \begin{bmatrix} f'(f^{-1}(x))\end{bmatrix}^{-1}\quad (\forall x\in V)$
  \end{enumerate}
  }
\end{te}
\textbf{Alkalmazás.} Nemlineáris egyenletrendszerek megoldhatósága\\
$\left.\!\begin{array}{rcl}
f_1(x_1,\dotsc,x_n) & = & b_1\\
\vdots & & \vdots\\
f_n(x_1,\dotsc,x_n) & = & b_n\end{array}\right\}\qquad \begin{array}{cl} \left.\begin{array}{c}f_i\in\RnR\\b_i\end{array}
\right\} &\text{adott}\\x_1,\dotsc,x_n&\text{ismeretlen}\end{array}$\\
\vphantom{x}\\
$f := \begin{pmatrix}f_1\\\vdots\\f_n\end{pmatrix}\quad b := \begin{pmatrix}b_1\\\vdots\\b_n\end{pmatrix}$.\\
Legyen $a=\begin{pmatrix}a_1\\\vdots\\a_n\end{pmatrix}\in\R^n\colon f(a)=b$.\\
Keresni kell egy ilyet. Ha $f'(a)$
invertálható ($\ekviv \det f'(a)\neq 0$) $\overset{\text{Inverz fv}}{\underset{\text{tétel}}{\nn}}\\
\exists K(b)\ \forall y=(y_1,\dotsc,y_n)\in K(b)\ f(x)=y$ egyenletrendszer $x$-re egyéretlműen megoldható.
\begin{megj}Létezést biztosít.\\
A fixpont-tétel segítségével közelítő megoldás adható\end{megj}


\begin{de}$f\in\R^2\n\R$ adott. Ha $\exists I\subset \R$ intervallum és $\exists \vfi: I\n\R$, hogy\\ $f(x,\vfi(x))=0
  (x\in I)$, akkor a $\vfi$ az $f(x,y) =0$ \emph{implicit egyenlet megoldása}.\\ (a $\vfi$ függvény az $f(x,y)=0$
  implicit alakban van megadva)
\end{de}

\newpage
\begin{te}[Implicit függvény-tétel, spec: 2 változós] $f\in\R^2\n\R^1,\\D_f=\Omega\subset\R^2$ nyílt. Tfh:
  {\listazjromai \begin{enumerate}
  \item $f$ folytonosan differenciálható $\Omega$-n
  \item $\exists (a,b) \in \Omega:\ f(a,b)=0$ és $\partial_2f(a,b)\neq 0$
  \end{enumerate}}
  Ekkor
  {\listazjbetu
    \begin{enumerate}
    \item $\exists K(a) =: U\colon\ \forall x\in K(a)$-hoz $\exists \vfi(x)\in \R$, melyre $f(x,\vfi(x)) = 0\quad(\forall
      x\in U)$
    \item A $\vfi\colon U\n\R$ fv folytonosan deriválható és
      \[\vfi'(x) = - \dfrac{\partial_1f(x,\vfi(x))}{\partial_2f(x,\vfi(x))}\quad \forall x\in U \tag{$*$}\label{eq:*}\]
    \end{enumerate}
  }
\end{te}

\begin{Megj}
\item $($\ref{eq:*}$)$-ról: Ha $\vfi$ deriválható: $f(x,\vfi(x))=0\quad(x\in I) \stackrel{\text{láncszabály}}{\nn}\\
  \partial_1f(x,\vfi(x))\cdot1 + \partial_2f(x,\vfi(x))\cdot\vfi'(x) = 0$ 
\item Általánosítás: $n_1,n_2\in\N;\ \Omega_1\subset\R^{n_1};\ \Omega_2\subset\R^{n_2}$ nyíltak. $a\in\Omega_1,\
  b\in\Omega_2,\\ f\colon \Omega_1\times\Omega_2\n\R^{n_2}$, ahol $\Omega_1 \times \Omega_2$ az első, illetve a második
  változócsoportot jelöli.\\
  Legyen $\partial_1f(a,b) := (\Omega_1\owns x\mapsto f(x,b)\in\R^{n_2})'_{x=a}$ az első változócsoport szerinti
  derivált, illetve\\$\partial_2f(a,b) := (\Omega_2\owns y\mapsto f(a,y)\in\R^{n_2})'_{y=b}$ az második
  változócsoport szerinti derivált
\end{Megj}

\begin{te}[Implicit függvény-tétel, ált] \ldots $f\in\Omega_1\times \Omega_2\n\R^{n_2}$. Tfh:
  \begin{enumerate}[\quad(i)]
  \item $f$ folytonosan differenciálható
  \item $\exists a \in \Omega_1,\ b\in\Omega_2:\ f(a,b)=0$ és $\det(\partial_2f(a,b))\neq 0$
  \end{enumerate}
  \noindent Ekkor
  {\listazjbetu
    \begin{enumerate}
    \item $\exists U_1:= K(a)\subset R^{n_1}$ és $\exists U_2:=K(b)\subset \R^{n_2}$:\\
      $\forall x\in U_1$-hoz $\exists! \vfi(x)\in U_2$, melyre $f(x,\vfi(x)) = 0$
    \item A $\vfi\colon U_1\n U_2$ fv folytonosan deriválható és
      \[\vfi'(x) = - [\partial_2f(x,\vfi(x))]^{-1}\cdot \partial_1f(x,\vfi(x))\quad\forall x\in U_1 \]
    \end{enumerate}
      }
\end{te}

\subsection{Szélsőértékek ($\R^n\n \R^1$)}
\begin{te}[Elsőrendű szükséges feltétel lokális szélsőértékre]
Tfh: \\$\vfi\in U\n\R,\ U\subset \R^n$ nyílt,
\begin{enumerate}
\item $\vfi \in \der a\quad a\in U$ (belső pont!!!)
\item $\vfi$-nek lokális szélsőértéke van $a$-ban
\end{enumerate}
Ekkor  \[\vfi'(a) = (\partial_1\vfi(a),\dotsc,\partial_n\vfi(a))=0\tag{$**$}\label{eq:**}\]

\end{te}
\begin{biz}Trivi, ui: $t\mapsto f(a+te_i)\ (\in\R\n\R)$ parciális függvénynek is lokális szélsőértéke van $t=0$-ban.
\end{biz}

\newpage
\begin{Megj}
\item Szükséges, de nem elégséges: $(n=1\colon f(x) := x^3)$
\item $($\ref{eq:**}$)\ekviv \left.\begin{array}{c}\partial_1\vfi(a)=0\\\partial_n\vfi(a)=0\end{array} \right\}$
  $n$ db egyenlet, $n$ db ismeretlen: $(a_1,\dotsc,a_n)$\\
  Itt lehet csak szélsőérték
\end{Megj}

\begin{de}
  $\vfi: U\n\R,\ U\subset \R^n$ nyílt, $a\in U$. A $\vfi$-nek az $a$-ban lokális minimuma [maximuma] van, ha $\exists
  K(a) (\subset U)\colon \vfi(a) \leqq \vfi(x) \ [\vfi(a) \geqq \vfi(x)]  \quad(x\in K(a))$
\end{de}
\begin{megj}
  Lokális szélsőérték $\ekviv$ lokális minimum vagy lokális maximum
\end{megj}

\begin{de}[Kvadratikus alak]\ 
  Az $A=[a_{ij}]\in\R^{n\times n}$ szimmetrikus mátrix,\\ $h=(h_1,h_2,\dotsc,h_n)\in\R^n$. A  $Q\colon \RnR$
  \[ Q(h) := \skalar{Ah}h = \sum_{i=1}^n a_{ij}h_ih_j\]
  fv-t az \emph{$A$ mátrix által meghatározott kvadratikus formának} nevezzük 
\end{de}

\begin{megj}
  $\di Q(h) = \sum_{|i|=2}a_ih^i\quad i=(i_1,\dotsc,i_n)$ multiindex\\
  Ez egy homogén n-változós másodfokú polinom.
\end{megj}
\begin{de}
  $A=[a_{ij}]\in\R^{n\times n}$ szimmetrikus.\\
  A $Q(h)$ kvadratikus forma (vagy az $A$ mátrix)
  \begin{itemize}[\quad]
  \item \underline{pozitív definit}, ha $Q(h)>0\quad\forall h\in\R^n\setminus\{0\}$
  \item \underline{negatív definit}, ha $Q(h)<0\quad\forall h\in\R^n\setminus\{0\}$
  \item \underline{pozitív szemidefinit}, ha $Q(h)\geq0\quad\forall h\in\R^n$
  \item \underline{negatív szemidefinit}, ha $Q(h)\leq0\quad\forall h\in\R^n$
  \end{itemize}
\end{de}


\begin{te}[Sylvester-kritérium]$Q(h) = \skalar{Ah}h$ kvadratikus alak,\\$A=[a_{ij}]\in\R^{n\times n}$ szimmetrikus
  mátrix
  
  \[A=\begin{bmatrix}a_{11} & a_{12} & \cdots & a_{1n}\\
  a_{21} & a_{22} & \cdots & a_{2n}\\ \vdots & \vdots & \ddots & \vdots \\
  a_{n1} & a_{n2} & \cdots & a_{nn}\end{bmatrix};\qquad \Delta_k = \det\begin{bmatrix}a_{11}&\cdots&a_{1k}\\
  \vdots &\ddots& \vdots\\a_{k_1} & \cdots & a_{kk}\end{bmatrix} \text{sarok-aldeterminánsok}\]
Ekkor
\begin{enumerate}
\item $Q$ pozitív definit $\ekviv\Delta_1>0,\,\Delta_2>0,\dotsc,$ azaz $\sgn\Delta_k=1\quad k=1,2,\dotsc,n$
\item $Q$ negatív definit $\ekviv\Delta_1<0,\,\Delta_2>0,\dotsc,$ azaz $\sgn\Delta_k=(-1)^k \quad k=1,2,\dotsc,n$
\end{enumerate}
\end{te}

\begin{te}
  Ha $Q$ kvadratikus forma $\nn\\\exists m,M\in\R\colon m\norma{h}^2\leq Q(h)\leq M\norma{h}^2\quad(h\in\R^n)$
\end{te}
\begin{biz}
  $Q\colon \RnR$ folytonos függvény, $H:=\{x\in\R^n,\norma x = 1\}$ kompakt $\stackrel{\text{Weierstrass}}{\nn}$\\
  \[\exists M:=\max\{Q(h) : \norma h = 1\},\ \exists m:=\min\{Q(h) : \norma h = 1\}\]
  DE!\\
  \[Q(h) =  Q\left(\norma{h}\cdot\dfrac{h}{\norma{h}}\right) = \norma{h}^2Q\left(\dfrac{h}{\norma{h}}\right)\ \ \nn\ 
  \ Q(h)\leq \norma{h}^2 M,\ Q(h)\geq m\norma{h}^2\]
\end{biz}

\begin{kov}
  $Q(h)$ kvadratikus forma,\\
  $Q$ pozitív definit $\ekviv \exists c_1>0\colon Q(h)\geq c_1\norma{h}^2$\\
  $Q$ negatív definit $\ekviv \exists c_2<0\colon Q(h)\leq c_2\norma{h}^2$\\
  Az előző tételből adódik
\end{kov}

\begin{te}[Másodrendű elégséges feltétel, lokális szélsőértékre]\ \\
  Tfh $\vfi\colon U\subset\R,\ U\n\R^n$ nyílt, $a\in U$ belső pont!!!
  {\listazjromai
    \begin{enumerate}
    \item $\vfi$ kétszer folytonosan deriválható
    \item $\vfi'(a)=0$
    \item $\vfi''(a)$ Hesse-féle mátrix által generált kvadratikus alak pozitív [negatív] definit.
    \end{enumerate}
}
Ekkor $\vfi$-nek $a$-ban lokális minimuma [maximuma] van.
\end{te}

\begin{megj} $\vfi''(a)=\ldots$ + Sylvester
\end{megj}
\begin{biz}$a,a+h\in K(a)$, $f\in\dern2x\ \forall x\in K(a)$\\
  Taylor-formula alapján $\exists \nu\in(0,1)$:
\begin{gather*}
  f(a+h)-f(a)=\sum_{i=1}^n\partial_if(a)\cdot h_i+\dfrac12\cdot\sum_{i,j=1}^n\partial_i\partial_jf(a+\nu h)h_ih_j=
  \\ = \sum_{|i|=1}\dfrac{\partial^i f(a)}{i!} + \sum_{|i|=2}\dfrac{\partial^i f(a+\nu h)}{i!} =\\
  =\dfrac12\sum_{i,j=1}^n\partial_i\partial_jf(a+\nu h)h_ih_j,\text{ ui }f'(a)=0\text{, így }\partial_if(a)=0.\\
  \epsilon_ij(h):=\partial_i\partial_jf(a+\nu h)-\partial_i\partial_jf(a)\quad(i,j=1,\dotsc,n)\\
  \text{Az első feltétel alapján} \lim_0\epsilon_{ij}=0\\
  f(a+h)-f(a)=\dfrac12(Q(h)+R(h))\\
  \intertext{ahol}
  R(h):=\sum_{i,j=1}^n\epsilon_{ij}(h)h_ih_j\\
  a+h\in K(a),\, h\ne 0\ \nn\ \vert R(h)\vert=\norma{h}^2\left\vert\sum_{i,j=1}^n\epsilon_{ij}(h)\dfrac{h_i}{ 
    \norma{h}}\dfrac{h_j}{\norma{h}}\right\vert\leq\norma{h}^2\cdot\sum_{i,j=1}^n|\epsilon_{ij}(h)|\\
  \nn \exists \delta>0\colon \forall h\in\R^n\ a+h\in K(a),\  \norma h<\delta:\\
  |R(h)|\leq \dfrac m2\norma{h}^2\quad m:=\min\{Q(h)\in\R:\norma h=1\}\\
  f(a+h)-f(a)\geq \frac m2\norma{h}^2-\frac m4\norma{h}^2=\frac m4\norma{h}^2
\end{gather*}
vagyis $f$-nek az $a$ pontban lokális minimuma van.
\end{biz}

\begin{te}
  $\vfi\colon U\n\R,\ U\subset \R^n$ nyílt, $a\in U$
{\listazjromai
\begin{enumerate}
\item $\vfi$ kétszer folytonosan deriválható
\item $\vfi'(a)=0$ és a $\vfi''(a)$ álatal generált kvadratikus forma indefinit
\end{enumerate}
}
Ekkor $a$-ban $\vfi$-nek nincs lokális szélsőértéke
\end{te}
\begin{biz}
  Ha indefinit: nem szemidefinit $\nn$ szükséges feltétel alapján nincs szélsőérték
\end{biz}

\subsubsection{Feltételes szélsőérték}
\begin{PlSS}
  Adott: $x+y-2=0$ egyenletű egyenes. Melyik rajta lévő $P$ pont esetén lesz $\overline{OP}$, azaz az origótól való
  távolság minimális? Azaz:\\
  $f(x,y) := x^2 + y^2\ (x,y)\in\R^2$\\
  $H:=\{(x,y)\in\R^2| x+y-2=0\}$\\
  $f_{|H}\n\min$
\end{PlSS}
\begin{PlSS}\label{plss:fsz2}
  Adott körbe maximális területű téglalapot kell tenni.\\
  Egyszerűsítés: elég az első síknegyed, mert szimmetrikus.
  $T(x,y) := 4xy$\\
  $H:=\{(x,y)\in\R^2| x^2 + y^2 - R^2 =0\}$\\
  $f_{|H}\n\max$
\end{PlSS}
Adott $m,n\in\N,\ U\subset\R^n$ nyílt,\\
$f\colon U\n\R$ és\\
$g_i\colon U\n\R\quad i=1..m$\\
$H:=\{ z\in\R^n\,|\,g_i(z)=0,\ i=1,\dotsc,m\}$ feltételek.\\
Határozzuk meg $f_{|H}$ lokális szélsőértékeit.

\begin{de}
  Az $f\colon U\n\R$ fv-nek  a $c\in U$-ban a $g_i(z)=0\ \,(i=1..m)$ feltételekre vonatkozó \emph{lokális feltételes
    minimuma van}, ha
  \[\exists K(c): f(x) \geq f(c)\quad \forall x\in K(c)\cap H\]
\end{de}

\begin{te}[Lokális feltételre vonatkozó szükséges feltétel]
  Tfh
{\listazjbetu
  \begin{enumerate}
  \item $n,m\in\N,\ U\in\R^n$ nyílt,\\$f\colon U\n\R,\ g_i: U\n\R\ \ (i=1,\dotsc,m)$ folytonosan deriválhatóak.
  \item $f$-nek a $c\in U$-ban a $g_i(c)=0\ \,(i=1,\dotsc,m)$ feltételekre vonatkozó lokális szélső
  \item $g_i'(c)\ (i=1,\dotsc,m)$ lineárisan független vektorok
  \end{enumerate}
}
  Ekkor $\exists\lambda_1,\dotsc,\lambda_m\in\R\colon \di F:= f+\sum_{i=1}^m \lambda_ig_i$ fv-nek $c$-ben  $F'(c) = 0$
\end{te}
\begin{megj}Alkalmazáshoz:
  \[ \left.\begin{array}{l}
    \left.\begin{array}{c}\partial_1F(c)=0\\\partial_2F(c)=0\\\vdots\\\partial_nF(c)=0\end{array}\right\} \text{ n db}\\
      \left.\begin{array}{c}g_1(c)=0\\\vdots\\g_m(c)=0\end{array}\right\} \text{ m db}\\
  \end{array}\right\} \text{ n+m db egyenlet a $\lambda_1,\dotsc,\lambda_m$ és $c_1,\dotsc,c_n$ ismeretlenekre} 
  \tag{$\sharp$}\label{eqs:lagrange-mult}
  \]
  Lokális szélsőérték csak ilyen $c$-ben lehet	
\end{megj}


\textbf{Alkalmazás}\\
\Aref{plss:fsz2} példa:
\begin{gather*}
  f(x,y) := 4xy\quad (\,(x,y)\in\R^2\,)\\
  g(x,y) := x^2 + y^2 -R^2\\
  F(x,y) := 4xy + \lambda(x^2 + y^2 - R^2)\\
  \left.
  \begin{gathered}
    \partial_1 F(x,y) = 4y + 2\lambda x = 0\\
    \partial_2 F(x,y) = 4x + 2\lambda y = 0  
  \end{gathered}\,
  \right\}\quad \oplus\colon 2(x+y)(\lambda+2) = 0\\
  x^2+y^2 - R^2 = 0 
  \intertext{Innen:}
  \lambda = -2\\
  x=y=\dfrac{R}{\sqrt{2}}
\end{gather*}
Azaz $\left(\dfrac{R}{\sqrt{2}},\,\dfrac{R}{\sqrt{2}}\right)$-ben lehet szélsőérték.

\begin{te}[Elégséges feltétel a lokális szélsőértékre]\ 
{\listazjbetu
  \begin{enumerate}
  \item $f,g_i\colon U\n\R$, $U\subset\R^n$ nyílt, $i=1,\dotsc,m$ kétszer folytonosan differenciálható
  \item $c=(c1,\dotsc,c_n)$, $\lambda_1,\dotsc,\lambda_m$ kielégíti \aref{eqs:lagrange-mult}-t
  \item $F:= f + \di\sum_{i=1}^m\lambda_i g_i$-nek $c$-ben lokális szélsőértéke van (feltétel nélküli: a teljes
    értelemezési tartományt figyelembe véve).  
  \end{enumerate}
}
Ekkor $f$-nek $c$-ben a $g_1=\dotsb=g_m=0$ feltételekre vonatkozó feltételes lokális szélsőértéke van.
\end{te}
\ref{plss:fsz2}: $\lambda=2$; $F(x,y) = 4xy - 2(x^2 +y^2 - R^2) = 2R^2 - 2(x+y)^2$.



% Local Variables:
% fill-column: 120
% TeX-master: t
% End:

  \newpage
  \section{Paraméteres integrál}
\begin{te}
  $I=[a,\,b],\ U\subset \R^n$ nyílt és $f\colon U\times[a,b]\n\R$ folytonos,\\
  $\vfi(x) := \di\Int_a^bf(x,t) \diff t\quad(x\in U)\quad$ az $f$ paraméteres integrálja.\\
Ekkor:
{\listazjbetu
\begin{enumerate}
\item  $\vfi\colon U\n\R$ is folytonos
\item Ha $\exists \partial_if$ és $\partial_if\in \Folyt\quad (i=1,\dotsc,n)$, akkor $\vfi$ is deriválható és
\begin{gather*} \di \partial_i\vfi(x) = \Int_a^b\partial_if(x,t)\diff t\quad x\in U\end{gather*}
\end{enumerate}
}
\end{te}
\begin{pl}
  \begin{gather*}
    \vfi(x) := \Int_0^1 ln(t^2+x^2) \diff t\quad(x>0).\\
    \vfi \in\Der \text{ és } \vfi'(x) = \Int_0^1\dfrac{\partial}{\partial x} ln(t^2+x^2) \diff t =
    \Int_0^1 \dfrac{2x}{x^2+t^2} \diff t= \dfrac{2x}{x^2}\Int_0^1 \dfrac1{1+\frac t x} \diff x = {}\\
	{}=\dfrac2x[x \cdot \arctg\dfrac t x]^1_{t=0} = 2 \arctg \dfrac1x
  \end{gather*}
\end{pl}
\newpage
\section{Vonalintegrál ($\R^n\to\R^n$ függvényekre)}
\subsection{Sima utak, görbék}
\begin{de}[Sima út]$n\in\N\ \vfi\colon[a,\,b]\to\R^n$ folytonosan deriválható függvényt\\\emph{$\R^n$-beli sima út}nak
    nevezzük.\\  Az $R_\vfi = \Gamma\subset\R^n$ halmaz \emph{sima görbe}, $\vfi$ a $\Gamma$ görbe egy paraméterezése.
\end{de}

\begin{de}[Szakaszonként sima út]$a,b\in\R;\ a\leq b$. A $\vfi\colon[a,b]\to \R^n$ függvény  \emph{$\R^n$-beli
    szakaszonként sima út}, ha 
{\listazjromai
  \begin{enumerate}
  \item $\vfi\in\Folyt$
  \item $\exists a=t_0<t,\ 1<\dotsb<t_m=b$: $\vfi_{|[t_i,\,t_i+1]}\ \ i=1,\dotsc,m-1$ sima út.
\end{enumerate}
}
\end{de}

\begin{Pl}
\item Szakasz: $a,b\in \R^n\ \vfi(t) := a+t(b-a)\quad(t\in[0,\,1])$
\item Töröttvonal - szakaszonként sima út
\item Kör: $\vfi(t) := (\sin t,\cos t)\quad t\in[0,2\pi]\\
  R_\vfi=\Gamma$
\end{Pl}


\begin{de}[Szakaszonként sima utak egyesítése]
  $\vfi\colon [a,a+h]\to\R^n$\\$\psi\colon[b,b+k]\to\R^n$ szakaszonként sima utak, és tfh: $\vfi(a+h)=\psi(b)$, azaz
  $\vfi$ végpontja megegyezik $\psi$ kezdőpontjával. \\
  A $\vfi$ és $\psi$ egyesítése $(\vfi\cup\psi)$:
\[\Phi(t) = \begin{cases}\vfi(t) & t\in[a,a+h]\\\psi(t-a-h+b) & t\in[a+h,a+h+k]\end{cases}\]
\end{de}

\begin{de}[$\vfi$ ellentettje] $\widetilde{\vfi} := \vfi(2a+h-t)\qquad(t\in[a,\,a+h])$\\
  az út $a+h\to a$ irányú lett.
\end{de}

\subsection{Vonalintegrál definíciója}
\begin{te}Legyen $U\subset \R^n$ nyílt.\\
  $U$ összegüggő $\ekviv \forall x,y\in  U$ összeköthető $U$-beli szakaszonként
  sima úttal.
\end{te}

\begin{de}[Tartomány]Az $U\subset \R^n$ halmaz \emph{tartomány}, ha
{\listazjromai
  \begin{enumerate}
    \item $U$ nyílt $\R^n$-ben
    \item $U$ összefüggő
  \end{enumerate}
}
\end{de}
\begin{de}[Úton vett vonalintegrál]
  Legyen $U\subset \R^n$ tartomány, $f\colon U\to\R^n$ \underline{folytonos}, $\vfi\colon [a,b]\to\R^n$ szakaszonként
  sima. Ekkor
\[\Int_a^b\skalar{f\circ\vfi}{\vfi'} = \Int_a^b\skalar{f(\vfi(t))}{\vfi'(t)\,}\diff t =: \Int_\vfi f\]
szám az $f$ függvény $\vfi$ útra vett vonalintegrálja.
\end{de}
\newpage
\begin{Megj}
  \item $f$ folytonos $\nn$ az integrandus folytonos $\nn$ az integrál létezik.
\item $n=1,\ \vfi(t) := t\quad t\in[a,b]$
\[\Int_\vfi f \text{ az } \Int_a^bf(t)\diff t \text{ Riemann-integrálja}\]
\end{Megj}

\begin{te}[A vonalintegrál egyszerű tulajdonságai]
  $U\subset \R^n$ tartomány,\\$\vfi\colon [a,a+h]\to\R^n$ és $\psi\colon [b,b+k]\to\R^n$ szakaszonként sima utak,
  $\vfi(a+h) = \psi(b)$.\\$f,g\colon U\to\R^n$ folytonos. Ekkor
  \begin{enumerate}
  \item $\di\Int_\vfi(\lambda_1 f +\lambda_2g) = \lambda_1\Int_\vfi f+ \lambda_2\Int_\vfi g$
  \item $\di\Int_\vfi f = -\Int_{\widetilde{\vfi}}$\qquad (ellentett út)
  \item $\di\Int_{\vfi\cup\psi}\!\!\! f = \Int_\vfi f + \Int_\psi f$
  \item $\di\Big\vert \Int_\vfi f\Big\vert \leqq M\cdot l(\vfi)$, ahol $l(\vfi)$ a $\vfi$ (vagy a $\Gamma$ görbe)
  hossza és $M:= \max \{\,\norma{f(x)}_2:x\in R_\vfi\}$
  \end{enumerate}
\end{te}

\subsection{Primitív függvények}
\begin{de}[Primitív függvény]$U\subset\R^n$ tartomány, $f\colon U\to\R^n$.\\
  Az $F\colon U\to\R^n$ függvény az $f$ primitív függvénye, ha
  {\listazjromai
    \begin{enumerate}
    \item $F\in\Der$
    \item $F'(x) = f(x)\quad (\forall x\in U)$
    \end{enumerate}
  }
\end{de}

\begin{megj}
  Ha $F\in\Der$: $F'=(\partial_1F,\dotsc\partial_nF) =(f_1,\dotsc,f_n)=f$ 
\end{megj}

\begin{te}\ 
  \begin{enumzjromai}
  \item Ha $F\colon U\to\R$ az $f$ primitív függvénye $\nn \forall c\in\R\colon F+c$ is az
  \item Ha $F_1,\,F_2\colon U\to\R$ az $f$ primitív függvényei $\nn \exists c\in\R\colon F_1(x)-F_2(x) = c \quad \forall
  x\in U$
  \end{enumzjromai}
\end{te}
\begin{te}[Newton-Leibniz]
  Tfh:
\begin{enumzjromai}
  \item $U\subset \R^n$ tartomány
  \item $f\colon U\to \R^n$ folytonos
  \item $\vfi\colon [a,b]\to U$ szakaszonként sima út
  \item $f$-nek $\exists F$: a primitív fv-e
\end{enumzjromai}
Ekkor $\di\Int_\vfi f = F(\vfi(b))-F(\vfi(a))$
\end{te}
\begin{biz}
\begin{gather*}\text{(ii)}\nn a=t_0<t_1<\dotsb<t_m=b\ (m\in\N).\ \ F\circ\vfi\in\Der[t_{i-1},t_i]\ (i=1,\dotsc,m).\\
  (F\circ\vfi)'(t)=\skalar{F'(\vfi(t))}{\vfi'(t)}=\skalar{(f\circ\vfi)(t)}{\vfi'(t)}\quad(t\in[t_{i-1},\,t_i],\ 
  i=1,\dotsc,m)\\
  \intertext{Ezekre az intervallumokra alaklmazva az egyváltozós Newton-Leibniz formulát}
  \Int_\vfi f=\sum_{i=1}^m\Int_{t_{i-1}}^{t_i}\skalar{(f\circ\vfi)(t)}{\vfi'(t)}\diff t = \sum_{i=1}^m
  \left(F(\vfi(t_i)-F(\vfi(t_{i-1}))\right) = F(\vfi(b))-F(\vfi(a))
\end{gather*}
\end{biz}

\begin{pl}
  Primitív fv. meghatározása\\
  $f(x,y) = \left(\dfrac{y}{x^2},\,-\dfrac1x\right)\qquad(x>0,y>0)$\\
  Létezik-e $F\colon \R^2\to\R$, hogy $F'=f$.\\
  $\left.\!\begin{gathered}\partial_xF(x,y) = \dfrac{y}{x^2}\\\partial_yF(x,y)=-\dfrac1x+h'(y)\end{gathered}
  \right\}\nn F(x,y) = -\dfrac y x+h(y)$\\
  $h'(y) = 0\nn h\equiv$ állandó\\
  így $F(x,y) = -\dfrac y x +c\quad (x,y)>0$
\end{pl}

\begin{te}
  $U\subset \R^n$ taromány, $f\colon U\to\R$ folytonos.\\
  $f$-nek létezik primitív függvénye $\ekviv \left\{\begin{array}{l}\forall U\text{-ban haladó szakaszonként sima és
  zárt }\vfi\text{ útra:}\\\di\Int_\vfi f= 0\end{array}\right.$
\end{te}

\begin{megj}Jelölés: $\oint$: zárt útra vett integrál, körintegrál.
\end{megj}

\begin{biz}
  $\underline{\Rightarrow:}$ trivi. Newton-Leibniz: $\di\oint f=F(\vfi(b)) - F(\vfi(a))$, de $\vfi$ zárt:
  $\vfi(a)=\vfi(b)$
  $\underline{\Leftarrow:}$ Több lépésben
  \begin{enumzjbetu}
    \item Ha $\di \Oint_\vfi f=0\nn \forall x,a\in U$ és $\forall \vfi_1,\vfi_2$, ami $x$-et, $a$-t összeköti:
    $\di \Oint_{\vfi_1} f = \Oint_{\vfi_2} f$: a vonalintegrál független a két pontot összekötő útttól, ugyanis\\
      $\vfi_1\cup\widetilde{\vfi}_2$ zárt görbe, $\di 0=\Int_{\vfi_1\cup\widetilde{\vfi}_2}\!\!\!\!f = \Int_{\vfi_1} f+
    \Int_{\widetilde{\vfi}_2}\!f\quad\nn\quad \Int_{\vfi_1}\!f - \Int_{\vfi_2}\!f = 0$
  \item Ha $\di\forall \Oint_\vfi f=0$, akkor definiálhatjuk a következő függvényt:
    \[\di a\in U \text{ rögzített; }\Phi(x) := \Int_{\overline{ax}}f(x)\quad\forall x\in U\]
    ahol $\overline{ax}$ az $a$-t $x$-szel összektő szakaszonként sima út. Ez a függvény az $f$-nek $a$-ban eltűnő
    integrálfüggénye.
  \item Ha $\di\forall\Oint_\vfi f=0\,\nn$ a fenti $\Phi$ integrálfüggvény deriválható és $\Phi'=f$, azaza a $\Phi$
    integrálfüggvény az $f$ egy primitív függvénye (az integrálfüggvény deriválhatóságára vonatkozó tétel alapján)
    \begin{gather*}
       \Phi(x+h)-\Phi(x)=\Int_0^1\skalar{f(x+th}h\diff t
       \intertext{$f$ folytonos, ezért}
       \epsilon(h):=\sup{\norma{f(x+th)-f(x)}:0\leq t\leq 1}\to 0\quad(h\to0).
       \intertext{Továbbá}
       \left\vert\Phi(x+h)-\Phi(x)-\skalar{f(x)}h\right\vert = \left\vert\Int_0^1\skalar{f(x+th)-f(x)}h\diff t
       \right\vert \leq\epsilon(h)\cdot\norma h.
    \end{gather*}
    Vagyis $\Phi$ differenciálható és $\Phi'=f$.
  \end{enumzjbetu}
\end{biz}
\begin{te}[Szükéges feltétel a primitív függvény létezésére]
  $U\subset\R^n$ tartomány,\\$f\colon U\to\R^n$
  \begin{enumzjr}
    \item $f$ \underline{deriválható}
    \item $f$-nek létezik primitív függvénye
  \end{enumzjr}
  Ekkor $f'$ deriváltmátrix szimmetrikus, azaz $\partial_if_j=\partial_jf_i\ (\forall 1\leq i,j\leq n)$ és
  $f=(f_1,\dotsc,f_n)$
\end{te}

\begin{biz}(ii) $\nn \exists F\colon U\to\R,\ F\in\Der$ és $F'=f$\\
  (i)$\nn F\in\Der^2\overset{\text{Young-t.}}{\nn} \partial_i(\underbrace{\partial_jF}_{f_j}) =
  \partial_j(\underbrace{\partial_iF}_{f_i})$\\
$F'=(\partial_1,\dotsc,\partial_n)=(f_1,\dotsc,f_n)$
\end{biz}

\begin{Megj}
\item $\R\to\R$ esetén $\forall$ folytonos függvénynek létezik primitív függvény\\
  Ha $n\geq 2$, akkor $\exists f$ deriválható, melynek nincs primitív függvénye.
\item Csillagtartományon ez a szükséges felétel elégséges is
\end{Megj}
\begin{de}[Csillagtartomány]
  $U\subset \R^n$ az $a\in U$ pontra nézve csillagtartomány, ha $\forall x\in U: [a,x]\subset U$
\end{de}

\begin{te}[Elégséges feltétel a primitív függvény létezésére]
  Tfh:
  \begin{enumzjr}
  \item $U\subset \R^n$ az $a\in U$-ra csillagtartomány
  \item $f\colon U\to\R^n$ folytonosan deriválható
  \item $f'$ deriváltmátrix szimmetrikus
  \end{enumzjr}
  Ekkor $F$-nek $\exists$ primitív függvénye, az
  \[\di U\owns x\mapsto\!\!\Int_{[a,x]}\!\!\!f\]
  az  $f$ egy $a$-ban eltűnő primitív függvénye
\end{te}

\begin{biz}
  Megmutatjuk, hogy
\begin{gather*}
  U\owns x\mapsto F(x):=\Int_a^xf=\Int_0^1\skalar{f(a+t(x-a))}{x-a}\diff t
  \intertext{függvény differenciálható és $F'=f$.}
  \partial_iF(x) = \Int_0^1\left(\sum_{j=1}^nt\cdot\partial_if_j(a+t(x-a))\cdot (x_j-a_j)+f_i(a+t(x-a))\right)\diff t
  \intertext{$f'$ szimmetrikus, így}
  \partial_iF(x) = \Int_0^1\left(t\cdot \sum_{j=1}^n\partial_jf_i(a+t(x-a))(x_j-a_j)+f_i(a+t(x-a))\right)\diff t = \\
  \hspace*{2em}= \Int_0^1\left(t\cdot\dfrac{\partial}{\partial t}f_i(a+t(x-a))+f_i(a+t(x-a))\right)\diff t =\\
  \hspace*{2em}=f_i(x)-\Int_0^1f_i(a+t(x-a))\diff t + \Int_0^1f_i(a+t(x-a))\diff t =f_i
\end{gather*}  
\end{biz}

\newpage
\section{Többszörös integrál}
\subsection{A többszörös integrál fogalma}
Két lépésben: $N$-dimeziós intervallum, majd tetszőleges $H\subset\R^N$ korlátos halmaz

\subsubsection{$N$-dimeziós intervallum és felosztása}
\begin{de}
  $I^j:=[a^j,\,b^j]\subset \R\quad \forall j=1,\dotsc,N$\\
  $I := I^1\times I^2\times\dotsb\times I^N\quad\R^N$-beli kompakt intervallum\\
  $\mu(I) := (b^1-a^1)\cdot (b^2-a^2)\dotsm(b^N-a^N)$ az $I$ mértéke.
\end{de}
\begin{de}[Felosztás - egydimenziós]
  $[a,b]\subset \R,\\\tau := \{\,[x_{r-1},\,x_r]:r=1,2,\dotsc,m\}\in\mathcal{F}([a,b])$
\end{de}
\begin{de}[Felosztás]
  Legyen $\tau_j\in\mathcal{F}(I^j)\quad (j=1,\dotsc,N)$.\\
  $\tau := \tau_1\times\tau^2\times\dotsb\times\tau^N=\{I^1_{r_1}\times I_{r_2}^2\times I^N_{r_N}:1\leq r_j\leq m\}$ az
  $I$ intervallum egy felosztása\\
  Jel: $\mathcal{F}(I)$ a felosztás egy halmaza
\end{de}

\begin{de}
  $f\colon I\to\R,\ I\subset \R^N$ kompakt intervallum, $f$ korlátos, $\tau\in\mathcal{F}(I)$.
  \begin{gather*}
    s(f,\tau):=\sum_{J\in\tau}(\inf f_{|J})\cdot \mu(J)\\
    S(f,\tau) :=\sum_{J\in\tau}(\sup f_{|J})\cdot \mu(J)
  \end{gather*}
  az $f$ függvény $\tau$ felosztáshoz tartozó alsó- illetve felső közelítő összege.
\end{de}
\begin{megj}
$\{\,s(f,\tau):\tau\in\mathcal{F}(I)\}$ és $\{\,S(f,\tau):\tau\in\mathcal{F}(I)\}$ korlátosak
\end{megj}
\begin{de}
  $f\colon I,\to \R,\ I\subset \R^N$ kompakt intervallum, $f$ korlátos függvény.
  \begin{gather*}
    \sup\, \{\,s(f,\tau):\tau\in\mathcal{F}(I)\,\} =: I_*f\\
    \inf\, \{\,S(f,\tau):\tau\in\mathcal{F}(I)\,\} =: I^*f
  \end{gather*}
  az $f$ Darboux-féle alsó-, ill felső integrálja.
\end{de}

\begin{de}Az $f$ Riemann-integrálható, jel: $f\in\Rint(I)$, ha $I_*f=I^*f=\Int_If$.\\
  $\di\Int_If$ az $f$ Riemann integrálja $I$-n.
\end{de}

\begin{megj}$I_*f$, $I^*f$ létezik minden ilyen függvényre
\[I_*f\leq I^*f,\text{ ui }\forall\tau,\sigma\in\mathcal{F}(I)\colon s(f,\tau)\leq S(f,\sigma)\]
\end{megj}

\subsection{A többszörös integrál alapvető tulajdonságai}
\begin{te}
  $I\subset\R^N$ kompakt, $f,g\colon I\to\R$ korlátos függvények, $f,g\in\Rint(I)$. Ekkor
\begin{enumzjb}
  \item $f+g\in\R(I)$, és $\di\Int_I (f+g) = \Int_I f + \Int_I g$
  \item $\forall \lambda\in\R\colon \lambda f\in\Rint(I)$ és $\di\Int_I(\lambda f)=\lambda\Int_I f$
\end{enumzjb}
\end{te}

\begin{pl}Nem integrálható a következő függvény, ahol $(x,y)\in [0,1]\times[0,1]=:I$.\\
  $f(x,y):=\begin{cases}0 & x,y \text{ racionális }\\1 &\text{ különben}\end{cases}$\\
  $f\not\in\Rint(I)$, ui $I_*f=0\neq I^*f=1$
\end{pl}
\begin{te}[Egyenlőtlenség]$I\subset\R^N$ kompakt intervallum, $f,g\colon I\to\R$ és tfh:\\$f,g\in\Rint(I)$. Ekkor
  \begin{enumzjb}
    \item $f\leq g\ I$-n $\di\nn \Int_If\leq \Int_Ig$
    \item $\vert f\vert\in\Rint(I)$ és $\di\left\vert\Int_If\right\vert\leq\Int_I\vert f\vert$
    \item \textbf{első középértéktétel} Ha $g\geq 0\ I$-n:
      \begin{gather*}
	m\Int_Ig\leq \Int_I(fg)\leq M\Int_Ig\\
	M:=\sup_If\\
	m:=\inf_If
      \end{gather*}
    \item \textbf{második középértéktétel} Ha  még $f\in\Folyt(I)$ is:
      \[\exists \xi \in I\quad \Int_I fg = f(\xi)\Int_Ig\quad g\geq 0\ I\text{-n}\]
  \end{enumzjb}
\end{te}
\begin{biz}Mint $\R\to\R$\end{biz}

\begin{te}$f\colon I\to\R,\ I\subset\R^N$ kompakt intervallum\\
  $f$ folytonos $\nn$ $f$ integrálható $I$-n.
\end{te}
\begin{biz}Lásd $\R\to\R$\end{biz}

\subsection{Az integrál értelmezése tetszőleges $H\subset \R^N$ korlátos tartományon}
\begin{de}Legyen $H\subset\R^N$ \underline{korlátos} halmaz, $f\colon H\to\R$. Ekkor $\exists I\subset\R^N$ kompakt
  intervallum, hogy $H\subset I$. Az $f$ függvény Riemann-integrálható a $H$-n, ha az
\[ \tilde{f}(x) := \begin{cases}f(x)& x\in H\\0& x\in I\setminus H \end{cases}\]
függvény integrálható az $I$-n. Ekkor
\[ \Int_Hf := \Int_I\tilde{f}\]
\end{de}

\begin{Megj}
  \item A definíció nem függ az $I$ megválasztásától
    \item A $H$-n vett integrálokra is érvényesek az alapvető tulajdonságokra kimondott állítások.
\end{Megj}

\subsection{Az integrál számítása}
$N=2$-re, $N\geq 2$ esetén hasonlóan.\\
Három eset:
\begin{enumerate}
\item $D_f$ kompakt intervallum (szukcesszív: egymás utáni integrálással)
\item $D_f$ ún. normáltartomány (szukcesszív integrálással)
\item $D_f$ egyéb tartomány (integrál-transzformációval)
\end{enumerate}

\subsubsection{Intervallumon}
$N=2,\ D_f=[a,b]\times[c,d]=:I$, $f\colon I\to\R$ korlátos függvény. Vesszük a tengelyekkel párhuzamos megtszetgörbéket:
$\forall x\in[a,b]\colon\vfi_x(y):=f(x,y)$, ahol $y\in[c,d]$. $\vfi_x\colon[c,d]\to\R$


\begin{te}Ha $f\in\Rint(I)$, akkor az
\[M(x):=I^*\vfi_x\text{ és }m(x):=I_*\vfi_x\quad x\in[a,b]\]
függvények integrálhatók $[a,b]$-n és
\[\Int_If=\Int_a^bM(x)\diff x=\Int_a^bm(x)\diff x\]
\end{te}

\begin{ko}
  Ha $f\in\Rint(I)$ és még tfh: $\forall x\in[a,b]\colon \vfi_x\in\Rint[c,d]$, azaz $I^*\vfi_x=I_*\vfi_x$, akkor
  \[\Int_If=\Int_a^b\left(\Int_c^df(x,y)\diff y\right)\diff x\]
  Jel: $\di\Int_If=\Int_a^b\!\!\Int_c^df(x,y)\diff x\diff y$
\end{ko}
\begin{ko}  Ha
  \begin{enumzjr}
  \item $f\in\Rint(I)$
  \item $\forall x\in[c,d]\ \vfi_x\in\Rint[c,d]$
  \item $\forall y\in[c,d]\ \vfi_y(x):=f(x,y)\quad x\in[a,b]$ függvény integrálható $[a,b]$-n
  \end{enumzjr} Ekkor
  \[\Int_I f=\Int_a^b\left(\Int_c^df(x,y)\diff y\right)\diff x=\Int_c^d\left(\Int_a^bf(x,y)\diff x\right)\diff y\]
  azaz a változók felcserélhetók
\begin{pl}
\begin{gather*}
  f(x,y):= x^2y,\ (x,y)\in[0,1]\times[0,1]\\
  \Int_0^1\!\!\Int_0^1f(x,y)\diff x\diff y=\Int_0^1\left(\Int_0^1f(x,y)\diff y\right)\diff x =\Int_0^1\left[
    \dfrac{x^2y^2}2 \right]_{y=0}^{y=1}\diff x=\Int_0^1\dfrac{x^2}2\diff x=\\\hspace{1em}=
  \left[\dfrac{x^3}6\right]_0^1=\dfrac16
  \intertext{illetve}
  \Int_0^1\!\!\Int_0^1f(x,y)\diff x\diff y=\Int_0^1\!\left(\Int_0^1f(x,y)\diff x\right)\diff y =\Int_0^1\left[
    \dfrac{x^3y}3 \right]_{x=0}^{x=1}\diff y=\Int_0^1\dfrac{y}3\diff y=\left[\dfrac{y^2}6\right]_0^1=\dfrac16
\end{gather*}
\end{pl}
\end{ko}

\subsubsection{Normáltartományon}
\begin{de}\ 
  \begin{enumzjb}
  \item $x$-re nézve normáltartomány: $\vfi_1(x)\leq\vfi_2(x)\quad(x\in[a,b])$
    \[H:=\{(x,y)\in\R^2:a\leq x\leq b;\  \vfi_1(x)\leq y\leq\vfi_2(x)\}\]
  \item $y$-ra nézve normáltartomány: $\psi_1(y)\leq\psi_2(y)\quad(y\in[c,d])$
    \[K:=\{(x,y)\in\R^2:c\leq y\leq d;\  \psi_1(y)\leq x\leq\psi_2(y)\}\]
  \end{enumzjb}
\end{de}

\begin{te}\ 
  \begin{enumzjb}
    \item $H\subset \R^2\ x$-re nézve normáltartomány, $f\colon H\to\R$  folytonos. Ekkor $f\in\Rint(H)$ és
      \[\Int_Hf=\Int_a^b\left(\Int_{\vfi_1(x)}^{\vfi_2(x)}\!\!f(x,y)\diff y\right)\diff x\]
    \item $K\subset \R^2\ y$-ra nézve normáltartomány, $f\colon K\to\R$  folytonos. Ekkor $f\in\Rint(K)$ és
      \[\Int_Kf=\Int_c^d\left(\Int_{\psi_1(y)}^{\psi_2(y)}\!\!f(x,y)\diff x\right)\diff y\]
  \end{enumzjb}
\end{te}

\subsection{Integrál-transzformációval}
Bármilyen egyéb tartományon integrál-transzformációval téglalapra vett integrálásra vezetjük vissza a kiuszámítását
(helyettesítéses integrálás).

\begin{pl}
\begin{gather*}
  x=r\cos\vfi=\Phi_1(r,\vfi)\\
  y=r\sin\vfi=\Phi_2(r,\vfi)\\
  \R^2\to\R^2\owns\Phi:=(\Phi_1,\Phi_2)\colon U\to V\text{ folytonosan deriválható bijektív leképezés}\\
  \det\Phi'(r,\vfi)=\det\begin{bmatrix}\cos\vfi & -r\sin\vfi\\\sin\vfi & r\cos\vfi\end{bmatrix} = r \ne 0\\
  \intertext{Ezért $\exists\Phi^{-1}\nn \Phi$  bijektív}
  \intertext{Igazolható:}
  \iint\limits_Vf(x,y)\diff x\diff y=\Int_V f=\iint\limits_U f(r\cos\vfi,r\sin\vfi)\cdot\underbrace{
  \vert\det\Phi'(r,\vfi)\vert }_{\underset{r}{\scriptsize{\Vert}}}\diff r\diff \vfi  
\end{gather*}
\end{pl}
\begin{te}[Integrál-transzformáció]
  Legyen $\Phi:=(\Phi_1,\,\Phi_2)\colon U\to V\quad (U,V\in\R^2)$ folytonosan deriválható és $\det\Phi' \ne0\ \,U$-n
  (vagyis $\Phi$ bijekció).\\
  Ha  $f\colon V\to\R$ folytonos $\nn f\in\Rint(V)$ és
  \[\Int_Vf=\Int_Uf\circ\Phi\cdot\vert\det\Phi'\vert\]
\end{te}

% Local Variables:
% fill-column: 120
% TeX-master: t
% End:

\end{document}

% Local Variables:
% fill-column: 120
% TeX-master: t
% End:
