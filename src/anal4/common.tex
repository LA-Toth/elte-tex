

%
% - ---- -- PACKAGES--------------------------
%

\usepackage{amssymb}
\usepackage{amsmath}
%\usepackage[T1]{fontenc}
\usepackage[utf8]{inputenc}
\usepackage[magyar]{babel}
%\usepackage{amsthm}
\usepackage{theorem}
\usepackage{fancyhdr}
\usepackage{lastpage}
\usepackage{paralist}

%
% ---------- CODES --------------------------
%
\makeatletter
\gdef\th@magyar{\normalfont\slshape
  \def\@begintheorem##1##2{%
  \item[\hskip\labelsep \theorem@headerfont ##2.\ ##1.]}%
  \def\@opargbegintheorem##1##2##3{%
  \item[\hskip\labelsep \theorem@headerfont ##2. ##1.\ (##3)]}}
\makeatother


%
% ------------  N E W  C O M M A N D S --------
%
%\newcommand{\ob}{\begin{flushright} \leavevmode\hbox to.77778em{\hfil\vrule
%    \vbox to.675em{\hrule width.6em\vfil\hrule}\vrule}\hfil\end{flushright}}
\newcommand{\ob}{\hfill$\square$}
\newcommand{\ff}{f\in\mathbb{R}\rightarrow\mathbb{R}}
\newcommand{\fab}{f\colon (a,b)\rightarrow\mathbb{R}}
\newcommand{\fabk}{f\colon \left[a,b\right]\rightarrow\mathbb{R}}
\newcommand{\fir}{f\colon I\rightarrow\mathbb{R}}
\newcommand{\fdab}{f\in D(a,b)}
\newcommand{\fcab}{f\in C[a,b]}
\newcommand{\exist}{\exists}
\newcommand{\ek}{\Longleftrightarrow}
\newcommand{\la}{\lambda}
\newcommand{\ro}{\varrho}
\newcommand{\K}{\ensuremath{\mathbb{K}}}
\newcommand{\R}{\ensuremath{\mathbb{R}}}
\newcommand{\Q}{\ensuremath{\mathbb{Q}}}
\newcommand{\N}{\ensuremath{\mathbb{N}}}
\newcommand{\C}{\ensuremath{\mathbb{C}}}
\newcommand{\n}{\ensuremath{\to}} %azonos a rightarrow-val
%duplanyil, szuksegesseg
\newcommand{\nn}{\ensuremath{\Rightarrow}}
%\newcommand{\Omage}{\Omega}
%elegségesség, nem def :)
%\newcommand{\nb}{\Leftarrow}
\newcommand{\di}{\displaystyle}
\newcommand{\sarrow}{\downarrow}
\newcommand{\narrow}{\uparrow}
\newcommand{\lt}{<}
\newcommand{\gt}{>}
\newcommand{\Int}{\intop\limits}
\newcommand{\ures}{\varnothing}
\newcommand{\ekv}{\iff}
\newcommand{\ekviv}{\ekv}
\renewcommand{\epsilon}{\varepsilon}
\newcommand{\eps}{\varepsilon}
%
% ------------  NEW PART DEFS -----------------
%
\newcounter{Szaml}


\theoremstyle{magyar}
\theoremheaderfont{\itshape\bfseries}
\newtheorem{de}{definíció}[section]
\newtheorem{te}{tétel}[section]
\newtheorem{bi}{bizonyítás}[section]
\newtheorem{ko}{következmény}[section]
\newtheorem{me}{megjegyzés}[section]
\newtheorem{al}{állítás}[section]


\newenvironment{korlista}{\begin{enumerate}[\quad1$^\circ$]}{\end{enumerate}}

\newenvironment{biz}{\begin{trivlist}\item\relax\mbox{\textbf{Bizonyítás.\enskip}}\ignorespaces}{\ob\end{trivlist}}
\newenvironment{Biz}{\begin{trivlist}\item\relax\mbox{\textbf{Bizonyítás.\enskip}}\ignorespaces\begin{korlista}}{\ob\end{korlista}\end{trivlist}}
\newenvironment{kov}{\begin{trivlist}\item\relax\mbox{\textbf{Következmény.\enskip}}\ignorespaces}{\end{trivlist}}
\newenvironment{megj}{\begin{trivlist}\item\relax\mbox{\textbf{Megjegyzés.\enskip}}\ignorespaces}{\end{trivlist}}
\newenvironment{Megj}{\begin{megj}\begin{korlista}}{\end{korlista}\end{megj}}
\newenvironment{pl}{\begin{trivlist}\item\relax\mbox{\textbf{Példa.\enskip}}\ignorespaces}{\end{trivlist}}
\newenvironment{Pl}{\begin{pl}\begin{korlista}}{\end{korlista}\end{pl}}
\DeclareMathOperator{\D}{D}
\newenvironment{bizlist}{\setcounter{Szaml}{1}
    \begin{list}{\alph{Szaml})\hfill}
    {\usecounter{Szaml}\setlength{\itemsep}{0pt}
    \setlength{\itemindent}{-\labelsep}
    \setlength{\listparindent}{0pt}}}{\end{list}}




%
% - - - -- - - S E T T I N G S ----------------
%
%\setlength{\parindent}{0pt}
%\setlength{\parskip}{\baselineskip}
\addtolength{\voffset}{-1cm}
\addtolength{\textheight}{2cm}
%\addtolength{\marginparwidth}{-1cm}
\addtolength{\hoffset}{-1cm}
\addtolength{\textwidth}{2cm}
\setlength{\headheight}{23pt}
%
\pagestyle{fancy}

  \renewcommand{\sectionmark}[1]{\markboth{\Roman{section}. tétel\\#1}{}}

\newcommand{\mktoc}{
  \pagenumbering{roman}
  \setcounter{page}{1}
  \lhead{\textbf{\thepage}}
  \cfoot{}
  \tableofcontents
  \newpage
  \lhead{\textbf{\thepage}}%/\pageref{LastPage}}
  \pagenumbering{arabic}
  \setcounter{page}{1}
}

