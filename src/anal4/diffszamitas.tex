\section{Differenciálszámítás}

\begin{megj} Csak $\RnRm$ típusú fv-ek; $n,m\in\N$\end{megj}

\subsection{$\RnRm$ típusú leképezések}
\begin{de}
  $L: \RnRm$ lineáris, ha
{\listazjromai
\begin{enumerate}
  \item $L(x + y) = L(x) + L(y)\qquad \forall x,y\in \R^n$ (additivitás)
  \item $L(\lambda x) = \lambda L(x)\qquad \forall x\in\R^n,\ \lambda\in\R$ (homogenitás)  
\end{enumerate}
}
\end{de}

\begin{Megj}
\item Ha $L$ lineáris, akkor $\di L\left(\sum_{i=1}^s \lambda_i y_i\right) = \sum_{i=1}^s\lambda_i L(y_i)$
\item $L\in\R\to\R$ lineáris $\ekviv \exists!c\in \R\ L(x)\quad  x\in\R$\\
  $L$ és $c$ azonosítható egymással.
\item  Jelölés: $\Linearis :=  \{\,L\in R^n\to\R^m\,|\,L\text{ lineáris}\,\}$
\item  Szokásos műveletek: $+$, $\lambda\cdot$ pontonként.
\end{Megj}

\begin{te}
  \Linearis a szokásos műveletekkel LT az $\R$ felett
\end{te}

\subsubsection[Lineáris leképezés mátrix reprezentációja]{Lineáris leképezés mátrix reprezentációja (adott bázisban)}
$\R^n$-ben $e_1,\,e_2,\dotsc,\,e_n$ egy bázis (pl a kanonikus bázis: $e_i=(0,\dotsc,0, \overset{i}{\breve{1}},
0,\dotsc,0)$)\\
$\R^m$-ben $f_1,\,f_2,\dotsc,\,f_n$ egy bázis (pl a kanonikus bázis)\\
$L\in\Linearis,\ x\in \R^n,\di x = \sum_{j=1}^n x_j e_j\quad x_j$ az $x$ vektor $e$ bázisra vonatkozó
koordinátája
\begin{gather*}
  \di  L(x) = L\left(\sum_{j=1}^n x_j e_j\right) = \sum_{j=1}^n x_j L(e_j)\\
  \R^m \owns L(e_j) = \sum_{i=1}^m a_{ij} f_i\qquad\quad\text{$a_{ij}$ az $L(e_j)$ $i$. koordinátája az
  $(f_1,\dotsc,f_n)$ bázisban}\\
  A = \begin{pmatrix}
    a_{11} & a_{12} & \dots & a_{1n}\\
    a_{21} & a_{22} & \dots & a_{2n}\\
    \vdots & \vdots & \ddots & \vdots\\
    a_{m1} & a_{m2} & \dots & a_{mn}
  \end{pmatrix}\in \R^{m\times n}\\
  L(x) = \sum_{j=1}^n x_j L(e_j) = \sum_{j=1}^n x_j \left(\sum_{i=1}^m a_{ij} f_i\right) = \sum_{i=1}^n
  \left(\sum_{j=1}^n a_{ij} x_j\right) f_i\\
  \text{A } \sum_{j=1}^n a_{ij} x_j \text{ mátrixszorzással felírva:}\\
  \begin{pmatrix}
    a_{11} & \dots & a_{1n}\\
    \vdots & \ddots & \vdots\\
    \mathbf{a_{i1}} &  \dots & \mathbf{a_{in}}\\ 
    \vdots & \ddots & \vdots\\
    a_{m1}  & \dots & a_{mn}
  \end{pmatrix}
  \begin{pmatrix}x_1\\x_2\\\vdots\\x_n\end{pmatrix} = (Ax)_i
\end{gather*}

Tehát egyértelműen megfeleltethető e kettő egymásnak:\\
$L\in \Linearis\leftrightarrow A\in \R^{m\times n}$

\begin{te}
  Az $\Linearis$ LT izomorf  az $\R^{m\times n}$ LT-rel, azaz: 
  \begin{gather*}
    \exists \varphi: \Linearis\n\R^{m\times n} \text{ bijekció, melyre}\\
    \varphi(\lambda_1 L_1+\lambda_2 L_2) = \lambda_1\varphi(L_1) + \lambda_2 \varphi(L_2)
    \qquad \forall L_1,\,L_2\in\Linearis;\ \forall \lambda_1,\,\lambda_2\in\R
  \end{gather*}
\end{te}

\begin{megj}
  Az $\RnRm$ leképezés azonosítható egy $A\ m\times n$-es mátrix-szal (egy adott bázisban).
\end{megj}

\subsubsection{Norma értelmezése az $\Linearis$ LT-n}

\begin{te}
  \begin{enumerate}
    \item Ha $L\in\Linearis$, akkor 
      \[\di \opnorma {L} := \sup_{h\in\R^n}\{\,\underbrace{\norman{L(h)}{1}}_{\text{$\R^m$-beli}} : h\in\R^n,
      \underbrace{\norman{h}{2}}_{\text{$\R^n$-beli}}<1\} < \infty\]
      ún. \emph{operátornorma} norma az \Linearis\ téren.
      \item Ha $L\in\Linearis$, akkor
	\[ \norman{L(h)}1 \leqq \opnorma{L}\cdot\norman{h}2\qquad\forall h\in\R^n 
	\]
  \end{enumerate}
\end{te}

\begin{biz}?\end{biz}

\begin{megj}
  $f\in\R\n\R,\  f\in\D\{a\},\ a\in\intD_f$
  \begin{gather*}\di
    f\in\D\{a\} \ \ekviv\  \exists \lim_{h\n0} \dfrac {f(a+h)-f(a)}{h} < \infty \overset{\text{lineáris}}{\underset
      {\text{közelíés}}{\ekviv}}\\\ekviv \exists A\in \R \text{ és } \exists \epsilon \in \R\n\R,\, \lim_0\epsilon=0
    \colon f(a+h) -f(a) = Ah + \epsilon(h)\cdot h \ \ekviv\\
    \lim_{h\n0}\dfrac{|f(a+h)-f(a)-L(h)|}{|h|} = 0
  \end{gather*}
\end{megj}



\subsection{Totális deriválhatóság}

\begin{de}[Totális deriválhatóság]
  $f\in\RnRm,\ a\in\intD_f$. Az $f$ fv totálisan deriválható az $a\in\intD_f$ pontban, ha
  \[\exists L\in\Linearis\colon\quad \lim_{h\n\nullelem}\dfrac{\norman{f(a+h)-f(a)-L(h)}1}{\norman{h}2} = 0,\]
  ahol \Norman1\ egy $R^m$-beli tetszőleges, \Norman2 egy $\R^n$-beli tetszőleges norma.\\
  Az $f$ fv $a$-beli deriváltja: $f'(a) := L(a)$\\
  Jel: $f\in\D\{a\}$
\end{de}

\begin{te}
  Ha $f\in\derivp{a},\ f\in\RnRm$, akkor $f'(a)$ egyértelmű.
\end{te}

\begin{biz}
  Tegyük fel, hogy $\exists L,R\in\Linearis$, amire a definíció teljesül, továbbá 
  \begin{gather*}
     L-R=:S\in\Linearis\\
     \dfrac{\norman{S(h)}1}{\norman{h}2} = \dfrac{\norman{L(h)-R(h)}1}{\norman{h}2} \leqq\\
     \leqq\dfrac{\norman{f(a+h)-f(a)-L(h)}1}{\norman{h}2} + \dfrac{\norman{f(a+h)-f(a)-R(h)}1}{\norman{h}2} \xrightarrow
     [h\n\nullelem]{}0\\
     \nn \lim_{h\n\nullelem} \dfrac{\norman{S(h)}1}{\norman{h}2} = 0.\\\text{Legyen spec: }h:=\lambda e,\
     e\in\R^n,\ \norman{e}2 = 1,\ \lambda \n 0\, \nn\, h\n\nullelem. \\
     \dfrac{\norman{S(h)}1}{\norman{h}2} = \dfrac{\norman{S(\lambda e)}1}{\norman{\lambda e }2} =
     \dfrac{|\lambda|\,\norman{S(e)}1}{|\lambda|\,\norman{e}2}\, \nn\, S(e) = \nullelem \,\nn\, S \equiv 0 \,\nn\, L
     \equiv R
  \end{gather*}
\end{biz}

\begin{te}
  A deriválhatóság ténye és a derivált független attól, hogy $\R^n$-ben és $\R^m$-ben melyik normát választjuk.
\end{te}
\begin{biz}
  $\R^n$-ben a normák ekvivalensek.
\end{biz}

\begin{te}[Ekvivalens átfogalmazások]\ 
  \begin{enumerate}
    \item $\di f\in\der{a} \ekviv  \exists A\in\Rmn\colon  \lim_{h\n\nullelem} \dfrac{\norman{f(a+h) -f(a) -Ah}1}{
    \norman{h}2} = 0$\\
      Az $A$ az ún. \emph{deriváltmátrix}.
    \item $f\in\der{a}  \ekviv \left\{\begin{array}{l}\exists A\in\Rmn \text{  és } \di\exists \epsilon \in\RnRm\
    \lim_0\epsilon=\nullelem :\\f(a+h) -f(a)= Ah + \epsilon(h)\,\norman{h}2\qquad a,a+h\in \D_f\end{array}\right.$\\
      (lineáris fv-nyel való jó közelítés)
  \end{enumerate}
\end{te}
\begin{biz}Trivi\end{biz}
  

\textbf{Spec esetek:}
\begin{enumerate}
  \item $f\in\RnRm;\ f'(a)$ egy $m\times n$-es mátrix ($\in\Rmn)$
  \item $f\in\RnR;\ f'(a)$ egy sorvektor  ($\in R^{1\times n})$
  \item $f\in\RRm;\ f'(a)$ egy oszlopvektor ($\in R^{m\times1})$   
\end{enumerate}


\begin{Pl}
\item $f(x):=c\quad(x\in\R^n);\ c\in\R^m$ rögzített $\nn \forall a\in\R^n:\ f\in\der{a};\ f'(a)=0\in\Rmn$
\item $f=L\in\Linearis,\ \forall a\in\R^n\colon L\in\der a$ és $L'=L$
\end{Pl}

\begin{te}[Folytonosság-deriválhatóság]
  $f\in\RnRm$, $a\in\intD_f$
  \begin{enumerate}
    \item Ha $f\in\der a\nn f\in\folyt a$
    \item Visszafelé \emph{nem} igaz
  \end{enumerate}
\end{te}
\begin{biz}
  \[\nn: f(a+h) -f(a) = Ah + \epsilon(h)\,\norman{h}2\xrightarrow[h\n\nullelem]{}0 \nn f\in\folyt{a}\]
  \[\not\Leftarrow: n=m=1 \text{ és pl } f := \mathrm{abs}\]
\end{biz}

\begin{te}
  $f\in\RnRm, f=\begin{bmatrix}f_1\\\vdots\\f_m\end{bmatrix};\ f_i\in\RnR\ (i=1,\dotsc,m)$ az $f$
  koordinátafüggvényei\\
Ekkor
\[f\in\der{a} \ekviv \forall i = 1,2,\dotsc,m\text{ és } f_i\in\der{a} \text{ és } f'(a)= \begin{bmatrix}f_1'(a) \\
  f_2'(a) \\
  \vdots\\ f_m'(a)\end{bmatrix} = \begin{bmatrix}\hspace{0.8em}\dots\hspace{0.8em}\\\dots\\\vdots\\\dots
\end{bmatrix}\in\Rmn\]
\end{te}
\begin{biz}Trivi
\end{biz}

\begin{megj} $f\in\RnRm\ \nn\ \RnR$ elég vizsgálni.
\end{megj}
\subsubsection{Műveletek}
\begin{te}$f,g\in\RnRm,\ a\in(\intD_f)\cap(\intD_g),\ f,g\in\der a$\\
  $\begin{array}{@{\quad}rcl}
    1^\circ &  f+g\in\der a & (f+g)'(a) = f'(a) + g'(a)\\
    2^\circ &  \lambda g\in\der a\quad \forall \lambda\in\R & (\lambda f)'(a) = \lambda f'(a)
  \end{array}$
\end{te}
\begin{biz}
  trivi
\end{biz}

\begin{te}[Kompozíció, láncszabály]
  $g\in\RnRm,\ a\in\intD_g, g\in\der a$;\\ $f\in\R^m\n\R^r,\ R_g\subset D_f,\ f\in \der{g(a)}$.\\
  Ekkor $f\circ g\in\R^n\n\R^r$ deriválható az $a$-ban és $(f\circ g)'(a) = f'(g(a) \circ f'(a)$
\end{te}
\begin{biz} nem kell (ld. $\R\n\R$ eset)
\end{biz}

\begin{megj}\ \\
  \begin{tabular}{r@{$\,\in\,$}l@{$\ \nn\ $}l}
    $f\circ g$ & $\R^n\n\R^r$ &  $C:=(f\circ g)'(a)\in\R^{r\times n}$\\
    $g$ & $\R^n\n\R^m$ &  $A:=g'(a)\in\R^{m\times n}$\\
    $f$ & $\R^m\n\R^r$ &  $B:=f'(a)\in\R^{r\times m}$\\
  \end{tabular}$ C = BA$\\
Lineáris leképezések kompozíciója $\equiv$ mátrixreprezentációk szorzata
\end{megj}


\subsection{Parciális deriválhatóság}

\begin{de}[Parciális derivált]
  $f\in\RnRm,\ a\in\intD_f,\ e_1,e_2,\dotsc,e_n\in\R^n$ kanonikus bázis, azaz $e_i = (0,\dotsc,0,
  \overset{i}{\breve{1}}, 0,\dotsc,)0 $\\
  \emph{Az $f$-nek $\exists$ az $i$. változó szerinti parciális deriváltja az $a\in\intD_f$ pontban}, ha az\\
  $F: K(0)\owns t \mapsto f(a+t e_i)\quad (F: \R\n\R^m)$\\
  fv deriválható a $0$ pontban.\\
  Az $F'(0)$ oszlopvektor az $f$ $i$. változó szerinti parciális deriváltja az $a$-ban.\\
  Jel: $\partial_i f(a) := F'(0);\quad \dfrac {\partial f}{\partial {x_i}}(a)$ 
\end{de}

\begin{Pl}
  \item $f(x,\,y) = x^3 y^2\quad (x,y)\in\R^2$\\
    $\di\partial_1 f(x_0,y_0) = (x\n f(x,y_0)'_{x=x_0} = 3{x_0}^2{y_0}^2$\\
    $\di\partial_2 f(x_0,y_0) = (y\n f(x_0,y)'_{y=y_0} = 2{x_0}^3y_0$
  \item $f(x,y,z) = x^3 y^2 z$\\
    $\partial_2 f(x_0,\,y_0,\,z_0)= 2{x_0}^3y_0z_0$
\end{Pl}


\begin{te}
  $f\in\RnRm,\ a\in\intD_f$\\
  Ha $f\in\der{a}$, akkor $\forall i=1,2,\dotsc,n\colon  \partial_if(a)$ létezik
\end{te}

\begin{biz}
  \begin{gather*}
  F(t) := f(a+te_i),\qquad F:=f\circ g,\qquad g(t)=a+te_i.\\
  g\in\der 0,\ g'(a)=e_i\nn F=f\circ g\in\der 0\nn \exists.\\
  F'(0) = \partial_i f(a) = f'(a)g'(a) = f'(a)\cdot(e_1) = \underbrace{f'(a)}_{\text{mátrix}} \cdot
  \underbrace{e_1}_{\text{vektor}}
  \end{gather*}
\end{biz}

\begin{te}[A deriváltmátrix előállítása]
  $f\in\RnRm,\ a\in\intD_f,\\ f=\begin{bmatrix}f_1\\\vdots\\f_n\end{bmatrix};\ f_i\in\RnR\ (i=1,\dotsc,m)$\\
  \vspace{.1em}
  Ha $f\in\der a\nn\\$
  \[\Rmn\owns f'(a) = \begin{bmatrix}
    \partial_1 f_1(a) & \partial_2 f_1(a) & \dots & \partial_n f_1(a)\\
    \partial_1 f_2(a) & \partial_2 f_2(a) & \dots & \partial_n f_2(a)\\
    \vdots & \vdots & \ddots & \vdots \\
    \partial_1 f_1(a) & \partial_2 f_1(a) & \dots & \partial_n f_1(a)
    \end{bmatrix}\]
  deriváltmátrix vagy \emph{Jacobi-mátrix}
\end{te}

\begin{biz}
    $f\in\der a \nn \exists f'(a) = A = \left[ a_{ij}\right] \in \Rmn$
    \[\di\lim_{h\n\nullelem} \dfrac{\norman{f(a+h)-f(a) -Ah}1}{\norman h2}=0\]
    $\nn \forall i = 1,2,\dotsc,m \text{ esetén}$
    \[\di\lim_{h\n\nullelem} \dfrac{\left|f_i(a+h)- f_i(a) - \sum\limits_{k=1}^n a_{ik} h_k\right|}{\norman h2}=0\]
    Legyen spec: $h := te_j\quad (t\in\R)\quad (e_j = (0,\dotsc,0,\overset{j}{\breve{1}},0,\dotsc,0))\quad h\n\nullelem
    \ekviv t\n0$

    \[\di\nn \lim_{t\n0}\dfrac{\left|f_i(a+te_j)-f_i(a) - a_{ij}t\right|}{|t|\,\norman{e_j}2} = 0\]
    $\overset{\text{parc.der.}}{\nn} a_{ij} = \partial_j f_i(a)$
\end{biz}

\begin{Megj}
\item \underline{Spec. esetek}\\
\begin{enumerate}
  \item $f\in\RnR,\ f\in\der a\\ f'(a) = \begin{bmatrix}\partial_1 f(a) & \partial_2 f(a) & \dots & \partial_n
    f(a)\end{bmatrix}$
    az $f$ gradiens vektora, $\grad f(a)$
  \item $f\in\RRm\quad f=\begin{pmatrix}f_1\\\vdots\\f_m\end{pmatrix}\quad f_i\in\R\n\R$\\
    $f\in\der a\nn f'(a) = \begin{bmatrix}f_1'(a)\\\vdots\\f_m'(a)\end{bmatrix}$
\end{enumerate}
\item Totális és parciális derivált kapcsolata\\
  Tudjuk: $f\in\der a \nn \forall $ parciális derivált létezik.\\
  Várható: visszafele NEM igaz.\\
  Pl: $f(x,y) := \sqrt{|xy|}\quad(x,y)\in\R^2$.\\
  Ekkor $\partial_1 f(0,0) = 0 = \partial_2 f(0,0)$, de $f\not\in\der{(0,0)}$\\
  Hf, gyak
\item Meglepő: elégséges feltétel adható
\item $f\in\RnRm\qquad f=\begin{bmatrix}f_1\\\vdots\\f_m\end{bmatrix}$\\
  $f\in\der a \ekviv \forall i=1,\dotsc,m\ f_i\in\der a$\\
  azaz: elég az $m=1$ esetet vizsgálni  
\item Jelölés:\\
  $\varphi\in\RnR\\
  f\in\RnRm$
\end{Megj}

\begin{te}[Elégséges feltétel a deriválhatóságra]
  Legyen $\varphi\in\RnR,\ a\in\intD_\varphi,\\\varphi\in K(a)\n\R$.
  Tfh. $\forall i=1,\dotsc,n$-re
  \begin{enumerate}
    \item a $\partial_i\vfi$ parciális deriváltak léteznek $\forall x\in K(a)$-ra.
    \item $\partial_i\vfi(x)\colon K(a)\n\R,\ x\n\partial_i \vfi(x)$ parciális derivált függvények folytonosak az
    $a$-ban: $\partial_i \vfi\in\folyt a$      
  \end{enumerate}
Ekkor  $\vfi\in\der a$ (totálisan deriválható)
\end{te}

\begin{biz} $n=2$-re ($n>2$-re hasonlóan):
  \begin {gather*}
    \di\vfi\colon \R^2\n\R,\ a=(a_1,a_2),\ h=(h_1,h_2)\\
    \vfi\in\der a \overset{\text{def}}{\ekviv} \begin{array}{l}
      \exists A,B\in\R,\ \exists \epsilon\in\R^2\n\R,\ \di\lim_\nullelem\epsilon=0\text{, melyre:}\\
      \lim\limits_{h\n\nullelem}\dfrac{\left|\vfi(a+h)-\vfi(a)-(A,\,B)\begin{pmatrix}h_1\\h_2\end{pmatrix}\right|}{\norma h}=0
    \end{array}\\
    \vfi(a+h) - \vfi(a) = \vfi(a_1+h_1,\,a_2+h_2) - \vfi(a_1,\,a_2) =\\
    = \vfi(a_1+h_1,\,a_2+h_2) - \vfi(a_1+h_1,\,a_2) +\vfi(a_1+h_1,\,a_2) - \vfi(a_1,\,a_2) = \star
  \end{gather*}
  A valós-valós Lagrange-középértéktételt felhasználva legyen: $\nu_1\in(0,1)$; $a_2$ rögzített:
  \[\vfi(a_1+h_1,\,a_2) - \vfi(a_1,\,a_2) = \partial_1\vfi(a_1+\nu_1h_1,\,a_2)\cdot h_1\]
  Hasonlóan legyen $\nu_2\in(0,1)$; $a_1+h_1$ rögzített:
  \[\vfi(a_1+h_1,\, a_2+h_2) - \vfi (a_1+h_1, a_2) = \partial_2\vfi(a_1+h_1,a_2+\nu h_2)\cdot h_2\]
  Behelyettesítve:
  \[ \star=\partial_1\vfi(a_2+\nu_1h_1,a_2)\cdot h_1 + \partial_2\vfi(a_1+h_1,a_2+\nu_2 h_2)\cdot h_2 = \sharp\]
  De!
  \begin{gather*}\begin{array}{l}\partial_1\vfi \in\folyt{(a_1,a_2)}\\
      \partial_2\vfi \in\folyt{(a_1,a_2)}\end{array} \nn 
    \partial_1(a_1+\nu_1 h_1, a_2) = \partial_1\vfi(a_1,a_2)+\epsilon_1(h)\\
    \text{ahol a folytonosság miatt} \lim_{h\n\nullelem}\epsilon_1(h)=0 \text{, illetve:}\\
    \partial_2(a_1+h_1, a_2+\nu_2 h_2) = \partial_2\vfi(a_1,a_2)+\epsilon_2(h),\qquad\lim_{h\n\nullelem}\epsilon_2(h)=0    
  \end{gather*}
  Így:
  \begin{gather*}\sharp = \vfi(a+h)-\vfi(a) = \big[\underbrace{\partial_1\vfi(a_1,\,a_2)}_{A}+\epsilon_1(h)\big]h_1 + 
    \big[\underbrace{\partial_2\vfi(a_1,\,a_2)}_{B}+\epsilon_2(h)\big]h_2.\\
    \dfrac{\left|\vfi(a+h) - \vfi(a) - \begin{pmatrix}A & B\end{pmatrix} \begin{pmatrix}h_1\\h_2\end{pmatrix}\right|}
    {\norma h} = \dfrac{\left|\epsilon_1(h)h_1 + \epsilon_2(h)h_2\right|}{\norma h} \leq |\epsilon_1(h) + \epsilon_2(h)| 
    \xrightarrow[h\n\nullelem]{} 0 \\\nn \vfi\in\der{a}
  \end{gather*}
\end{biz}

\subsection{Iránymenti deriválhatóság}

\begin{de}[$e$ irány szerinti iránymenti derivált]
  $f\in\RnRm,\\a\in\intD_f,\ e\in\R^n $ egyésgvektor $(\norma{e}_2 = 1)$.\\
  Az $f$ fv-nek az $a\in\intD_f$ pontban az $e$  irány szerinti iránymenti deriváltja létezik, ha az
  \[F: K(a)\owns t \mapsto f(a+te) \in \R^n\]
  fv a $0$ pontban deriválható. Az $F'(0)\in\R^n$ az $e$ irányban vett iránymenti deriváltja $a$-ban.\\
  Jel: $\partial_ef(a) := F'(0)$
\end{de}

\begin{te}[Az iránymenti derivált kiszámolása]
  Ha $f\in\RnRm,\ a\in\intD_f$,\\
  Ekkor $\forall e\in\R^n$ egységvektor $(\norma{e}_2 = 1)$ esetén $\exists\partial_ef(a)$ és
\[\partial_ef(a) = f'(a)\cdot e\]
\end{te}
\begin{megj} $\partial_ef(a)\in\R^m;\ f'(a) \in\RnRm;\ e\in\R^n$ - mátrixszorzás
\end{megj}
\underline{Spec. esetek:}
\begin{enumerate}
  \item Iránymenti derivált: parciális derivált általánosítása. Ha $e=e_i$, akkor $\partial_ef(a) = \partial_if(a)$
  \item $m=1\colon \ \vfi\colon\RnR^1; e\in\R^n; \norma{e}_2=\di\sqrt{\sum_{k=1}^n{e_k}^n} = 1$
    \[\di \partial_e\vfi(a) = \skalar{\grad f(a)}{e} = \sum_{k=1}^n \partial_kf(a)e_k\]
\end{enumerate}
\begin{Megj}
\item Lényeges: $\norma{e}_2=1$
\item Totális derivált és iránymenti derivált kapcsolata:\\
  Tudjuk: $f\in\der{a}\nn\ \forall$ irányban deriválható\\
  Visszafele NEM igaz!
\end{Megj}

\subsection{Középértéktétel}
\begin{te}[Lagrange-középértéktétel]
  $n\geq1,~U\subset\R^n$ nyílt és $\forall a\in U,\,a+h\in U$.\\
  Szakasz: $[a,\,a+h] := \{\,a+th:t\in(0,1)\,\}\subset U$
  {\listazjbetu
    \begin{enumerate}
    \item Ha $\vfi\colon U\n\R,\ \vfi\in\D(U)$: $U$ minden pontjában deriválható\\
      akkor $\exists \nu\in(0,1):\vfi(a+h)-\vfi(a)=\vfi'(a+\nu h)\cdot h=\skalar{\grad\vfi(a+\nu h)}{h}$
    \item Ha $f\colon U\n\R^m,\ m\geq2,\ f\in\der{x}\ (x\in U)$, akkor
      \[ \norma{f(a+h)-f(a)}_\infty\leq \di\sup_{0\leq\nu\leq1} f'(a+\nu h)(h) \leq
      \sup_{0\leq\nu\leq1}\opnorma{f'(a+\nu h)}\cdot \norma{h}_\infty\]
    \end{enumerate}  
  }
\end{te}
\begin{biz} nem kell
\end{biz}

\subsection{Többször deriválható függvények}
\begin{de}
  $\vfi\in\RnR,\ a\in\intD_f$. A $\vfi$ kétszer deriválható az $a\in\intD_f$, ha
  {\listazjromai
    \begin{enumerate}
    \item $\exists K(a)\colon \forall x\in K(a)$-ban $\vfi$ deriválható
    \item $\forall i=1,\dotsc,n\ \partial_i\vfi$ parciális függvények deriválhatóak $a$-ban: $\partial_i\in\der{a}$
    \end{enumerate}
  }
  Jel: $\vfi \in\dern2a$
\end{de}
\begin{megj}
  $(i) \ekviv \exists\vfi'=(\partial_1\vfi,\dotsc,\partial_n\vfi)\in\R^n\n\R^n$ függvény\\
  $(ii)\ \partial_i\vfi\in\der{a}\ (i=1,2,\dotsc,n) \ekviv \vfi'\in\der{a}$\\\\
  $\vfi'' = (\vfi')'$, de $\vfi'\in\R^n\n\R^n,\ \vfi''\in\R^n\n\R^{n\times n}$.\\
  \[\vfi(a)'' = \begin{bmatrix}
  \partial_1\partial_1\vfi &\partial_2\partial_1\vfi & \cdots & \partial_n\partial_1\vfi \\
  \vdots & \vdots & \ddots & \vdots \\
  \partial_1\partial_n\vfi &\partial_2\partial_n\vfi & \cdots & \partial_n\partial_n\vfi 
  \end{bmatrix}\in\R^{n\times n}\]
  az ún. \emph{Hesse-féle mátrix}  
\end{megj}

\begin{de}
  $\vfi \in\RnR,\ a\in\intD_\vfi$. \\A $\vfi$ $s$-szer $(s\geq2)$ deriválható az $a$-ban $(\vfi \in\dern n a)$, ha
  {\listazjromai
    \begin{enumerate}
    \item $\exists K(a)\subset D_\vfi$, hogy $\vfi$ (s-1)-szer deriválható a $K(a)\ \forall$ pontjában.
    \item Az összes $\partial_{i_1}\partial_{i_2}\ldots\partial_{i_{s-1}}\vfi\quad 1\leq i_1, i_2,\dotsc,i_{s-1} \leq n$
      \quad (s-1)-edrendű parciális derivált függvény deriválható az $a$-ban.
    \end{enumerate}
  }
\end{de}
\begin{te}[Young-tétel]$\vfi\in\RnR,\ a\in\intD_\vfi$
  \[\vfi \in\dern2a\nn\forall i,j=1..n\quad \partial_j(\partial_i\vfi) = \partial_i(\partial_j\vfi) \]
\end{te}
\begin{te}\label{te:youngkov}(Következmény!!) $\vfi\in\RnR,\ a\in\intD_\vfi, \vfi\in\dern s a\ (s\geq2)$
  \[ (\partial_{i_1}\partial_{i_2}\ldots\partial_{i_s}\vfi)(a) = 
  (\partial_{\sigma_1}\partial_{\sigma_2}\ldots\partial_{\sigma_s}\vfi)(a)\]
  ahol $1\leq i_1, i_2,\dotsc,i_s \leq n$ és a $\sigma_1, \sigma_2,\dotsc,\sigma_s$ az $i_1, i_2,\dotsc,i_s$ egy
  permutációja  
\end{te}

\subsubsection{Taylor-formula}
\begin{te}(Emlékeztető)
  $f\in\R\n\R;\ m\in\N,\ f\in\D^{m+1}\left(K\left(a\right)\right);\ a,a+h\in\D_f;\\\exists \nu\in(0,1):$
  \[\di f(a+h) = f(a) + \sum_{k=1}^m\dfrac{f^{(h)}(a)}{k!}h^k + \dfrac{f^{(m+1)}(a+\nu h)}{(m+1)!}h^{m+1}\]  
\end{te}

\begin{de}[Multiindex] $n\geq 1$ rögzített, $i$ multiindex, $i:=(i_1,\dotsc,i_n),\ i_k\geq 0$ egészek.\\
  $|i| := i_1 + i_2 + \ldots + i_n$ a multiindex rendje\\
  $i!~ := i_1! \cdot i_2! \dotsm i_n!$\\
  $x=(x_1,\dotsc,x_n)\in\R^n\colon\quad x^i := x_1^{i_1}\cdot x_2^{i_2}\dotsm x_n^{i_n}$\\
  $\partial^i\vfi := \partial_1^{i_1}\partial_2^{i_2}\dotsb\partial_n^{i_n}\vfi$ vagyis az első változó szerint
  $i_1$-szer, stb.\\
  $\partial^0\vfi :=  \vfi$
\end{de}

\begin{de}[Homogén $n$ változós $m$-edfokú polinom] \ \\$n=1,2,\dotsc$; $m=0,1,2,\dotsc$; $i$ multiindex: $|i|=m$
  \[\R^n\owns x\mapsto \sum_{|i|=m}a_ix^i\quad \text{ahol }a_i\in\R\]  
\end{de}

\begin{spec}{\listazjromai\begin{enumerate}
  \item $n=1;\ m=0,1,2,\dotsc\colon\quad \R\owns x\mapsto ax^m$
  \item $n=2;\ m=1\colon\quad i\colon (0,1) \text { v. }(1,0)\colon\quad \R^2\owns(x_1,x_2)\mapsto a_1x_1+a_2x_2$
  \item $n=2;\ m=2\colon\quad i\colon (2,0),\ (1,1),\ (0,2)\colon\quad \R^2\owns(x_1,x_2)\mapsto a{x_1}^2+bx_1x_2 +
    c{x_2}^2$    
  \end{enumerate} }
\end{spec}

\begin{te}[Taylor-formula a Lagrange-maradéktaggal]Tegyük fel, hogy
  {\listazjbetu \begin{enumerate}
    \item $\vfi\colon U\n\R,\ U\subset\R^n$ nyílt halmaz
    \item $a\in U,\ h\in\R^n\colon\ [a,a+h] := \{a+th:t\in(0,1)\}\subset U$
    \item $\vfi\in\D^{m+1}([a,a+h])\quad(m=0,1,2,\dotsc\text{rögzített})$
  \end{enumerate} }
  Ekkor $\exists \nu\in(0,1)$
  \[ \di\vfi(a+h) = \vfi(a) + \underbrace{\sum_{k=1}^m\left(\sum_{|i|=k}\dfrac{\partial^i\vfi(a)}{i!}h^i\right)}
  _{\text{Taylor-polinom}} + \underbrace{\sum_{|i|=m+1}\dfrac{\partial^i\vfi(a+\nu h)}{i!}h^i}
  _{\text{Lagrange-féle maradéktag}}\]  
\end{te}

\begin{biz}Visszavezethető $\R\n\R$-re\\
  \[ F(t) := \vfi(a+th)\qquad(t\in[0,1])\]
  Az $F\in\R\n\R$ függvényre a Taylor-formula alkalmazható a $[0,1]$ intervallumon (a feltételek teljesülnek).\\
  \[\di\exists \nu\in(0,1)\colon F(1) = F(0) + \sum_{k=1}^m \dfrac{F^{(k)}(0)}{k!} (1-0)^k + 
  \dfrac{F^{(m+1)}(\nu)} {(m+1)!}\]
  
  A tétel állítása a következő lemma felhasználásával adódik.
  \begin{lemma} A fenti $F$ függvény esetén ($\vfi$ $s$-szer deriválható $[a,a+h]$-n)
    \[\di\dfrac{F^{(k)}(t)}{k!}= \sum_{|i|=k}\dfrac{\partial^i\vfi(a+th)}{i!}h^i\qquad k=0,1,2,\dotsc,s\]
  \end{lemma}
  \textbf{A lemma bizonyítása} $k$-ra vonatkozó teljes indukcióval.\\
  $k=1$ esetén $F$ definiciója és az összetett függvény deriválási szabálya alapján
  \[ F'(t) = \skalar{\grad \vfi(a+th)}{h} = \sum_{|i|=1} \partial^i \vfi(a+th)\cdot h^i\qquad(t\in[0,1])\]
  Tegyük fel, hogy $k\in\{1,\dotsc,s-1\}$ esetén igaz az állítás. Így $k+1$-re:
  \begin{align*}
    \dfrac1{(k+1)!}F^{(k+1)}(t) &= \dfrac1{(k+1)!}(F^{(k)})'(t)\\
    &=\dfrac1{k+1}\sum_{|i|=k} \dfrac1{i!}
    (\partial_1\partial^i\vfi(a+th)h^ih_1+\ldots + \partial_n\partial^i\vfi(a+th)h^ih_n) ={}\\
    &\stackrel{\text{\ref{te:youngkov} alapján}}{=}
    \sum_{|i|=k+1}\dfrac{\partial^i\vfi(a+th)}{i!}h^i\qquad(t\in[0,1])
  \end{align*}
\end{biz}

\subsection{Inverz függvények ($\RnRn$)}
\underline{Globális, $\R\n\R$-beli tétel:}\\
\[f\colon (a,b)\n\R,\ f\in\D,\ f'>0\ (a,b)$-n, ekkor $\exists f^{-1}$ inverz, ui $f\uparrow\]
Ez nem általánosítható, de a lokális változata igen.:\\\\
\underline{Lokális, $\R\n\R$-beli tétel:}\\
$f: I\n \R, I\subset\R$ intervallum\\
$f$ folytonosan deriválható\\
$a\in I$-ben $f'(a)\neq 0$\\
EKKOR $\exists U := K(a)$ és $\exists V:= K(f(a))$: $f_{|U}\colon U\n V$ bijekció $\nn \exists$ inverz\\
és az $f^{-1}$ inverz differincálható és $\left(f^{-1}\right)'(x) = \dfrac1{f'(f^{-1}(x))}\quad\forall x\in V$


\begin{te}[Inverz függvény tétel] $\Omega \subset \R^n$ nyílt, $a\in\Omega,\ f:\Omega\n\R^n$. Tfh:
  {\listazjromai \begin{enumerate}
  \item $f$ folytonosan deriválható $\Omega$-n
  \item $\det f'(a) \neq 0$
  \end{enumerate} }
  Ekkor
  { \listazjbetu \begin{enumerate}
  \item $\exists U := K(a)$ és $\exists V:= K(f(a))$ $f_{|U}\colon U\n V$ bijekció (azaz $\exists f^{-1}$ inverz)
  \item $f^{-1}$ deriválható, $(f^{-1})'(x) = \begin{bmatrix} f'(f^{-1}(x))\end{bmatrix}^{-1}\quad (\forall x\in V)$
  \end{enumerate}
  }
\end{te}
\textbf{Alkalmazás.} Nemlineáris egyenletrendszerek megoldhatósága\\
$\left.\!\begin{array}{rcl}
f_1(x_1,\dotsc,x_n) & = & b_1\\
\vdots & & \vdots\\
f_n(x_1,\dotsc,x_n) & = & b_n\end{array}\right\}\qquad \begin{array}{cl} \left.\begin{array}{c}f_i\in\RnR\\b_i\end{array}
\right\} &\text{adott}\\x_1,\dotsc,x_n&\text{ismeretlen}\end{array}$\\
\vphantom{x}\\
$f := \begin{pmatrix}f_1\\\vdots\\f_n\end{pmatrix}\quad b := \begin{pmatrix}b_1\\\vdots\\b_n\end{pmatrix}$.\\
Legyen $a=\begin{pmatrix}a_1\\\vdots\\a_n\end{pmatrix}\in\R^n\colon f(a)=b$.\\
Keresni kell egy ilyet. Ha $f'(a)$
invertálható ($\ekviv \det f'(a)\neq 0$) $\overset{\text{Inverz fv}}{\underset{\text{tétel}}{\nn}}\\
\exists K(b)\ \forall y=(y_1,\dotsc,y_n)\in K(b)\ f(x)=y$ egyenletrendszer $x$-re egyéretlműen megoldható.
\begin{megj}Létezést biztosít.\\
A fixpont-tétel segítségével közelítő megoldás adható\end{megj}


\begin{de}$f\in\R^2\n\R$ adott. Ha $\exists I\subset \R$ intervallum és $\exists \vfi: I\n\R$, hogy\\ $f(x,\vfi(x))=0
  (x\in I)$, akkor a $\vfi$ az $f(x,y) =0$ \emph{implicit egyenlet megoldása}.\\ (a $\vfi$ függvény az $f(x,y)=0$
  implicit alakban van megadva)
\end{de}

\newpage
\begin{te}[Implicit függvény-tétel, spec: 2 változós] $f\in\R^2\n\R^1,\\D_f=\Omega\subset\R^2$ nyílt. Tfh:
  {\listazjromai \begin{enumerate}
  \item $f$ folytonosan differenciálható $\Omega$-n
  \item $\exists (a,b) \in \Omega:\ f(a,b)=0$ és $\partial_2f(a,b)\neq 0$
  \end{enumerate}}
  Ekkor
  {\listazjbetu
    \begin{enumerate}
    \item $\exists K(a) =: U\colon\ \forall x\in K(a)$-hoz $\exists \vfi(x)\in \R$, melyre $f(x,\vfi(x)) = 0\quad(\forall
      x\in U)$
    \item A $\vfi\colon U\n\R$ fv folytonosan deriválható és
      \[\vfi'(x) = - \dfrac{\partial_1f(x,\vfi(x))}{\partial_2f(x,\vfi(x))}\quad \forall x\in U \tag{$*$}\label{eq:*}\]
    \end{enumerate}
  }
\end{te}

\begin{Megj}
\item $($\ref{eq:*}$)$-ról: Ha $\vfi$ deriválható: $f(x,\vfi(x))=0\quad(x\in I) \stackrel{\text{láncszabály}}{\nn}\\
  \partial_1f(x,\vfi(x))\cdot1 + \partial_2f(x,\vfi(x))\cdot\vfi'(x) = 0$ 
\item Általánosítás: $n_1,n_2\in\N;\ \Omega_1\subset\R^{n_1};\ \Omega_2\subset\R^{n_2}$ nyíltak. $a\in\Omega_1,\
  b\in\Omega_2,\\ f\colon \Omega_1\times\Omega_2\n\R^{n_2}$, ahol $\Omega_1 \times \Omega_2$ az első, illetve a második
  változócsoportot jelöli.\\
  Legyen $\partial_1f(a,b) := (\Omega_1\owns x\mapsto f(x,b)\in\R^{n_2})'_{x=a}$ az első változócsoport szerinti
  derivált, illetve\\$\partial_2f(a,b) := (\Omega_2\owns y\mapsto f(a,y)\in\R^{n_2})'_{y=b}$ az második
  változócsoport szerinti derivált
\end{Megj}

\begin{te}[Implicit függvény-tétel, ált] \ldots $f\in\Omega_1\times \Omega_2\n\R^{n_2}$. Tfh:
  \begin{enumerate}[\quad(i)]
  \item $f$ folytonosan differenciálható
  \item $\exists a \in \Omega_1,\ b\in\Omega_2:\ f(a,b)=0$ és $\det(\partial_2f(a,b))\neq 0$
  \end{enumerate}
  \noindent Ekkor
  {\listazjbetu
    \begin{enumerate}
    \item $\exists U_1:= K(a)\subset R^{n_1}$ és $\exists U_2:=K(b)\subset \R^{n_2}$:\\
      $\forall x\in U_1$-hoz $\exists! \vfi(x)\in U_2$, melyre $f(x,\vfi(x)) = 0$
    \item A $\vfi\colon U_1\n U_2$ fv folytonosan deriválható és
      \[\vfi'(x) = - [\partial_2f(x,\vfi(x))]^{-1}\cdot \partial_1f(x,\vfi(x))\quad\forall x\in U_1 \]
    \end{enumerate}
      }
\end{te}

\subsection{Szélsőértékek ($\R^n\n \R^1$)}
\begin{te}[Elsőrendű szükséges feltétel lokális szélsőértékre]
Tfh: \\$\vfi\in U\n\R,\ U\subset \R^n$ nyílt,
\begin{enumerate}
\item $\vfi \in \der a\quad a\in U$ (belső pont!!!)
\item $\vfi$-nek lokális szélsőértéke van $a$-ban
\end{enumerate}
Ekkor  \[\vfi'(a) = (\partial_1\vfi(a),\dotsc,\partial_n\vfi(a))=0\tag{$**$}\label{eq:**}\]

\end{te}
\begin{biz}Trivi, ui: $t\mapsto f(a+te_i)\ (\in\R\n\R)$ parciális függvénynek is lokális szélsőértéke van $t=0$-ban.
\end{biz}

\newpage
\begin{Megj}
\item Szükséges, de nem elégséges: $(n=1\colon f(x) := x^3)$
\item $($\ref{eq:**}$)\ekviv \left.\begin{array}{c}\partial_1\vfi(a)=0\\\partial_n\vfi(a)=0\end{array} \right\}$
  $n$ db egyenlet, $n$ db ismeretlen: $(a_1,\dotsc,a_n)$\\
  Itt lehet csak szélsőérték
\end{Megj}

\begin{de}
  $\vfi: U\n\R,\ U\subset \R^n$ nyílt, $a\in U$. A $\vfi$-nek az $a$-ban lokális minimuma [maximuma] van, ha $\exists
  K(a) (\subset U)\colon \vfi(a) \leqq \vfi(x) \ [\vfi(a) \geqq \vfi(x)]  \quad(x\in K(a))$
\end{de}
\begin{megj}
  Lokális szélsőérték $\ekviv$ lokális minimum vagy lokális maximum
\end{megj}

\begin{de}[Kvadratikus alak]\ 
  Az $A=[a_{ij}]\in\R^{n\times n}$ szimmetrikus mátrix,\\ $h=(h_1,h_2,\dotsc,h_n)\in\R^n$. A  $Q\colon \RnR$
  \[ Q(h) := \skalar{Ah}h = \sum_{i=1}^n a_{ij}h_ih_j\]
  fv-t az \emph{$A$ mátrix által meghatározott kvadratikus formának} nevezzük 
\end{de}

\begin{megj}
  $\di Q(h) = \sum_{|i|=2}a_ih^i\quad i=(i_1,\dotsc,i_n)$ multiindex\\
  Ez egy homogén n-változós másodfokú polinom.
\end{megj}
\begin{de}
  $A=[a_{ij}]\in\R^{n\times n}$ szimmetrikus.\\
  A $Q(h)$ kvadratikus forma (vagy az $A$ mátrix)
  \begin{itemize}[\quad]
  \item \underline{pozitív definit}, ha $Q(h)>0\quad\forall h\in\R^n\setminus\{0\}$
  \item \underline{negatív definit}, ha $Q(h)<0\quad\forall h\in\R^n\setminus\{0\}$
  \item \underline{pozitív szemidefinit}, ha $Q(h)\geq0\quad\forall h\in\R^n$
  \item \underline{negatív szemidefinit}, ha $Q(h)\leq0\quad\forall h\in\R^n$
  \end{itemize}
\end{de}


\begin{te}[Sylvester-kritérium]$Q(h) = \skalar{Ah}h$ kvadratikus alak,\\$A=[a_{ij}]\in\R^{n\times n}$ szimmetrikus
  mátrix
  
  \[A=\begin{bmatrix}a_{11} & a_{12} & \cdots & a_{1n}\\
  a_{21} & a_{22} & \cdots & a_{2n}\\ \vdots & \vdots & \ddots & \vdots \\
  a_{n1} & a_{n2} & \cdots & a_{nn}\end{bmatrix};\qquad \Delta_k = \det\begin{bmatrix}a_{11}&\cdots&a_{1k}\\
  \vdots &\ddots& \vdots\\a_{k_1} & \cdots & a_{kk}\end{bmatrix} \text{sarok-aldeterminánsok}\]
Ekkor
\begin{enumerate}
\item $Q$ pozitív definit $\ekviv\Delta_1>0,\,\Delta_2>0,\dotsc,$ azaz $\sgn\Delta_k=1\quad k=1,2,\dotsc,n$
\item $Q$ negatív definit $\ekviv\Delta_1<0,\,\Delta_2>0,\dotsc,$ azaz $\sgn\Delta_k=(-1)^k \quad k=1,2,\dotsc,n$
\end{enumerate}
\end{te}

\begin{te}
  Ha $Q$ kvadratikus forma $\nn\\\exists m,M\in\R\colon m\norma{h}^2\leq Q(h)\leq M\norma{h}^2\quad(h\in\R^n)$
\end{te}
\begin{biz}
  $Q\colon \RnR$ folytonos függvény, $H:=\{x\in\R^n,\norma x = 1\}$ kompakt $\stackrel{\text{Weierstrass}}{\nn}$\\
  \[\exists M:=\max\{Q(h) : \norma h = 1\},\ \exists m:=\min\{Q(h) : \norma h = 1\}\]
  DE!\\
  \[Q(h) =  Q\left(\norma{h}\cdot\dfrac{h}{\norma{h}}\right) = \norma{h}^2Q\left(\dfrac{h}{\norma{h}}\right)\ \ \nn\ 
  \ Q(h)\leq \norma{h}^2 M,\ Q(h)\geq m\norma{h}^2\]
\end{biz}

\begin{kov}
  $Q(h)$ kvadratikus forma,\\
  $Q$ pozitív definit $\ekviv \exists c_1>0\colon Q(h)\geq c_1\norma{h}^2$\\
  $Q$ negatív definit $\ekviv \exists c_2<0\colon Q(h)\leq c_2\norma{h}^2$\\
  Az előző tételből adódik
\end{kov}

\begin{te}[Másodrendű elégséges feltétel, lokális szélsőértékre]\ \\
  Tfh $\vfi\colon U\subset\R,\ U\n\R^n$ nyílt, $a\in U$ belső pont!!!
  {\listazjromai
    \begin{enumerate}
    \item $\vfi$ kétszer folytonosan deriválható
    \item $\vfi'(a)=0$
    \item $\vfi''(a)$ Hesse-féle mátrix által generált kvadratikus alak pozitív [negatív] definit.
    \end{enumerate}
}
Ekkor $\vfi$-nek $a$-ban lokális minimuma [maximuma] van.
\end{te}

\begin{megj} $\vfi''(a)=\ldots$ + Sylvester
\end{megj}
\begin{biz}$a,a+h\in K(a)$, $f\in\dern2x\ \forall x\in K(a)$\\
  Taylor-formula alapján $\exists \nu\in(0,1)$:
\begin{gather*}
  f(a+h)-f(a)=\sum_{i=1}^n\partial_if(a)\cdot h_i+\dfrac12\cdot\sum_{i,j=1}^n\partial_i\partial_jf(a+\nu h)h_ih_j=
  \\ = \sum_{|i|=1}\dfrac{\partial^i f(a)}{i!} + \sum_{|i|=2}\dfrac{\partial^i f(a+\nu h)}{i!} =\\
  =\dfrac12\sum_{i,j=1}^n\partial_i\partial_jf(a+\nu h)h_ih_j,\text{ ui }f'(a)=0\text{, így }\partial_if(a)=0.\\
  \epsilon_ij(h):=\partial_i\partial_jf(a+\nu h)-\partial_i\partial_jf(a)\quad(i,j=1,\dotsc,n)\\
  \text{Az első feltétel alapján} \lim_0\epsilon_{ij}=0\\
  f(a+h)-f(a)=\dfrac12(Q(h)+R(h))\\
  \intertext{ahol}
  R(h):=\sum_{i,j=1}^n\epsilon_{ij}(h)h_ih_j\\
  a+h\in K(a),\, h\ne 0\ \nn\ \vert R(h)\vert=\norma{h}^2\left\vert\sum_{i,j=1}^n\epsilon_{ij}(h)\dfrac{h_i}{ 
    \norma{h}}\dfrac{h_j}{\norma{h}}\right\vert\leq\norma{h}^2\cdot\sum_{i,j=1}^n|\epsilon_{ij}(h)|\\
  \nn \exists \delta>0\colon \forall h\in\R^n\ a+h\in K(a),\  \norma h<\delta:\\
  |R(h)|\leq \dfrac m2\norma{h}^2\quad m:=\min\{Q(h)\in\R:\norma h=1\}\\
  f(a+h)-f(a)\geq \frac m2\norma{h}^2-\frac m4\norma{h}^2=\frac m4\norma{h}^2
\end{gather*}
vagyis $f$-nek az $a$ pontban lokális minimuma van.
\end{biz}

\begin{te}
  $\vfi\colon U\n\R,\ U\subset \R^n$ nyílt, $a\in U$
{\listazjromai
\begin{enumerate}
\item $\vfi$ kétszer folytonosan deriválható
\item $\vfi'(a)=0$ és a $\vfi''(a)$ álatal generált kvadratikus forma indefinit
\end{enumerate}
}
Ekkor $a$-ban $\vfi$-nek nincs lokális szélsőértéke
\end{te}
\begin{biz}
  Ha indefinit: nem szemidefinit $\nn$ szükséges feltétel alapján nincs szélsőérték
\end{biz}

\subsubsection{Feltételes szélsőérték}
\begin{PlSS}
  Adott: $x+y-2=0$ egyenletű egyenes. Melyik rajta lévő $P$ pont esetén lesz $\overline{OP}$, azaz az origótól való
  távolság minimális? Azaz:\\
  $f(x,y) := x^2 + y^2\ (x,y)\in\R^2$\\
  $H:=\{(x,y)\in\R^2| x+y-2=0\}$\\
  $f_{|H}\n\min$
\end{PlSS}
\begin{PlSS}\label{plss:fsz2}
  Adott körbe maximális területű téglalapot kell tenni.\\
  Egyszerűsítés: elég az első síknegyed, mert szimmetrikus.
  $T(x,y) := 4xy$\\
  $H:=\{(x,y)\in\R^2| x^2 + y^2 - R^2 =0\}$\\
  $f_{|H}\n\max$
\end{PlSS}
Adott $m,n\in\N,\ U\subset\R^n$ nyílt,\\
$f\colon U\n\R$ és\\
$g_i\colon U\n\R\quad i=1..m$\\
$H:=\{ z\in\R^n\,|\,g_i(z)=0,\ i=1,\dotsc,m\}$ feltételek.\\
Határozzuk meg $f_{|H}$ lokális szélsőértékeit.

\begin{de}
  Az $f\colon U\n\R$ fv-nek  a $c\in U$-ban a $g_i(z)=0\ \,(i=1..m)$ feltételekre vonatkozó \emph{lokális feltételes
    minimuma van}, ha
  \[\exists K(c): f(x) \geq f(c)\quad \forall x\in K(c)\cap H\]
\end{de}

\begin{te}[Lokális feltételre vonatkozó szükséges feltétel]
  Tfh
{\listazjbetu
  \begin{enumerate}
  \item $n,m\in\N,\ U\in\R^n$ nyílt,\\$f\colon U\n\R,\ g_i: U\n\R\ \ (i=1,\dotsc,m)$ folytonosan deriválhatóak.
  \item $f$-nek a $c\in U$-ban a $g_i(c)=0\ \,(i=1,\dotsc,m)$ feltételekre vonatkozó lokális szélső
  \item $g_i'(c)\ (i=1,\dotsc,m)$ lineárisan független vektorok
  \end{enumerate}
}
  Ekkor $\exists\lambda_1,\dotsc,\lambda_m\in\R\colon \di F:= f+\sum_{i=1}^m \lambda_ig_i$ fv-nek $c$-ben  $F'(c) = 0$
\end{te}
\begin{megj}Alkalmazáshoz:
  \[ \left.\begin{array}{l}
    \left.\begin{array}{c}\partial_1F(c)=0\\\partial_2F(c)=0\\\vdots\\\partial_nF(c)=0\end{array}\right\} \text{ n db}\\
      \left.\begin{array}{c}g_1(c)=0\\\vdots\\g_m(c)=0\end{array}\right\} \text{ m db}\\
  \end{array}\right\} \text{ n+m db egyenlet a $\lambda_1,\dotsc,\lambda_m$ és $c_1,\dotsc,c_n$ ismeretlenekre} 
  \tag{$\sharp$}\label{eqs:lagrange-mult}
  \]
  Lokális szélsőérték csak ilyen $c$-ben lehet	
\end{megj}


\textbf{Alkalmazás}\\
\Aref{plss:fsz2} példa:
\begin{gather*}
  f(x,y) := 4xy\quad (\,(x,y)\in\R^2\,)\\
  g(x,y) := x^2 + y^2 -R^2\\
  F(x,y) := 4xy + \lambda(x^2 + y^2 - R^2)\\
  \left.
  \begin{gathered}
    \partial_1 F(x,y) = 4y + 2\lambda x = 0\\
    \partial_2 F(x,y) = 4x + 2\lambda y = 0  
  \end{gathered}\,
  \right\}\quad \oplus\colon 2(x+y)(\lambda+2) = 0\\
  x^2+y^2 - R^2 = 0 
  \intertext{Innen:}
  \lambda = -2\\
  x=y=\dfrac{R}{\sqrt{2}}
\end{gather*}
Azaz $\left(\dfrac{R}{\sqrt{2}},\,\dfrac{R}{\sqrt{2}}\right)$-ben lehet szélsőérték.

\begin{te}[Elégséges feltétel a lokális szélsőértékre]\ 
{\listazjbetu
  \begin{enumerate}
  \item $f,g_i\colon U\n\R$, $U\subset\R^n$ nyílt, $i=1,\dotsc,m$ kétszer folytonosan differenciálható
  \item $c=(c1,\dotsc,c_n)$, $\lambda_1,\dotsc,\lambda_m$ kielégíti \aref{eqs:lagrange-mult}-t
  \item $F:= f + \di\sum_{i=1}^m\lambda_i g_i$-nek $c$-ben lokális szélsőértéke van (feltétel nélküli: a teljes
    értelemezési tartományt figyelembe véve).  
  \end{enumerate}
}
Ekkor $f$-nek $c$-ben a $g_1=\dotsb=g_m=0$ feltételekre vonatkozó feltételes lokális szélsőértéke van.
\end{te}
\ref{plss:fsz2}: $\lambda=2$; $F(x,y) = 4xy - 2(x^2 +y^2 - R^2) = 2R^2 - 2(x+y)^2$.



% Local Variables:
% fill-column: 120
% TeX-master: t
% End:
