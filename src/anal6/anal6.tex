\documentclass[fleqn,10pt,a4paper,titlepage]{article}
\includeonly{common,komplexfv}



%
% - ---- -- PACKAGES--------------------------
%

\usepackage{amssymb}
\usepackage{amsmath}
%\usepackage[T1]{fontenc}
\usepackage[utf8]{inputenc}
\usepackage[magyar]{babel}
%\usepackage{amsthm}
\usepackage{theorem}
\usepackage{fancyhdr}
\usepackage{lastpage}
\usepackage{paralist}

%
% ---------- CODES --------------------------
%
\makeatletter
\gdef\th@magyar{\normalfont\slshape
  \def\@begintheorem##1##2{%
  \item[\hskip\labelsep \theorem@headerfont ##2.\ ##1.]}%
  \def\@opargbegintheorem##1##2##3{%
  \item[\hskip\labelsep \theorem@headerfont ##2. ##1.\ (##3)]}}
\makeatother


%
% ------------  N E W  C O M M A N D S --------
%
%\newcommand{\ob}{\begin{flushright} \leavevmode\hbox to.77778em{\hfil\vrule
%    \vbox to.675em{\hrule width.6em\vfil\hrule}\vrule}\hfil\end{flushright}}
\newcommand{\ob}{\hfill$\square$}
\newcommand{\ff}{f\in\mathbb{R}\rightarrow\mathbb{R}}
\newcommand{\fab}{f\colon (a,b)\rightarrow\mathbb{R}}
\newcommand{\fabk}{f\colon \left[a,b\right]\rightarrow\mathbb{R}}
\newcommand{\fir}{f\colon I\rightarrow\mathbb{R}}
\newcommand{\fdab}{f\in D(a,b)}
\newcommand{\fcab}{f\in C[a,b]}
\newcommand{\exist}{\exists}
\newcommand{\ek}{\Longleftrightarrow}
\newcommand{\la}{\lambda}
\newcommand{\ro}{\varrho}
\newcommand{\K}{\ensuremath{\mathbb{K}}}
\newcommand{\R}{\ensuremath{\mathbb{R}}}
\newcommand{\Q}{\ensuremath{\mathbb{Q}}}
\newcommand{\N}{\ensuremath{\mathbb{N}}}
\newcommand{\C}{\ensuremath{\mathbb{C}}}
\newcommand{\n}{\ensuremath{\to}} %azonos a rightarrow-val
%duplanyil, szuksegesseg
\newcommand{\nn}{\ensuremath{\Rightarrow}}
%\newcommand{\Omage}{\Omega}
%elegségesség, nem def :)
%\newcommand{\nb}{\Leftarrow}
\newcommand{\di}{\displaystyle}
\newcommand{\sarrow}{\downarrow}
\newcommand{\narrow}{\uparrow}
\newcommand{\lt}{<}
\newcommand{\gt}{>}
\newcommand{\Int}{\intop\limits}
\newcommand{\ures}{\varnothing}
\newcommand{\ekv}{\iff}
\newcommand{\ekviv}{\ekv}
\renewcommand{\epsilon}{\varepsilon}
\newcommand{\eps}{\varepsilon}
%
% ------------  NEW PART DEFS -----------------
%
\newcounter{Szaml}


\theoremstyle{magyar}
\theoremheaderfont{\itshape\bfseries}
\newtheorem{de}{definíció}[section]
\newtheorem{te}{tétel}[section]
\newtheorem{bi}{bizonyítás}[section]
\newtheorem{ko}{következmény}[section]
\newtheorem{me}{megjegyzés}[section]
\newtheorem{al}{állítás}[section]


\newenvironment{korlista}{\begin{enumerate}[\quad1$^\circ$]}{\end{enumerate}}

\newenvironment{biz}{\begin{trivlist}\item\relax\mbox{\textbf{Bizonyítás.\enskip}}\ignorespaces}{\ob\end{trivlist}}
\newenvironment{Biz}{\begin{trivlist}\item\relax\mbox{\textbf{Bizonyítás.\enskip}}\ignorespaces\begin{korlista}}{\ob\end{korlista}\end{trivlist}}
\newenvironment{kov}{\begin{trivlist}\item\relax\mbox{\textbf{Következmény.\enskip}}\ignorespaces}{\end{trivlist}}
\newenvironment{megj}{\begin{trivlist}\item\relax\mbox{\textbf{Megjegyzés.\enskip}}\ignorespaces}{\end{trivlist}}
\newenvironment{Megj}{\begin{megj}\begin{korlista}}{\end{korlista}\end{megj}}
\newenvironment{pl}{\begin{trivlist}\item\relax\mbox{\textbf{Példa.\enskip}}\ignorespaces}{\end{trivlist}}
\newenvironment{Pl}{\begin{pl}\begin{korlista}}{\end{korlista}\end{pl}}
\DeclareMathOperator{\D}{D}
\newenvironment{bizlist}{\setcounter{Szaml}{1}
    \begin{list}{\alph{Szaml})\hfill}
    {\usecounter{Szaml}\setlength{\itemsep}{0pt}
    \setlength{\itemindent}{-\labelsep}
    \setlength{\listparindent}{0pt}}}{\end{list}}




%
% - - - -- - - S E T T I N G S ----------------
%
%\setlength{\parindent}{0pt}
%\setlength{\parskip}{\baselineskip}
\addtolength{\voffset}{-1cm}
\addtolength{\textheight}{2cm}
%\addtolength{\marginparwidth}{-1cm}
\addtolength{\hoffset}{-1cm}
\addtolength{\textwidth}{2cm}
\setlength{\headheight}{23pt}
%
\pagestyle{fancy}

  \renewcommand{\sectionmark}[1]{\markboth{\Roman{section}. tétel\\#1}{}}

\newcommand{\mktoc}{
  \pagenumbering{roman}
  \setcounter{page}{1}
  \lhead{\textbf{\thepage}}
  \cfoot{}
  \tableofcontents
  \newpage
  \lhead{\textbf{\thepage}}%/\pageref{LastPage}}
  \pagenumbering{arabic}
  \setcounter{page}{1}
}


\usepackage{booktabs}

\newcommand{\eqrho}{\stackrel{\varrho}{=}}
\newcommand{\eqrhon}[1]{\stackrel{\varrho_{#1}}{=}}
\newcommand{\MT}{\ensuremath{(M,\varrho)}\,}
\newcommand{\MTn}[1]{\ensuremath{(M,\varrho_{#1})}\,}
\newcommand{\sorozat}{\ensuremath{(a_n)\colon \N\n M}\,}
\newcommand{\sorozatn}[1]{\ensuremath{(a_{#1})\colon \N\n M}}

\renewcommand{\sectionmark}[1]{\markboth{\Roman{section}. fejezet\\#1}{}}
\newcommand{\hullam}{\widetilde}
\newcommand{\mr}[1]{(M_{#1},\,\ro_{#1})}
\newcommand{\fmm}{f\in M_1\n M_2}
\newcommand{\X}{\ensuremath{\mathcal{X}}}
\newcommand{\lp}{\ensuremath{l_p}}
\newcommand{\norma}[1]{\ensuremath{\left\Vert #1\right\Vert}}
\newcommand{\norman}[2]{\ensuremath{\norma{#1}^{(#2)}}}
\newcommand{\Norma}{\norma{\cdot}}
\newcommand{\Norman}[1]{\ensuremath{\Norma^{(#1)}}}
\newcommand{\opnorma}[1]{\ensuremath{|||#1|||}}
\newcommand{\NT}{\ensuremath{(\X,\,\Norma)}}
\newcommand{\skalar}[2]{\ensuremath{\langle#1,\,#2\rangle}}
\newcommand{\Skalar}{\skalar{.}{.}}
\newcommand{\skalarsz}{\Skalar}
\newcommand{\ET}{\ensuremath{(\X,\,\skalarsz)}}
\newcommand{\nullelem}{\mathsf{0}}
%\renewcommand{\square}{\blacksquare}
\newcommand{\RnRm}{\R^n\to\R^m}
\newcommand{\RnRn}{\R^n\to\R^n}
\newcommand{\RnR}{\R^n\to\R}
\newcommand{\RRm}{\R\to\R^m}
\newcommand{\Rnrm}{\RnRm}
\newcommand{\Rnrn}{\RnRn}
\newcommand{\RRn}{\R\times\R^n}
\newcommand{\Linearis}{\ensuremath{\mathcal{L}(\R^n,\,\R^m)}}
\DeclareMathOperator{\intD}{int\,D}
\DeclareMathOperator{\Folyt}{C}
\newcommand{\folyt}[1]{\Folyt\{#1\}}
\newcommand{\derivp}[1]{\D\{#1\}}
\newcommand{\dern}[2]{\D^{#1}\{#2\}}
\newcommand{\der}[1]{\derivp{#1}}
\newcommand{\vekt}[1]{\mathbf{#1}}
\newcommand{\Rmn}{\ensuremath{\R^{m\times n}}}
\DeclareMathOperator{\grad}{grad}
\newcommand{\vfi}{\varphi}
\newenvironment{spec}{\begin{trivlist}\item\relax\mbox{\textbf{Spec. esetek.\enskip}}\ignorespaces}{\end{trivlist}}
\newtheorem{lemma}{lemma}[section]
\newtheorem{pelda}{PÉLDA}[section]
\DeclareMathOperator{\sgn}{sgn}
\DeclareMathOperator{\ch}{ch}
\DeclareMathOperator{\sh}{sh}
\DeclareMathOperator{\arctg}{arctg}
\newcommand*{\Der}{\D}
\newcommand*{\Oint}{\oint\limits}
\DeclareMathOperator{\Rint}{R}
\DeclareMathOperator{\re}{\Re e}
%\newcommand{\re}{\Re}
\DeclareMathOperator{\im}{\Im m}
%\newcommand{\im}{\Im}
\newcommand{\ERTT}{\mathcal{D}}
\newcommand{\fcc}{f\in\C\to\C}
\DeclareMathOperator{\fr}{fr}
\DeclareMathOperator{\intG}{int}
\newcommand{\Z}{\mathbb Z}

\newcommand{\listazjbetu}{
  \renewcommand{\theenumi}{\alph{enumi}}
  \renewcommand{\labelenumi}{(\theenumi)}
}
\newcommand{\listazjromai}{
  \renewcommand{\theenumi}{\alph{enumi}}
  \renewcommand{\labelenumi}{(\theenumi)}
}
\newcommand{\listabetu}{
  \renewcommand{\theenumi}{\alph{enumi}}
  \renewcommand{\labelenumi}{\theenumi}
}
\newcommand{\listaszamkor}{
  \renewcommand{\theenumi}{\alph{enumi}}
  \renewcommand{\labelenumi}{\theenumi$^\circ$}
}
\newenvironment{enumzjromai}{\listazjromai\begin{enumerate}}{\end{enumerate}}
\newenvironment{enumzjbetu}{\listazjbetu\begin{enumerate}}{\end{enumerate}}

\newenvironment{enumzjr}{\begin{enumzjromai}}{\end{enumzjromai}}
\newenvironment{enumzjb}{\begin{enumzjbetu}}{\end{enumzjbetu}}


\DeclareRobustCommand{\tmspace}[3]{%
  \ifmmode\mskip#1#2\else\kern#1#3\fi\relax}
\providecommand*{\negmedspace}{\tmspace-\medmuskip{.2222em}}
%
% written by Claudio Beccari
% published in TUGboat, Volume 18 (1997), No. 1, 39--48
%
% slightly modified by FW
\makeatletter
\providecommand*{\diff}%
  {\@ifnextchar^{\DIfF}{\DIfF^{}}}
\def\DIfF^#1{%
  \mathop{\mathrm{d{}}}%%% original version: {\mathstrut d}}%
    \nolimits^{#1}\gobblespace}
\def\gobblespace{%
  \futurelet\diffarg\opspace}
\def\opspace{%
  \let\DiffSpace\negmedspace%%% original version: \,
  \ifx\diffarg(%
    \let\DiffSpace\relax
  \else
    \ifx\diffarg[%
      \let\DiffSpace\relax
    \else
      \ifx\diffarg\{%
        \let\DiffSpace\relax
      \fi\fi\fi\DiffSpace}
\makeatother

\providecommand*{\deriv}[3][]{%
    \frac{\diff^{#1}#2}{\diff #3^{#1}}}

\title{Analízis 6 előadás jegyzet (2005/2006)\\(Szili László előadása alapján)}
\author{Tóth László Attila (panther@elte.hu)}
\date{}
\begin{document}
  \maketitle
  \mktoc
  \part{Komplex függvénytan}
\section{Komplex függvénytan}

\begin{megj}Az $\R\to\R$ és $\C\to\C$ függvények részben hasonlóak, részben különbözőek.\\
  
  \begin{tabular}{@{}lcl@{}}\toprule
    \multicolumn{1}{c}{$\R\to\R$} & & \multicolumn{1}{c}{$\C\to\C$}\\\midrule
    folytonosság & \emph{hasonló} & folytonosság\\
    differenciálhatóság & \emph{lényeges} & differenciálhatóság \\
    & \emph{különbségek} & \qquad(ha diffható, akárhányszor diffható)\\
    \bottomrule    
  \end{tabular}
\end{megj}

\subsection{Kapcsolat az $\R^2$ és a $\C$ között}
\begin{itemize}
  \item algebrai alak: $ z = x + iy$. Valós rész: $\re z = x$, képzetes rész: $\im z = y$
  \item trigonometrikus alak: $z= r\cos\vfi + i\cos \vfi$, $r\geq 0$ , $0\leqq \vfi < 2\pi$
  \item exponenciális alak: $z = r e^{i\vfi}$\quad (Euler-formulábólól)
\end{itemize}

\begin{te}[Metrika $\C$-n] A $\ro(z_1,\,z_2) := |z_1-z_2|,\quad (z_1,z_2\in\C)$ metrika \C-n és $(\C,\ro)$ \emph{teljes}
  MT.
\end{te}
\begin{biz} Lásd $\R$ eset
\end{biz}

\begin{te}[$\C$ és $\R^2$ izometrikusan izomorfak] Legyen: $I\colon \C\owns z=x+iy\mapsto (x,y)\in\R^2$. Ekkor
  \begin{enumerate}
    \item $I\colon\C\to\R^2$ bijekció
    \item művelettartó: $I(z_1 + z_2) = I(z_1) + I(z_2)$
    \item izometrikus: $\norma{I(z_1)-I(z_2)}_2 = |z_1-z_2|$
    \end{enumerate}
\end{te}

\subsection{$\C\to\C$ és $\R^2\to\R^2$ típusú függvények}
\begin{gather*}
  f\in\C\to\C,\ \forall z=x+iy\in \ERTT_f\colon f(z) = \re f(z) + i \im f(z)\\
  f_1(x,y) := \re f(z),\quad f_2(x,y) := \im f(z)\\
  f\in \C\to\C \iff \hat{f} := (f_1,f_2)\in \R^2\to \R^2\\
  \hat{f} = I\circ f\circ I^{-1}
\end{gather*}
Itt $f_1$ az $f$ függvény \emph{valós} része, $f_2$ a \emph{képzetes} része
\begin{Pl}
  \item $f(z) := z^2 = (x+iy)^2 = (x^2-y^2)  + 2ixy$\\
    $\hat{f}(x,y) = (x^2-y^2,\, 2xy)$
  \item $f(z) := e^z = e^{x+iy}= e^x + e^{iy} = e^x(\cos y + i\sin y) = e^x \cos y + i e^x \sin y$\\
    $\hat{f}(x,y) = (e^x\cos y,\, e^x \sin y)$
\end{Pl}

\subsubsection{$\C\to\C$ függvények szemléltetése}
\begin{Pl}
  \item $f(z) := z^2$\\
    a) $ z=r(\cos t + i\sin t)$ $z^2=r^2(\cos2t + i\sin2t)$\\
    b) $0\leq t < \infty,\quad z = \epsilon t $, $z^2=\epsilon^2t^2$
  \item $f(z) = e^z\quad (z\in\C)$ igazolható...
\end{Pl}
Lásd a kiadott anyag - vizsgára kell
\subsection{A $\C\to\C$ függvények folytonossága}
$(\C,\ro),\ \ro(z_1,z_2) := |z_1 - z_2|$ teljes metrikus tér. Topológiai fogalmak

\begin{enumzjb}
  \item nyílt, zárt, kompakt halmaz, korlátosság. Az $A\subset \C$ halmaz kompakt $\iff$ $A$ korlátos és zárt.
  \item folytonosság, haátérték - környezetek (a korábbiak speciális esete); folytonos függvények tulajdonságai
\end{enumzjb}


Persze közvetlenül is lehetne vizsgálni (MT nélkül):\\ 
$f\in\folyt{a} \iff \forall \epsilon > 0\ \exists \delta>0\ \forall
z\in K_\delta(a)\cap \ERTT_f\colon |f(z)-f(a)| < \epsilon$

\begin{te}
  $\C\to\C \owns f=f_1 + if_2\quad (f_1,f_2\in \R^2\to\R)$\\
  \[f\in\folyt{a}\iff f_1,\,f_2\in\folyt{(a_1,\,a_2)} \iff \hat{f}=(f_1,\,f_2)\in\R^2\to\R^2 \in\folyt{(a_1,a_2)}\]
\end{te}
\begin{biz}trivi, def, alapján
\end{biz}

\subsection{Diferenciálható $\C\to\C$ függvények}
\begin{de}[deriválhatóság] $f\in\C\to\C,\ a\in\intD_f$
  \begin{gather*}
    f\in\der{a} :\ekviv \exists \di\lim_{z\to a}\dfrac{f(z)-f(a)}{z-a} = A \in \C\\
    A =: f'(a)
  \end{gather*}  
\end{de}
\begin{megj} A definíció ugyanaz, mint $\R\to\R$ esetben. Azonban látni fogjuk, hogy ez jóval erősebb megkötés, mint
    $\R\to\R$ esetben (pl. ha 1x diffható, bárhányszor deriválható).
\end{megj}

\begin{te}[lineáris közelítés] $f\in \C\to\C,\ a\in\intD_f$
  \[ f\in\der{a} \iff \exists A \in \C \text{ és }\exists \epsilon\colon\C\to \C,\ \di\lim_a\epsilon = \colon
  f(z)-f(a)=A(z-a) + \epsilon(z)\cdot(z-a)\quad z\in\ERTT_f\]
\end{te}
\begin{biz}
  Mint $\R\to\R$ esetben.
\end{biz}
\begin{te}[Cauchy-Riemann-egyenletek]\ \\
  \begin{enumzjb} 
    \item $f\in\C\to\C,\ a=a_1+ia_2\in\intD_f,\ f=f_1+if_2,\ \hat{f}=(f_1,\,f_2)\in\R^2\to\R^2$
      \begin{gather*}
	f\in\der{a}\iff \begin{cases}
	  \hat{f} \text{ deriválható }(a_1,\,a_2)\text{-ben és}\\
	  (\partial_1f_1)(a_1,a_2) = (\partial_2f_2)(a_1,a_2)\\
	  (\partial_2f_1)(a_1,a_2) = -(\partial_1f_2)(a_1,a_2)
	\end{cases}
      \end{gather*}
    \item $f'(a) = A = A_1+iA_2 = (\partial_1f_1)(a_1,a_2) - i(\partial_2f_2)(a_1,a_2)$
  \end{enumzjb}
\end{te}

\begin{biz}
  \begin{gather*}
    f\in\C\to\C,\ f=f_1+if_2,\ a\in\intD_f,\ a=a_1+ia_2,\ z=z_1+iz_2.\\
    f\in\der a \iff \left\{\begin{array}{l}\exists A=A_1+iA_2\in \C,\ \exists \epsilon=\epsilon_1 + i\epsilon_2\in
    \C\to\C, \ \di\lim_a\epsilon = 0\\f(z)=f(a)+A(z-a)+\epsilon(z)(z-a)\quad \forall z\in D_f\end{array}\right.
    \intertext{Valós és képzetes részek}
    f_1(x,y) = f_1(a_1,a_2) + (A_1(x-a_1)-A_2(y-A_2)) + \epsilon_1(x,y)(x-a_1) - \epsilon_2(x,y)(y-a_2)\\
    f_2(x,y) = f_2(a_1,a_2) + A_2(x-a_1)-A_1(y-A_2)) + \epsilon_2(x,y)(x-a_1) - \epsilon_1(x,y)(y-a_2)\\
    \lim_{a_1,a_2}\epsilon_i = 0\ (i=1,2)
    \intertext{ami ekivalens a következőkkel:}
    f_1,f_2\in  \der{(a_1,2_2)}, \quad\begin{array}{l}
    A_1 =\partial_1f_1(a_1,a_2)\\
    -A_2=\partial_2f_2(a_1a_2)\end{array},\quad\begin{array}{l}
    A_1 =\partial_1f_2(a_1,a_2)\\
    A_2=\partial_2f_2(a_1a_2)\end{array}
    \intertext{Ebből megkapjuk a Cauchy-Riemann egyenleteket.\newline A \textit{b)} rész biztonyítása triviálisan adódik:}
    f'(a)=A_1+iA_2 =  (\partial_1f_1)(a_1,a_2) - i(\partial_2f_2)(a_1,a_2)    
  \end{gather*}
\end{biz}
\subsubsection{Invertálhatóság}
eml: $\R\to\R$, $\R^n\to\R^n$
\begin{te}
  Ha $f\in\C\to\C$ differenciálható az $a\in\intD_f$ pontban és $f'(a) \neq 0$, akkor $\exists K_r(a)$, amelyben az $f$
  invertálható.
\end{te}

\begin{biz}
\begin{gather*}
  f=f+if_2,\ \hat f = (f_1,f_2) \in \R^2\to\R^2\text{ (ha $f$ invertálható, $f$ is az)}\\
  \det (\hat{f}'(a_1,a_2)) = \det \begin{bmatrix}\partial_1f_1(a_1,a_2) & \partial_2f_1(a_1,a_2) \\
    \partial_1f_2(a_1,a_2) & \partial_2f_2(a_1,a_2)\end{bmatrix} \overset{\text{C-R}}{=}\\\overset{\text{C-R}}{=}
  \det\begin{bmatrix} \partial_1f_1(a_1,a_2) & \partial_2f_1(a_1,a_2) \\  -\partial_2f_1(a_1,a_2) &
  \partial_1f_1(a_1,a_2)\end{bmatrix} = (\partial_1f_1(a_1,a_2))^2 - (\partial_2f_1(a_1,a_2))^2 = \vert f'(a)\vert\neq 0  
\end{gather*}
Ekkor $\hat f$ nvertálható $K_r(a_1,a_2)$-ben  (inverz fv tétel) $\nn$ $f$ is invertálható $K_r(a)$-ban.
\end{biz}

\begin{te}[Deriválható függvények alaptulajdonságai] $\fcc,\ a\in\intD_f$
  \begin{enumerate}
    \item $f\in \der{a} \nn f\in \folyt{a}\qquad \not\Leftarrow$
    \item Műveleti tételek: +, *, /, kompozíció lásd $\R\to\R$
    \item Hatványsor deriválhatósága      
  \end{enumerate}
\end{te}

\noindent \underline{Elemi függvények} def, elemi tul ($\exp,\ \sin,\ \cos,\ \sh,\ \ch$; addíciós tételek, négyzetes öf,
Euler-formula...)

\begin{kov}\begin{korlista}
  \item polinomok, racionális $\C\to\C$ függvények deriválhatók
  \item $\exp,\ \sin,\ \cos,\ \sh,\ \ch$ deriválhatóak, deriváltjaik...
    \end{korlista}
\end{kov}

\begin{pl} Nem deriválható függvények, ezek egyetlen $a\in\C$ pontban sem deriválhatóak
  \begin{enumzjb}
  \item $f(z) := \re z\ (z\in\C)$, ui legyen $a\in \C$ rögzített,
      \[\di\dfrac{f((a+h)-f(a)}h = \dfrac{\re(a+h) - \re(a)}h = \begin{cases}1& h\in\R \\
	0 & h\text { tiszta képzetes} \end{cases}\]
      Tehát nem létezik határérték $h\to 0$ esetén
    \item $f(z) :=  \overline{z}\ (z\in\C)$, ui legyen $a\in \C$ rögzített,
      \[\di\dfrac{f((a+h)-f(a)}h = \dfrac{\overline{a+h} - \overline{a}}h = \dfrac{\overline{h}}h = \begin{cases}1& h\in\R \\
	-1 & h\text{ tiszta képzetes} \end{cases}\]      
  \end{enumzjb}  
\end{pl}

\subsubsection{Elemi függvények vizsgálata}
\begin{enumerate}
\item Lineáris függvények
\item $z\mapsto z^2$
\item gyökfv
\item $\exp$, infertálhatósága $\nn \log$ fv. Az $\exp$ periodikus!
\item $\sin,\,cos$ + inverzeik
\end{enumerate}

\begin{de}[Analitikus fv]
  Az $f$ fv. analitikus (v. holomorf v. reguláris) a $T\subset\C$ tartományon, ha $f$ deriválható a $T$ minden
  pontjában.\\
  Jel: $\mathcal A(T) \owns f$
\end{de}
\subsubsection{Az $\R^2\to\R$ harmonikus függvényei}

Legyen $T \subset \C tartomány$, ez megfeleltethető egy $D\subset \R^2$ tartománynak, hogy
$ z=x+iy\in T \iff (x,y) \in D$\\
Tfh. $f=f_1+if_2) \in \mathcal A(T)$, így (Cauchy Riemann alapján)
\begin{gather*}
  \partial_1f1 = \partial_2f_2\\
  \partial_2f1 = -\partial_2f_2
  \intertext{Tovább deriválva (később látjuk, többször deriválhatóak) + Young tételből:}
  \left\{\begin{array}{l}\partial_{11}f_1 + \partial_{22}f_2 =0\\
  \dfrac{\partial^2f_1}{\partial x^2} +  \dfrac{\partial^2f_1}{\partial x^2} =0
  \end{array}\right\} \text{ Laplace-egyenlet}
\end{gather*}
másodrendű dirrefenciál-egyenlet, ez igaz $f_2$-re is.

\begin{de}[Harmonikus függvény]A $g\colon D\to\R\quad(D\subset\R^2$ tartomány $)$ kétszer folytonosan deriválható fv
  \emph{harmonikus függvény} a $D$ tartományon, ha
  \[ \di\dfrac{\partial^2g}{\partial x^2}(x,y) + \dfrac{\partial^2g}{\partial y^2}(x,y) = 0\quad(\forall (x,y)\in D) \]
\end{de}

\begin{te}Ha $f=f_1+if_2\in \mathcal A(T)\nn f_1,\,f_2$ harmonikus a $T$-nek megfelelő $D\subset \R^2$ tartományon
\end{te}
\begin{megj}Differenciálható komplex függvények valós, kézetes része harmonikus, ezért sok ilyen van\end{megj}

\begin{de}[Harmonikus társak]$D\subset\R^2$ tartomány, $f_1,f_2\colon D\to\R$ egymás harmonikus társi, ha $\exists
  f=f_1+if_2 \in \C\to\C$, ami deriválható.
\end{de}

\subsection{Komplex vonalintegrál}
\subsubsection{Valós vonalintegrálok}
\begin{de}[Sima út]$n\in\N\ \vfi\colon[a,\,b]\to\R^n$ folytonosan deriválható függvényt\\\emph{$\R^n$-beli sima út}nak
    nevezzük.\\  Az $R_\vfi = \Gamma\subset\R^n$ halmaz \emph{sima görbe}, $\vfi$ a $\Gamma$ görbe egy paraméterezése.
\end{de}

\begin{de}[Szakaszonként sima út]$a,b\in\R;\ a\leq b$. A $\vfi\colon[a,b]\to \R^n$ függvény  \emph{$\R^n$-beli
    szakaszonként sima út}, ha 
{\listazjromai
  \begin{enumerate}
  \item $\vfi\in\Folyt$
  \item $\exists a=t_0<t,\ 1<\dotsb<t_m=b$: $\vfi_{|[t_i,\,t_i+1]}\ \ i=1,\dotsc,m-1$ sima út.
\end{enumerate}
}
\end{de}

\begin{Pl}
\item Szakasz: $a,b\in \R^n\ \vfi(t) := a+t(b-a)\quad(t\in[0,\,1])$
\item Töröttvonal - szakaszonként sima út
\item Kör: $\vfi(t) := (\sin t,\cos t)\quad t\in[0,2\pi]\\
  R_\vfi=\Gamma$
\end{Pl}


\begin{de}[Szakaszonként sima utak egyesítése]
  $\vfi\colon [a,a+h]\to\R^n$\\$\psi\colon[b,b+k]\to\R^n$ szakaszonként sima utak, és tfh: $\vfi(a+h)=\psi(b)$, azaz
  $\vfi$ végpontja megegyezik $\psi$ kezdőpontjával. \\
  A $\vfi$ és $\psi$ egyesítése $(\vfi\cup\psi)$:
\[\Phi(t) = \begin{cases}\vfi(t) & t\in[a,a+h]\\\psi(t-a-h+b) & t\in[a+h,a+h+k]\end{cases}\]
\end{de}

\begin{de}[$\vfi$ ellentettje] $\widetilde{\vfi} := \vfi(2a+h-t)\qquad(t\in[a,\,a+h])$\\
  az út $a+h\to a$ irányú lett.
\end{de}

\begin{te}Legyen $U\subset \R^n$ nyílt.\\
  $U$ összegüggő $\ekviv \forall x,y\in  U$ összeköthető $U$-beli szakaszonként
  sima úttal.
\end{te}

\begin{de}[Tartomány]Az $U\subset \R^n$ halmaz \emph{tartomány}, ha
{\listazjromai
  \begin{enumerate}
    \item $U$ nyílt $\R^n$-ben
    \item $U$ összefüggő
  \end{enumerate}
}
\end{de}
\begin{de}[Úton vett vonalintegrál]
  Legyen $U\subset \R^n$ tartomány, $f\colon U\to\R^n$ \underline{folytonos}, $\vfi\colon [a,b]\to\R^n$ szakaszonként
  sima. Ekkor
\[\Int_a^b\skalar{f\circ\vfi}{\vfi'} = \Int_a^b\skalar{f(\vfi(t))}{\vfi'(t)\,}\diff t =: \Int_\vfi f\]
szám az $f$ függvény $\vfi$ útra vett vonalintegrálja.
\end{de}
\begin{Megj}
  \item $f$ folytonos $\nn$ az integrandus folytonos $\nn$ az integrál létezik.
\item $n=1,\ \vfi(t) := t\quad t\in[a,b]$
\[\Int_\vfi f \text{ az } \Int_a^bf(t)\diff t \text{ Riemann-integrálja}\]
\end{Megj}

\begin{te}[A vonalintegrál egyszerű tulajdonságai]
  $U\subset \R^n$ tartomány,\\$\vfi\colon [a,a+h]\to\R^n$ és $\psi\colon [b,b+k]\to\R^n$ szakaszonként sima utak,
  $\vfi(a+h) = \psi(b)$.\\$f,g\colon U\to\R^n$ folytonos. Ekkor
  \begin{enumerate}
  \item $\di\Int_\vfi(\lambda_1 f +\lambda_2g) = \lambda_1\Int_\vfi f+ \lambda_2\Int_\vfi g$
  \item $\di\Int_\vfi f = -\Int_{\widetilde{\vfi}}$\qquad (ellentett út)
  \item $\di\Int_{\vfi\cup\psi}\!\!\! f = \Int_\vfi f + \Int_\psi f$
  \item $\di\Big\vert \Int_\vfi f\Big\vert \leqq M\cdot l(\vfi)$, ahol $l(\vfi)$ a $\vfi$ (vagy a $\Gamma$ görbe)
  hossza és $M:= \max \{\,\norma{f(x)}_2:x\in R_\vfi\}$
  \end{enumerate}
\end{te}

\begin{de}[Primitív függvény]$U\subset\R^n$ tartomány, $f\colon U\to\R^n$.\\
  Az $F\colon U\to\R^n$ függvény az $f$ primitív függvénye, ha
  {\listazjromai
    \begin{enumerate}
    \item $F\in\Der$
    \item $F'(x) = f(x)\quad (\forall x\in U)$
    \end{enumerate}
  }
\end{de}

\begin{megj}
  Ha $F\in\Der$: $F'=(\partial_1F,\dotsc\partial_nF) =(f_1,\dotsc,f_n)=f$ 
\end{megj}

\begin{te}\ 
  \begin{enumzjromai}
  \item Ha $F\colon U\to\R$ az $f$ primitív függvénye $\nn \forall c\in\R\colon F+c$ is az
  \item Ha $F_1,\,F_2\colon U\to\R$ az $f$ primitív függvényei $\nn \exists c\in\R\colon F_1(x)-F_2(x) = c \quad \forall
  x\in U$
  \end{enumzjromai}
\end{te}
\begin{te}[Newton-Leibniz]
  Tfh:
\begin{enumzjromai}
  \item $U\subset \R^n$ tartomány
  \item $f\colon U\to \R^n$ folytonos
  \item $\vfi\colon [a,b]\to U$ szakaszonként sima út
  \item $f$-nek $\exists F$: a primitív fv-e
\end{enumzjromai}
Ekkor $\di\Int_\vfi f = F(\vfi(b))-F(\vfi(a))$
\end{te}

\begin{te}
  $U\subset \R^n$ taromány, $f\colon U\to\R$ folytonos.\\
  $f$-nek létezik primitív függvénye $\ekviv \left\{\begin{array}{l}\forall U\text{-ban haladó szakaszonként sima és
  zárt }\vfi\text{ útra:}\\\di\Int_\vfi f= 0\end{array}\right.$
\end{te}

\begin{te}[Szükéges feltétel a primitív függvény létezésére]
  $U\subset\R^n$ tartomány,\\$f\colon U\to\R^n$
  \begin{enumzjr}
    \item $f$ \underline{deriválható}
    \item $f$-nek létezik primitív függvénye
  \end{enumzjr}
  Ekkor $f'$ deriváltmátrix szimmetrikus, azaz $\partial_if_j=\partial_jf_i\ (\forall 1\leq i,j\leq n)$ és
  $f=(f_1,\dotsc,f_n)$
\end{te}

\begin{Megj}
\item $\R\to\R$ esetén $\forall$ folytonos függvénynek létezik primitív függvény\\
  Ha $n\geq 2$, akkor $\exists f$ deriválható, melynek nincs primitív függvénye.
\item Csillagtartományon ez a szükséges felétel elégséges is
\end{Megj}
\begin{de}[Csillagtartomány]
  $U\subset \R^n$ az $a\in U$ pontra nézve csillagtartomány, ha $\forall x\in U: [a,x]\subset U$
\end{de}

\begin{te}[Elégséges feltétel a primitív függvény létezésére]
  Tfh:
  \begin{enumzjr}
  \item $U\subset \R^n$ az $a\in U$-ra csillagtartomány
  \item $f\colon U\to\R^n$ folytonosan deriválható
  \item $f'$ deriváltmátrix szimmetrikus
  \end{enumzjr}
  Ekkor $F$-nek $\exists$ primitív függvénye, az
  \[\di U\owns x\mapsto\!\!\Int_{[a,x]}\!\!\!f\]
  az  $f$ egy $a$-ban eltűnő primitív függvénye
\end{te}

\subsubsection{Komplex görbék, utak}
$\vfi:=\vfi_1 + i\vfi_2\colon [\alpha,\beta]\to \C\quad(\vfi_i\colon[\alpha,\beta]\to\R)$\\
\underline{egyszerű út}, ha $\vfi\in C$\\
\underline{sima út}, ha $\vfi\in C^1$\\
\underline{szakaszonként sima út}....\\
\underline{görbe}:$\vfi\in\C$, $\Gamma=R_\vfi$, $\vfi$ szakaszonként sima út\\
$\vfi \cup \psi$: utak v. görgék egyesítése\\
$\widetilde{\vfi}$ a $\vfi$ ellentettje\\\\

\begin{Pl}
\item $[a, b]$ irányított szakasz: $a,b\in \C,\quad \vfi(t) := a + t(b-a)\quad t\in[0,1]$
\item $a$ középpontú, $R$ sugarú kör: $\vfi(t) := a + Re^{it}\quad t\in[0,2\pi]$
\end{Pl}

\subsection{Komplex halmazok}
$(C, \ro),\ \ro(z_1,\,z_2) := | z_1 - z_2 |$ teljes metrikus tér $\nn$ topológiai fogalmak
\begin{enumzjb}
  \item $A\subset C$ zárt, ha minden pontja belső pont; zárt, ha...
  \item $A\subset C$ összefüggő $\iff$ bármely két pontja összeköthető az $A$-ban haladfó szakaszonként sima úttal
  \item tartomány: nyílt, összefüggő halmaz
  \item csillagtartomány: $T\subset \C$ csillagartomány, ha $\exists a\in T,\ \forall x\in T\colon [a,x]\subset T$
\end{enumzjb}

\subsubsection{Egyszeresen összefüggő halmazok értelmezése}
\begin{de} $-\infty<\alpha < \beta < +\infty,\ \vfi\colon [\alpha,\beta] \to \C$ folytonos függvényt \emph{egyszerű,
    zárt út}nak nevezzük, ha
  \begin{enumzjr}
    \item $\vfi(\alpha) = \vfi(beta)$
    \item $\vfi_{|[\alpha,\beta]}$ injektív
    \end{enumzjr}
\end{de}

\begin{te}[Jordan] Bármely $\vfi$ egyszerű, zárt út esetén $\exists A,B\subset\C$ tartomány:
  \begin{enumzjr}
    \item $A\cap B=\emptyset; \quad A\cup B=\C\setminus R_\vfi$
    \item $\fr A = \fr B = R_\vfi$ (a határpontok halmaza)
    \item Az $A$ korlátos, a $B$ nem korlátos
  \end{enumzjr}
\end{te}

\begin{de}A $\vfi$ egyszerű, zárt görbe belseje: $\intG \vfi := A$ (az előző tételből)
\end{de}
\begin{de}
  A $\emptyset \neq T\subset \C$ tartomány \emph{egyszeresen összefüggő}, ha $\forall$ olyan $\vfi$ egyszerű, zárt körbe
  esetén, melyre teljesül, hogy $R_\vfi\subset T$, az is igaz, hogy $\intG \vfi \subset T$
\end{de}

\subsection{A komplex vonalintegrál értelmezése}
\begin{de} $f\in\C\to\C,\ f$ folytonos, $\vfi=\vfi_1+\vfi_2\colon [\alpha,\beta]\to \C$ szakaszonként sima út,
  $R_\vfi\subset D_\vfi$
\[ \di \Int_\vfi f = \Int _\alpha^\beta f\circ \vfi \cdot \vfi' \in C\]
\end{de}

\noindent Elemi tulajdonságai:\\
\begin{enumzjb}
  \item lineáris: $\di\Int_\vfi (\lambda_1 f + \lambda_2 g) = \lambda_1 \Int_\vfi g + \lambda_2 \Int_\vfi g$
  \item egyesítés: $\di\Int_{\vfi \cup \psi }f = \Int_\vfi f+ \Int_\psi f$
  \item ellentett: $\di\Int_{\Tilde{\vfi}} f = - \Int_\vfi f$
  \item $ \di\left\vert \int_\vfi f\right\vert = ML$, ahol $L$ a görbe hossza, $M := \max_{z\in R_\vfi |f(z)|}$
\end{enumzjb}
\begin{te}[Kapcsolat az $\R^2\to\R^2$ fv vonalintegráljával]
  $f\colon \C\to\C,\ f\in\Folyt\\f=f_1 + if_2,\ f\colon \R^2\to\R,\ \vfi=\vfi_1 +i\vfi_2\colon [\alpha,\beta]\to\C$
  szakaszonként sima út.\\
  $\hat f = (f_1,f_2)\colon \R^2\to\R^2,\ \hat\vfi := (\vfi_1,\vfi_2)\colon[\alpha,\beta]\to\R$. Ekkor 
  \[\di\Int_\vfi f = \Int_{\hat\vfi}\begin{bmatrix}f_1\\-f_2 \end{bmatrix} + i\Int_{\hat\vfi}\begin{bmatrix}f_2\\f_1
  \end{bmatrix}\]
\end{te}

\begin{biz}
  \begin{gather*}
    \di\Int_\vfi f = \Int_\alpha^\beta  f\circ \vfi \cdot \vfi' =\Int_\alpha^\beta \re( f\circ \vfi \cdot \vfi') +
    i \Int_\alpha^\beta  \im( f\circ \vfi \cdot \vfi') = \\=\Int_\alpha^\beta  \re[(f_1\circ \vfi + if_2\circ \vfi)
      (\vfi_1'+i\vfi_2')] + i\Int_\alpha^\beta  \im[(f_1\circ \vfi + if_2\circ \vfi) (\vfi_1'+i\vfi_2')] =\\
    =\Int_\alpha^\beta  (f_1\circ \vfi\cdot \vfi_1' - f_2\circ \vfi \cdot \vfi_2') 
    + i\Int_\alpha^\beta  (f_2\circ \vfi\cdot \vfi_1' + f_1\circ \vfi\cdot \vfi_2') = \\
    =  \Int_\alpha^\beta \skalar{ \begin{bmatrix} f_1\\-f_2\end{bmatrix} \circ  \hat\vfi}{\begin{bmatrix}
    \vfi_1'\\\vfi_2'\end{bmatrix}} + i\Int_\alpha^\beta \skalar{ \begin{bmatrix} f_2\\f_1\end{bmatrix} \circ
    \hat\vfi}{\hat\vfi'}    
  \end{gather*}
\end{biz}

\begin{te}[Cauchy alaptétel speciális esete]
  Ha $f\in\C\to\C$ és
  \begin{enumzjr}
    \item $D_f$ csillagtartomány
    \item $f\in\Folyt^1$
    \item $\vfi$ szakaszonként sima $D_f$-beli zárt út
  \end{enumzjr}
  Ekkor
  \[ \di\Int_\vfi f = 0\]
\end{te}

\begin{biz}
  $f=f_1 +if_2\ \nn\ \begin{bmatrix}f_1\\-f_2\end{bmatrix},\,\begin{bmatrix}f_2\\f_1\end{bmatrix}\in\R^2\to\R^2$
  folytonosan deriválhatók.\\
  Cauchy-Riemann egyenletek: $\left.\begin{array}{l}\partial_1f_1 =\partial_2f_2 \\ \partial_2f_1 = -\partial_1f_2
  \end{array}\right\}\nn$ állítás ($\R^n\to\R^n$ eset alapján)\\
  Ugyanis mindkét esetben a deriváltmátrix szimmetrikus.
\end{biz}

\begin{te}
  $l(z) := a + bz\quad(z\in \C)\ (a,b\in\C)$\\
  $\vfi$ szakaszonként sima zárt út
  \[ \di\Int_\vfi l(z)dz = 0\]
\end{te}
\begin{biz} $l$-nek létezik primitív függvénye.
\end{biz}

\begin{te}[Cauchy-féle alaptétel]$f\in\C\to\C$, $f\in\Der$, $D_f$ \underline{egyszeresen öf} tartomány. Ekkor $\forall
  \vfi$ $D_f$-beli szakaszonként sima zárt útra
  \[\di\Int_\vfi f = 0\]
\end{te}

\begin{biz}
  Elég igazolni, hogy $\forall\epsilon>0\colon \left|\di\Int_\vfi\!\! f\right| < \epsilon$\\
  1 lépés: háromszögekre\\
  (i) Trükk: lokalizáláshoz 4 részre osztani (mivel deriválhatóság lokális tulajdonság, csak lokálisan közelíthető). Az
  oldalfelező pontjai mentén osztjuk fel, hogy azokat összekötjük.
  \[\left|\Int_\vfi\!\! f \right| = \left|\sum_{i=1}^4 \Int_{\vfi_i}\!\!f\right| \leqq 4* \max_i
  \left|\,\Int_{\vfi_i}\!\! f\right| =  4  \left|\,\Int_{\vfi_{i_1}}\!\! f\right|\]
  Itt legyen a $\vfi_{i_1}$ az a görbe, amelynél az integrál a maximumot veszi fel. Ezt a görbét osztjuk tovább a
  felezőpontjai mentén, stb, így kapjuk a  $\vfi_{i_2}$-t,  $\vfi_{i_n}$-et.\\
  Egymásba skatulyázott hármoszögek.
   \[\left|\Int_\vfi f \right| \leqq 4^n \left|\Int_{\vfi_{i_n}}f\right|\quad (n\in \N)\]
   Cantor-típusú tétel metrikus terekre, így:
   \[ \exists! a\in D_f\colon \forall n \in \N\ a\in\overline{\intG\vfi_{i_n}}\quad\text{(lezárás)}\]
   (ii) $f\in \derivp{a} \nn f(z) = \underbrace{f(a) + f'(a)(z-a)}_{=:\,l(z)}+\eta(z)(z-a)\qquad \lim_a\eta=0$\\
   (iii) $qlim_a\eta=0\nn \forall \epsilon > 0\  \exists K(a)\colon |\eta(z)|<\epsilon\ (\forall z \in K(a))$.\\
   A háromszögek $a$-ra zsugorodnak $\nn \exists i_n\colon \overline{ \intG\vfi_{i_n}}\subset K(a)$\\
   (iv) tehát lokalizáltuk a problémát: 
   \begin{gather*}\di     
     \left|\Int_\vfi\!f\right| \leqq 4^n\left|\,\Int_{\vfi_{i_n}}\!\!f\right| = 4^n \left|\,\Int_{\vfi_{i_n}}\! l(z)
     +\eta(z) (z-a)\diff z\right| = \\
     = 4^n  \left|\,\Int_{\vfi_{i_n}}\! l(z)\diff z +\Int_{\vfi_{i_n}}\!\eta(z) (z-a)\diff z\right| \leqq  4^n
     \max_{z\in R_{\vfi_{i_n}}} |\eta(z) (z-a)| \Vert\vfi_{i_n}\Vert \leqq \\
     \leqq 4^n \,\epsilon\, \Vert \vfi_{i_n}\Vert^2 = 4^n\, \epsilon
     \left(\Vert\dfrac{ \vfi_{i_n}}{2^n}\Vert\right)^2 = \epsilon\, \Vert\vfi\Vert^2
     \intertext{Ugyanis}
     |z-a| \leqq \dfrac{\vfi_{i_n}}2 < \vfi_{i_n} \text{ háromszög-egyenlőtlenségek sorozata}\\
     \text{a háromszög kerülete lépésenként feleződött: }\Vert\vfi_{i_n}\Vert = \dfrac{\Vert\vfi\Vert}2
   \end{gather*}
   \\2. lépés: sokszögekre\\
   \\3. lépés: tetszőleges $\vfi-re$ - vázlat
   Ötlet: közelítés sokszöggel, közeli pontokban (jó közelítés) a fv-értékek közel lesznek.
   (i) $\vfi$ körül $\ro$ sugarú sáv, jel: $G_\ro(\vfi)$.\\
   $\exists \ro > 0\colon \underbrace{G_\ro(\vfi)}_{\text{kompakt}}\subset D_f$\\
   (ii) $f$ folytonos a kompakt $G_\ro(\vfi)$-n $\overset{\text{Heine}}{\nn}\ f$ egyenletesen folytonos, így\\
   $\forall \epsilon > 0,\  \exists \delta > 0\  \forall z',\,z''\ |z'-z''|<\delta \nn |f(z')-f(z'')| < \epsilon$\\
   (iii) $\vfi$-be alkalmas sokszöget írunk. $z_i$ osztópont, $\psi_i\colon [z_i,z_{i+1}]$ szakasz, $\vfi_i$: a
   $z_i,\,z_{i+1}$ közti ív. Ezek ellenkező irányításúak. $\omega_i :=\vfi_i\cup \psi_i$.\\
   Osztópontok megválasztásától függően a függvényértékek különbsége bármilyen kicsi lehet.
   \begin{gather*}
     \di\Int_\vfi\! f = \Int_\vfi\! f - \Int_\psi\!f = \sum_{i=0}^{n+1}\left(\Int_{\vfi_i}\!f - \Int_{\psi_i}\!f\right)
     = \sum_{i=1}^n+1 \Int_{\omega_i}\!f.\\
     \left|\Int_\vfi\!f\right| = ???
   \end{gather*}
\end{biz}

\subsection{A Cauchy-féle alaptétel következményei}
\subsubsection{Primitív függvény létezése}
\begin{de}$f\in\C\to\C$, az $F\in\C\to\C$ az $f$ egy primitív függvénye, ha
  \begin{enumzjr}
  \item $f\in \Der$
  \item $F'(z)=f(z) \quad \forall z\in D_f$
  \end{enumzjr}
\end{de} 

\begin{te}
  Ha $f\in \C\to\C$, $D_f$ tetszőleges tartomány, az $F$ az $f$ egy primitív függvénye, akkor 
  \begin{enumzjr}
  \item $\forall c\in\C\colon F+c$ is primitív föggvénye
  \item $\forall G$ primitív függvényhez $\exists c\in\C\colon G = F+c$
  \end{enumzjr}
\end{te}

\begin{te}
  $f\in \C\to\C$, $f\in\Der$, $D_f$, tfh \underline{egyszeresen összefüggő}. Ekkor az $f$-nek létezik primitív függvénye.
\end{te}

\begin{te}[Newton-Leibniz]
  $f\in \C\to\C$, tfh
  \begin{enumzjr}
  \item $f\in \Der$, $D_f$ tartomány
  \item $f$-nek $\exists$ primitív függvénye.
  \end{enumzjr}
  Ekkor.  \begin{enumzjr}
  \item $f\in \Der$
  \item $F'(z)=f(z) \quad \forall z\in D_f$
  \end{enumzjr}
  Ekkor $\forall D_f$-beli szakaszonként sima $\vfi$ útra:
  \[\di\Int_\vfi\!f = \Phi(b) - \Phi(a) \]
  ahol a $\Phi$ az $f$ primitív függvénye és $a$ a $\vfi$ kezdőpontja, $b$ a $\vfi$ végpontja
\end{te}

\subsubsection{A Cauchy-féle integrálformula}
Jel: $\vfi_{a,R}$ az $a\in\C$ közepű, $R>$ sugarú, pozitív irányítású körvonal
  \[ \vfi_{a,R}(t) = a + Re^{it}\quad (t\in[0,2\pi])\]

\begin{lemma}
\begin{gather*}
  f(z) := \dfrac1{z-a}\quad z\in \C\setminus\{a\}\\
  \forall R>0\colon \di\Int_{\vfi_{a,R}}\!\!f = 2\pi i \quad R\text{-től függetlenül}
\end{gather*}
\end{lemma}

\begin{lemma} $T\subset \C$ gyűrűszerű tartomány, $f\in\C\to\C$, $f\in\Der(T)$\\
  $\vfi,\psi$ szakaszonként sima, zárt utak, $\subset T$
  EKKOR
  \[ \di\Int_\vfi\!f = \Int_\psi\!f\]
\end{lemma}

\begin{lemma}[Riemann]Ha
  \begin{enumzjr}
    \item $T\subset \C$ tetszőleges tartomány, $a\in T$
    \item $f\in\C\to\C$, $f\in \Der(T\setminus \{a\})$, esetleg $a$-ban is
    \item $\exists M>0\colon |f(z)| \leq M\quad \forall z\in T\setminus\{a\}$
  \end{enumzjr}
  akkor $\forall$ olyan $\vfi$ szakaszonként sima zárt útra, amely a belsejével együtt $T$-ben van:
  \[ \Int_\vfi\!\! f = 0\]
\end{lemma}


\begin{te}[A Cauchy-féle integrálformula]
  $f\in\C\to\C,\ f\in\Der,\ a\in D_f$ás legyen $R>0$, melyre
  $\{z\in\C\colon |z-a|\leq R\}\subset D_f$\\
  Ekkor
  \begin{gather*}
    \forall z_0\in \C\colon |z_0-a|<R\\
    f(z_0) = \dfrac1{2\pi i}\Int_{\vfi_{a,R}} \dfrac{f(t)}{t-z_0} \diff t = \dfrac1{2\pi i}\Int_\vfi
    \dfrac{f(t)}{t-z_0}\diff t
  \end{gather*}
\end{te}
\subsubsection{Holomorf függvény Taylor-sorba fejthető}

\begin{te} $f\in\C\to\C, f\in\Der$, legyen $a\in D_f$: $\exists R>0,\ \{z\in\C : |z-a| \leq R\}\subset D_f$.\\
  Ekkor
  \begin{enumzjr}
    \item $f\in\Der^\infty\{a\}$
    \item $\di f(z)= \sum_{k=0}^\infty \dfrac{f^{(k)}}{k!}(z-a)^k\qquad\forall |z-a|<R$
  \end{enumzjr}  
\end{te}

\begin{kov}
  $f\in\C\to\C,\ f\in\Der,\ D_f$ tartomány, $a\in D_f$, $\vfi$ az $a$-t pozitív irányban megkerülő szakaszonként sima
  zárt út. Ekkor
  \[ f^{(k)} (a) = \dfrac{k!}{2\pi i}\Int_\vfi \dfrac{f(t)}{(t-a)^{k+1}}\diff t\quad k=0,1,2\]
\end{kov}

\subsubsection{Holomor függvény zérushelyei}
\begin{te}[A zérushelyek izoláltak]
  $f\in\C\to\C$
  \begin{enumzjr}
  \item $D_f$ tetszőleges tartomány
  \item $f\in\Der(D_f)$
  \item $f(a)=0,\ a\in D_f$
  \item $f\not\equiv 0$
\end{enumzjr}
EKKOR\\
\[\exists K(a)\colon f(z) \neq 0\qquad \forall z\in K(a)\setminus\{a\}\]
\end{te}

\begin{te}
  $f\in\C\to\C,\ f\in \Der,\ D_f$ tartomány.\\
  Ha $f\not\equiv 0\ \nn$ az $f$ zérushelyei nem torlódhatnak.
\end{te}
\begin{megj}
  Ha a zérushelyek torlódnak $\nn f \equiv 0$
\end{megj}
\begin{te} tfh.
  \begin{enumzjr}
  \item $T\subset \C$ tartomány
  \item $f,g\colon T\to\C,\ f,g\in \Der$
  \item $\exists (z_n), z\in T$\\
    $f(z_n) = g(z_n)\qquad \forall n \in \N$\\
    $z_n\xrightarrow[n\to+\infty]{} z,\ z_i\neq z_j$
  \end{enumzjr}  
  EKKOR
  \[f\equiv g \qquad T\text{-n}\]
\end{te}

\begin{megj}
  Azaz ha két analitikus füöggvény egy torlódó pontsorozat mentén azonos értéket vesz fel, akkor a két függvény
  megegyezik. $\R\to\R$ esetén viszot más!
\end{megj}



\subsubsection{Az algebra alaptétele}

\begin{lemma}[Cauchy-féle egyenlőtlenség]
  $f\in\C\to\C,\ f\in\Der$ és $a\in D_f,\ r>0$\\
  $K_r(a)= \{z\in\C : |z-a| \leq r \} \subset D_f $, EKKOR\\
  $\left|f^{(k)}(a) \right| \leqq \dfrac{k!}{r^k} M_a(r)\quad k=0,1,...$\\
  Ahol $\exists M_a(r) = \max \{\, |f(z)| : |z-a| = r \,\} < \infty$
\end{lemma}

\begin{lemma}[Louville]
  tfh. $f\in\C\to\C$
  \begin{enumzjr}
  \item $D_f = \C$ (lényeges!!)
  \item $f$ analitikus az egész $\C$-n
  \item $\exists M>0\colon |f(z)|\leqq M\ (\forall z\in\C)$
  \end{enumzjr}
  $\nn f$ állandó az egész $C$-n.
\end{lemma}

\begin{te}[Algebra alaptétele] $n=1,2,\ldots;\quad a_0,a_1,\dotsc,a_n-1\in\C$. A
  \[ \di P(z) := z^n + \sum_{k=0}^{n-1}a_k z^k \quad (z\in\C \]
  polinomnak van (legalább egy) zérushelye.
\end{te}


\subsubsection{Maximumtétel}
\begin{te}
  Tfh. $f\in\C\to\C$
  \begin{enumzjr}
  \item $D_f$ tartomány, $f\in\Der(D_f)$
  \item $a\in\intD_f\colon |f(a)| = \max \{\, |f(z)|: z\in D_f \,\}$
  \end{enumzjr}
  Ekkor $f \equiv$ állandó
\end{te}
\begin{megj}
  Azaz ha $f$ nem azonosan állandó, akkor belső pontban nem veheti fel a maximumát.
\end{megj}
\begin{kov}
  $f\colon T\to \C,\ T \subset \C,\ T$ tartomány, $f\in\Der(T)$ és $f\in \Folyt(\overline{T})$ (itt $\overline(T)$
  kompakt!)\\
  EKKOR $|f$ a maximumát a $T$ határán veszi fel, és ha $f\not\equiv$ áll, akkor a maximumot csakis a határon veheti fel.
\end{kov}

\subsection{A Cauchy-alaptétel megfordítása}
\begin{te}[Morena-tétel]
  Legyen $f\in\C\to\C$ és tfh
  \begin{enumzjr}
  \item $D_f$ tetszőleges tartomány
  \item $f\in C(D_f)$
  \item $\forall \vfi$egyszerű, zárt útra, melyre: $\int_\vfi\subset D_f$
    \[\di\Int_\vfi\! f = 0 \]
  \end{enumzjr}
  Ekkor $f$ homomorf a $D_f$-en.
\end{te}


\subsection{Egyenletesen konvergens függvénysorozatok}
\begin{te}[Weierstrass tétele föügvvénysorozatokra]
  $T\subset \C$ tetszőleges tartomány, $f_n\colon T\to\C\ (n\in\N)$, $f_n\in \Der(T)$.
  Tfh. $f_n$ egyenletesen konvergens $T-n$: $f_n \underset{n\to+\infty}{\hookrightarrow}f$.
  Ekkor 
  \begin{enumzjr}
  \item $f\in\Der(T)$    
  \item $\forall k\in\N\colon f_n^{(k)} \underset{n\to+\infty}{\hookrightarrow} f^{(k)}$ a  $T$-n
  \end{enumzjr}
\end{te}

\begin{kov}[Wierstrass tétele függvénsorokra]
  $T\subset \C$ tetszőleges tartomány, $f_n\colon T\to\C\ (n\in\N)$, $f_n\in \Der(T)$.
  Tfh. $\Sigma f_n$ egyneletesen konvergens $T-n$: $f_n \underset{n\to+\infty}{\hookrightarrow}f$.
  Ekkor 
  \begin{enumzjr}
  \item $f\in\Der(T)$    
  \item $\forall k\in\N\colon f_n^{(k)} \underset{n\to+\infty}{\hookrightarrow} f^{(k)}$ a  $T$-n
  \end{enumzjr}  
\end{kov}


\subsection{Laurent-sor}
\begin{te}
  Tfh. $f\in \C\to\C$,
  \begin{enumzjr}
  \item $0 \leq r < R < +\infty\colon\   T := \{\, z\in\C :r < |z-a| < R \,\}$
  \item $f\in\Der(T)$    
  \end{enumzjr}
  EKKOR
  \begin{gather*}
    \exists c_k\in \C,\ k\in \Z
    f(z) = \sum_{k=-\infty}^{+\infty}(z-a)^k
    \intertext{ahol}
    c_k=\dfrac{1}{2\pi i}\Int_\gamma\dfrac{f(t)}{(t-a)^{k+1}}\diff t\quad \forall k\in\Z
  \end{gather*}
  Ahol a $\gamma$ egyszerű, zárt görbe (poz. forgásirányú), $\gamma\in T$, az $a$-t megkerüli
\end{te}

\subsection{Izolált szingularitások}
\begin{de}
  Az $f\in \C\to\C$ fv-nek az $a\in\C$ izoltált szingularitása, ha $\exists \R>0\colon f$ analitikus a$K_R(a)\setminus\{a\}$-ban
\end{de}

\subsubsection{A szinguláris helyek osztályozása}
\begin{enumerate}
  \item \emph{1. eset:} (az $a\in\C$ megszüntethető, izolált szingularitás). Nincs negatív indeű tag:
    \begin{gather*}
      \forall c_k = 0\quad k=-1,-2,\ldots
      f(z) = \sum_{k=0}^\infty c_k(z-a)^k\quad 0<|z-a|<R
    \end{gather*}
    \begin{te}
      $f$-nek az $a$-ban megsz9ntethető izolált szingularitása van.\\
      Megszüntethető $\iff \exists K_{r_1}(a)\colon\\f$ korlátos $K_{r_1}(a)-ban$
    \end{te}
  \item \emph{2. eset:} (az $a$ $n$-edrendű pólus)\\
    Ha a Laurent-sorban $\exists n=1,2,\ldots\colon c_{-n} \neq 0$ és $\forall c_{-(n+1)}=c_{-(n+2)}=\dotsb=0$
    \begin{gather*}
      f(z) = \dfrac{c_{-n}}{(z-a)^n} + \dotsb + c_k + c_1(z-a) + \dotsb =\\= \dfrac1{(z-a)^n} \underbrace{(c_{-n} +
      c_{-n+1}(z-a)+\dotsb )}_{{ }=:\ g(z)}= \dfrac{g(z)}{(z-a)^n}
    \end{gather*}
    Itt $g(z)$ analitikus a $K_R(a)$-n és $g(a) \neq 0$
    \begin{te}
      $f\in\C\to\C$, $a$ izolált szingularitás. EKKOR\\
      az $a$ $n$-edrendű pólus $\iff\di f(z) = \dfrac{g(z)}{(z-a)^n},\quad 0<|z-a|<R$ \\
      ahoil $g$ analitikus $K_R(a)$-n és $g(a) \neq 0$
    \end{te}
  \item \emph{3. eset:} (lényeges szingularitás)\\
    Ha a Laurent-sorban végtelen sok nullától különböző negatív indexű tag van.\\
    Megj: az $a$ közelében ``megvadul'' a függvény.\\
    \begin{de}
      A $H\subset \C$ halmaz mindenütt sűrű a $\C$-ben, ha $\overline{H}=\C$ azaz
      \[\forall w \in \C,\ \forall \epsilon >0, \exists Z_0\in H\colon |w-z_0| < \epsilon\]
    \end{de}

    \begin{te}[Casatori-Weierstrass]
      Ha az $a\in\C$ az $f$ lényeges szingularitása, AKKOR\
      \[ \forall 0< r_1 < R\colon \{\, f(z) :  0 < |z-a| < r_1\,\} \]
      halmaz mindenütt sűrű a $C$-ben.      
    \end{te}
    \begin{te}[Picard]
      HA az $a\in\C$ az $f$ függvénynek lényeges szingularitása, akkor az $a\in\C$ pont $\forall$ környezetéven $f$
      $\forall \C$-beli értéket fülvesz, legfeljebb egy pont kivételével.     
    \end{te}

    Itt $g(z)$ analitikus a $K_R(a)$-n és $g(a) \neq 0$
    \begin{te}
      $f\in\C\to\C$, $a$ izolált szingularitás. EKKOR\\
      az $a$ $n$-edrendű pólus $\iff\di f(z) = \dfrac{g(z)}{(z-a)^n},\quad 0<|z-a|<R$ \\
      ahol $g$ analitikus $K_R(a)$-n és $g(a) \neq 0$
    \end{te}

\end{enumerate}


% Local Variables:
% fill-column: 120
% TeX-master: t
% End:

%  \newpage
%  \section{Differenciálszámítás}

\begin{megj} Csak $\RnRm$ típusú fv-ek; $n,m\in\N$\end{megj}

\subsection{$\RnRm$ típusú leképezések}
\begin{de}
  $L: \RnRm$ lineáris, ha
{\listazjromai
\begin{enumerate}
  \item $L(x + y) = L(x) + L(y)\qquad \forall x,y\in \R^n$ (additivitás)
  \item $L(\lambda x) = \lambda L(x)\qquad \forall x\in\R^n,\ \lambda\in\R$ (homogenitás)  
\end{enumerate}
}
\end{de}

\begin{Megj}
\item Ha $L$ lineáris, akkor $\di L\left(\sum_{i=1}^s \lambda_i y_i\right) = \sum_{i=1}^s\lambda_i L(y_i)$
\item $L\in\R\to\R$ lineáris $\ekviv \exists!c\in \R\ L(x)\quad  x\in\R$\\
  $L$ és $c$ azonosítható egymással.
\item  Jelölés: $\Linearis :=  \{\,L\in R^n\to\R^m\,|\,L\text{ lineáris}\,\}$
\item  Szokásos műveletek: $+$, $\lambda\cdot$ pontonként.
\end{Megj}

\begin{te}
  \Linearis a szokásos műveletekkel LT az $\R$ felett
\end{te}

\subsubsection[Lineáris leképezés mátrix reprezentációja]{Lineáris leképezés mátrix reprezentációja (adott bázisban)}
$\R^n$-ben $e_1,\,e_2,\dotsc,\,e_n$ egy bázis (pl a kanonikus bázis: $e_i=(0,\dotsc,0, \overset{i}{\breve{1}},
0,\dotsc,0)$)\\
$\R^m$-ben $f_1,\,f_2,\dotsc,\,f_n$ egy bázis (pl a kanonikus bázis)\\
$L\in\Linearis,\ x\in \R^n,\di x = \sum_{j=1}^n x_j e_j\quad x_j$ az $x$ vektor $e$ bázisra vonatkozó
koordinátája
\begin{gather*}
  \di  L(x) = L\left(\sum_{j=1}^n x_j e_j\right) = \sum_{j=1}^n x_j L(e_j)\\
  \R^m \owns L(e_j) = \sum_{i=1}^m a_{ij} f_i\qquad\quad\text{$a_{ij}$ az $L(e_j)$ $i$. koordinátája az
  $(f_1,\dotsc,f_n)$ bázisban}\\
  A = \begin{pmatrix}
    a_{11} & a_{12} & \dots & a_{1n}\\
    a_{21} & a_{22} & \dots & a_{2n}\\
    \vdots & \vdots & \ddots & \vdots\\
    a_{m1} & a_{m2} & \dots & a_{mn}
  \end{pmatrix}\in \R^{m\times n}\\
  L(x) = \sum_{j=1}^n x_j L(e_j) = \sum_{j=1}^n x_j \left(\sum_{i=1}^m a_{ij} f_i\right) = \sum_{i=1}^n
  \left(\sum_{j=1}^n a_{ij} x_j\right) f_i\\
  \text{A } \sum_{j=1}^n a_{ij} x_j \text{ mátrixszorzással felírva:}\\
  \begin{pmatrix}
    a_{11} & \dots & a_{1n}\\
    \vdots & \ddots & \vdots\\
    \mathbf{a_{i1}} &  \dots & \mathbf{a_{in}}\\ 
    \vdots & \ddots & \vdots\\
    a_{m1}  & \dots & a_{mn}
  \end{pmatrix}
  \begin{pmatrix}x_1\\x_2\\\vdots\\x_n\end{pmatrix} = (Ax)_i
\end{gather*}

Tehát egyértelműen megfeleltethető e kettő egymásnak:\\
$L\in \Linearis\leftrightarrow A\in \R^{m\times n}$

\begin{te}
  Az $\Linearis$ LT izomorf  az $\R^{m\times n}$ LT-rel, azaz: 
  \begin{gather*}
    \exists \varphi: \Linearis\n\R^{m\times n} \text{ bijekció, melyre}\\
    \varphi(\lambda_1 L_1+\lambda_2 L_2) = \lambda_1\varphi(L_1) + \lambda_2 \varphi(L_2)
    \qquad \forall L_1,\,L_2\in\Linearis;\ \forall \lambda_1,\,\lambda_2\in\R
  \end{gather*}
\end{te}

\begin{megj}
  Az $\RnRm$ leképezés azonosítható egy $A\ m\times n$-es mátrix-szal (egy adott bázisban).
\end{megj}

\subsubsection{Norma értelmezése az $\Linearis$ LT-n}

\begin{te}
  \begin{enumerate}
    \item Ha $L\in\Linearis$, akkor 
      \[\di \opnorma {L} := \sup_{h\in\R^n}\{\,\underbrace{\norman{L(h)}{1}}_{\text{$\R^m$-beli}} : h\in\R^n,
      \underbrace{\norman{h}{2}}_{\text{$\R^n$-beli}}<1\} < \infty\]
      ún. \emph{operátornorma} norma az \Linearis\ téren.
      \item Ha $L\in\Linearis$, akkor
	\[ \norman{L(h)}1 \leqq \opnorma{L}\cdot\norman{h}2\qquad\forall h\in\R^n 
	\]
  \end{enumerate}
\end{te}

\begin{biz}?\end{biz}

\begin{megj}
  $f\in\R\n\R,\  f\in\D\{a\},\ a\in\intD_f$
  \begin{gather*}\di
    f\in\D\{a\} \ \ekviv\  \exists \lim_{h\n0} \dfrac {f(a+h)-f(a)}{h} < \infty \overset{\text{lineáris}}{\underset
      {\text{közelíés}}{\ekviv}}\\\ekviv \exists A\in \R \text{ és } \exists \epsilon \in \R\n\R,\, \lim_0\epsilon=0
    \colon f(a+h) -f(a) = Ah + \epsilon(h)\cdot h \ \ekviv\\
    \lim_{h\n0}\dfrac{|f(a+h)-f(a)-L(h)|}{|h|} = 0
  \end{gather*}
\end{megj}



\subsection{Totális deriválhatóság}

\begin{de}[Totális deriválhatóság]
  $f\in\RnRm,\ a\in\intD_f$. Az $f$ fv totálisan deriválható az $a\in\intD_f$ pontban, ha
  \[\exists L\in\Linearis\colon\quad \lim_{h\n\nullelem}\dfrac{\norman{f(a+h)-f(a)-L(h)}1}{\norman{h}2} = 0,\]
  ahol \Norman1\ egy $R^m$-beli tetszőleges, \Norman2 egy $\R^n$-beli tetszőleges norma.\\
  Az $f$ fv $a$-beli deriváltja: $f'(a) := L(a)$\\
  Jel: $f\in\D\{a\}$
\end{de}

\begin{te}
  Ha $f\in\derivp{a},\ f\in\RnRm$, akkor $f'(a)$ egyértelmű.
\end{te}

\begin{biz}
  Tegyük fel, hogy $\exists L,R\in\Linearis$, amire a definíció teljesül, továbbá 
  \begin{gather*}
     L-R=:S\in\Linearis\\
     \dfrac{\norman{S(h)}1}{\norman{h}2} = \dfrac{\norman{L(h)-R(h)}1}{\norman{h}2} \leqq\\
     \leqq\dfrac{\norman{f(a+h)-f(a)-L(h)}1}{\norman{h}2} + \dfrac{\norman{f(a+h)-f(a)-R(h)}1}{\norman{h}2} \xrightarrow
     [h\n\nullelem]{}0\\
     \nn \lim_{h\n\nullelem} \dfrac{\norman{S(h)}1}{\norman{h}2} = 0.\\\text{Legyen spec: }h:=\lambda e,\
     e\in\R^n,\ \norman{e}2 = 1,\ \lambda \n 0\, \nn\, h\n\nullelem. \\
     \dfrac{\norman{S(h)}1}{\norman{h}2} = \dfrac{\norman{S(\lambda e)}1}{\norman{\lambda e }2} =
     \dfrac{|\lambda|\,\norman{S(e)}1}{|\lambda|\,\norman{e}2}\, \nn\, S(e) = \nullelem \,\nn\, S \equiv 0 \,\nn\, L
     \equiv R
  \end{gather*}
\end{biz}

\begin{te}
  A deriválhatóság ténye és a derivált független attól, hogy $\R^n$-ben és $\R^m$-ben melyik normát választjuk.
\end{te}
\begin{biz}
  $\R^n$-ben a normák ekvivalensek.
\end{biz}

\begin{te}[Ekvivalens átfogalmazások]\ 
  \begin{enumerate}
    \item $\di f\in\der{a} \ekviv  \exists A\in\Rmn\colon  \lim_{h\n\nullelem} \dfrac{\norman{f(a+h) -f(a) -Ah}1}{
    \norman{h}2} = 0$\\
      Az $A$ az ún. \emph{deriváltmátrix}.
    \item $f\in\der{a}  \ekviv \left\{\begin{array}{l}\exists A\in\Rmn \text{  és } \di\exists \epsilon \in\RnRm\
    \lim_0\epsilon=\nullelem :\\f(a+h) -f(a)= Ah + \epsilon(h)\,\norman{h}2\qquad a,a+h\in \D_f\end{array}\right.$\\
      (lineáris fv-nyel való jó közelítés)
  \end{enumerate}
\end{te}
\begin{biz}Trivi\end{biz}
  

\textbf{Spec esetek:}
\begin{enumerate}
  \item $f\in\RnRm;\ f'(a)$ egy $m\times n$-es mátrix ($\in\Rmn)$
  \item $f\in\RnR;\ f'(a)$ egy sorvektor  ($\in R^{1\times n})$
  \item $f\in\RRm;\ f'(a)$ egy oszlopvektor ($\in R^{m\times1})$   
\end{enumerate}


\begin{Pl}
\item $f(x):=c\quad(x\in\R^n);\ c\in\R^m$ rögzített $\nn \forall a\in\R^n:\ f\in\der{a};\ f'(a)=0\in\Rmn$
\item $f=L\in\Linearis,\ \forall a\in\R^n\colon L\in\der a$ és $L'=L$
\end{Pl}

\begin{te}[Folytonosság-deriválhatóság]
  $f\in\RnRm$, $a\in\intD_f$
  \begin{enumerate}
    \item Ha $f\in\der a\nn f\in\folyt a$
    \item Visszafelé \emph{nem} igaz
  \end{enumerate}
\end{te}
\begin{biz}
  \[\nn: f(a+h) -f(a) = Ah + \epsilon(h)\,\norman{h}2\xrightarrow[h\n\nullelem]{}0 \nn f\in\folyt{a}\]
  \[\not\Leftarrow: n=m=1 \text{ és pl } f := \mathrm{abs}\]
\end{biz}

\begin{te}
  $f\in\RnRm, f=\begin{bmatrix}f_1\\\vdots\\f_m\end{bmatrix};\ f_i\in\RnR\ (i=1,\dotsc,m)$ az $f$
  koordinátafüggvényei\\
Ekkor
\[f\in\der{a} \ekviv \forall i = 1,2,\dotsc,m\text{ és } f_i\in\der{a} \text{ és } f'(a)= \begin{bmatrix}f_1'(a) \\
  f_2'(a) \\
  \vdots\\ f_m'(a)\end{bmatrix} = \begin{bmatrix}\hspace{0.8em}\dots\hspace{0.8em}\\\dots\\\vdots\\\dots
\end{bmatrix}\in\Rmn\]
\end{te}
\begin{biz}Trivi
\end{biz}

\begin{megj} $f\in\RnRm\ \nn\ \RnR$ elég vizsgálni.
\end{megj}
\subsubsection{Műveletek}
\begin{te}$f,g\in\RnRm,\ a\in(\intD_f)\cap(\intD_g),\ f,g\in\der a$\\
  $\begin{array}{@{\quad}rcl}
    1^\circ &  f+g\in\der a & (f+g)'(a) = f'(a) + g'(a)\\
    2^\circ &  \lambda g\in\der a\quad \forall \lambda\in\R & (\lambda f)'(a) = \lambda f'(a)
  \end{array}$
\end{te}
\begin{biz}
  trivi
\end{biz}

\begin{te}[Kompozíció, láncszabály]
  $g\in\RnRm,\ a\in\intD_g, g\in\der a$;\\ $f\in\R^m\n\R^r,\ R_g\subset D_f,\ f\in \der{g(a)}$.\\
  Ekkor $f\circ g\in\R^n\n\R^r$ deriválható az $a$-ban és $(f\circ g)'(a) = f'(g(a) \circ f'(a)$
\end{te}
\begin{biz} nem kell (ld. $\R\n\R$ eset)
\end{biz}

\begin{megj}\ \\
  \begin{tabular}{r@{$\,\in\,$}l@{$\ \nn\ $}l}
    $f\circ g$ & $\R^n\n\R^r$ &  $C:=(f\circ g)'(a)\in\R^{r\times n}$\\
    $g$ & $\R^n\n\R^m$ &  $A:=g'(a)\in\R^{m\times n}$\\
    $f$ & $\R^m\n\R^r$ &  $B:=f'(a)\in\R^{r\times m}$\\
  \end{tabular}$ C = BA$\\
Lineáris leképezések kompozíciója $\equiv$ mátrixreprezentációk szorzata
\end{megj}


\subsection{Parciális deriválhatóság}

\begin{de}[Parciális derivált]
  $f\in\RnRm,\ a\in\intD_f,\ e_1,e_2,\dotsc,e_n\in\R^n$ kanonikus bázis, azaz $e_i = (0,\dotsc,0,
  \overset{i}{\breve{1}}, 0,\dotsc,)0 $\\
  \emph{Az $f$-nek $\exists$ az $i$. változó szerinti parciális deriváltja az $a\in\intD_f$ pontban}, ha az\\
  $F: K(0)\owns t \mapsto f(a+t e_i)\quad (F: \R\n\R^m)$\\
  fv deriválható a $0$ pontban.\\
  Az $F'(0)$ oszlopvektor az $f$ $i$. változó szerinti parciális deriváltja az $a$-ban.\\
  Jel: $\partial_i f(a) := F'(0);\quad \dfrac {\partial f}{\partial {x_i}}(a)$ 
\end{de}

\begin{Pl}
  \item $f(x,\,y) = x^3 y^2\quad (x,y)\in\R^2$\\
    $\di\partial_1 f(x_0,y_0) = (x\n f(x,y_0)'_{x=x_0} = 3{x_0}^2{y_0}^2$\\
    $\di\partial_2 f(x_0,y_0) = (y\n f(x_0,y)'_{y=y_0} = 2{x_0}^3y_0$
  \item $f(x,y,z) = x^3 y^2 z$\\
    $\partial_2 f(x_0,\,y_0,\,z_0)= 2{x_0}^3y_0z_0$
\end{Pl}


\begin{te}
  $f\in\RnRm,\ a\in\intD_f$\\
  Ha $f\in\der{a}$, akkor $\forall i=1,2,\dotsc,n\colon  \partial_if(a)$ létezik
\end{te}

\begin{biz}
  \begin{gather*}
  F(t) := f(a+te_i),\qquad F:=f\circ g,\qquad g(t)=a+te_i.\\
  g\in\der 0,\ g'(a)=e_i\nn F=f\circ g\in\der 0\nn \exists.\\
  F'(0) = \partial_i f(a) = f'(a)g'(a) = f'(a)\cdot(e_1) = \underbrace{f'(a)}_{\text{mátrix}} \cdot
  \underbrace{e_1}_{\text{vektor}}
  \end{gather*}
\end{biz}

\begin{te}[A deriváltmátrix előállítása]
  $f\in\RnRm,\ a\in\intD_f,\\ f=\begin{bmatrix}f_1\\\vdots\\f_n\end{bmatrix};\ f_i\in\RnR\ (i=1,\dotsc,m)$\\
  \vspace{.1em}
  Ha $f\in\der a\nn\\$
  \[\Rmn\owns f'(a) = \begin{bmatrix}
    \partial_1 f_1(a) & \partial_2 f_1(a) & \dots & \partial_n f_1(a)\\
    \partial_1 f_2(a) & \partial_2 f_2(a) & \dots & \partial_n f_2(a)\\
    \vdots & \vdots & \ddots & \vdots \\
    \partial_1 f_1(a) & \partial_2 f_1(a) & \dots & \partial_n f_1(a)
    \end{bmatrix}\]
  deriváltmátrix vagy \emph{Jacobi-mátrix}
\end{te}

\begin{biz}
    $f\in\der a \nn \exists f'(a) = A = \left[ a_{ij}\right] \in \Rmn$
    \[\di\lim_{h\n\nullelem} \dfrac{\norman{f(a+h)-f(a) -Ah}1}{\norman h2}=0\]
    $\nn \forall i = 1,2,\dotsc,m \text{ esetén}$
    \[\di\lim_{h\n\nullelem} \dfrac{\left|f_i(a+h)- f_i(a) - \sum\limits_{k=1}^n a_{ik} h_k\right|}{\norman h2}=0\]
    Legyen spec: $h := te_j\quad (t\in\R)\quad (e_j = (0,\dotsc,0,\overset{j}{\breve{1}},0,\dotsc,0))\quad h\n\nullelem
    \ekviv t\n0$

    \[\di\nn \lim_{t\n0}\dfrac{\left|f_i(a+te_j)-f_i(a) - a_{ij}t\right|}{|t|\,\norman{e_j}2} = 0\]
    $\overset{\text{parc.der.}}{\nn} a_{ij} = \partial_j f_i(a)$
\end{biz}

\begin{Megj}
\item \underline{Spec. esetek}\\
\begin{enumerate}
  \item $f\in\RnR,\ f\in\der a\\ f'(a) = \begin{bmatrix}\partial_1 f(a) & \partial_2 f(a) & \dots & \partial_n
    f(a)\end{bmatrix}$
    az $f$ gradiens vektora, $\grad f(a)$
  \item $f\in\RRm\quad f=\begin{pmatrix}f_1\\\vdots\\f_m\end{pmatrix}\quad f_i\in\R\n\R$\\
    $f\in\der a\nn f'(a) = \begin{bmatrix}f_1'(a)\\\vdots\\f_m'(a)\end{bmatrix}$
\end{enumerate}
\item Totális és parciális derivált kapcsolata\\
  Tudjuk: $f\in\der a \nn \forall $ parciális derivált létezik.\\
  Várható: visszafele NEM igaz.\\
  Pl: $f(x,y) := \sqrt{|xy|}\quad(x,y)\in\R^2$.\\
  Ekkor $\partial_1 f(0,0) = 0 = \partial_2 f(0,0)$, de $f\not\in\der{(0,0)}$\\
  Hf, gyak
\item Meglepő: elégséges feltétel adható
\item $f\in\RnRm\qquad f=\begin{bmatrix}f_1\\\vdots\\f_m\end{bmatrix}$\\
  $f\in\der a \ekviv \forall i=1,\dotsc,m\ f_i\in\der a$\\
  azaz: elég az $m=1$ esetet vizsgálni  
\item Jelölés:\\
  $\varphi\in\RnR\\
  f\in\RnRm$
\end{Megj}

\begin{te}[Elégséges feltétel a deriválhatóságra]
  Legyen $\varphi\in\RnR,\ a\in\intD_\varphi,\\\varphi\in K(a)\n\R$.
  Tfh. $\forall i=1,\dotsc,n$-re
  \begin{enumerate}
    \item a $\partial_i\vfi$ parciális deriváltak léteznek $\forall x\in K(a)$-ra.
    \item $\partial_i\vfi(x)\colon K(a)\n\R,\ x\n\partial_i \vfi(x)$ parciális derivált függvények folytonosak az
    $a$-ban: $\partial_i \vfi\in\folyt a$      
  \end{enumerate}
Ekkor  $\vfi\in\der a$ (totálisan deriválható)
\end{te}

\begin{biz} $n=2$-re ($n>2$-re hasonlóan):
  \begin {gather*}
    \di\vfi\colon \R^2\n\R,\ a=(a_1,a_2),\ h=(h_1,h_2)\\
    \vfi\in\der a \overset{\text{def}}{\ekviv} \begin{array}{l}
      \exists A,B\in\R,\ \exists \epsilon\in\R^2\n\R,\ \di\lim_\nullelem\epsilon=0\text{, melyre:}\\
      \lim\limits_{h\n\nullelem}\dfrac{\left|\vfi(a+h)-\vfi(a)-(A,\,B)\begin{pmatrix}h_1\\h_2\end{pmatrix}\right|}{\norma h}=0
    \end{array}\\
    \vfi(a+h) - \vfi(a) = \vfi(a_1+h_1,\,a_2+h_2) - \vfi(a_1,\,a_2) =\\
    = \vfi(a_1+h_1,\,a_2+h_2) - \vfi(a_1+h_1,\,a_2) +\vfi(a_1+h_1,\,a_2) - \vfi(a_1,\,a_2) = \star
  \end{gather*}
  A valós-valós Lagrange-középértéktételt felhasználva legyen: $\nu_1\in(0,1)$; $a_2$ rögzített:
  \[\vfi(a_1+h_1,\,a_2) - \vfi(a_1,\,a_2) = \partial_1\vfi(a_1+\nu_1h_1,\,a_2)\cdot h_1\]
  Hasonlóan legyen $\nu_2\in(0,1)$; $a_1+h_1$ rögzített:
  \[\vfi(a_1+h_1,\, a_2+h_2) - \vfi (a_1+h_1, a_2) = \partial_2\vfi(a_1+h_1,a_2+\nu h_2)\cdot h_2\]
  Behelyettesítve:
  \[ \star=\partial_1\vfi(a_2+\nu_1h_1,a_2)\cdot h_1 + \partial_2\vfi(a_1+h_1,a_2+\nu_2 h_2)\cdot h_2 = \sharp\]
  De!
  \begin{gather*}\begin{array}{l}\partial_1\vfi \in\folyt{(a_1,a_2)}\\
      \partial_2\vfi \in\folyt{(a_1,a_2)}\end{array} \nn 
    \partial_1(a_1+\nu_1 h_1, a_2) = \partial_1\vfi(a_1,a_2)+\epsilon_1(h)\\
    \text{ahol a folytonosság miatt} \lim_{h\n\nullelem}\epsilon_1(h)=0 \text{, illetve:}\\
    \partial_2(a_1+h_1, a_2+\nu_2 h_2) = \partial_2\vfi(a_1,a_2)+\epsilon_2(h),\qquad\lim_{h\n\nullelem}\epsilon_2(h)=0    
  \end{gather*}
  Így:
  \begin{gather*}\sharp = \vfi(a+h)-\vfi(a) = \big[\underbrace{\partial_1\vfi(a_1,\,a_2)}_{A}+\epsilon_1(h)\big]h_1 + 
    \big[\underbrace{\partial_2\vfi(a_1,\,a_2)}_{B}+\epsilon_2(h)\big]h_2.\\
    \dfrac{\left|\vfi(a+h) - \vfi(a) - \begin{pmatrix}A & B\end{pmatrix} \begin{pmatrix}h_1\\h_2\end{pmatrix}\right|}
    {\norma h} = \dfrac{\left|\epsilon_1(h)h_1 + \epsilon_2(h)h_2\right|}{\norma h} \leq |\epsilon_1(h) + \epsilon_2(h)| 
    \xrightarrow[h\n\nullelem]{} 0 \\\nn \vfi\in\der{a}
  \end{gather*}
\end{biz}

\subsection{Iránymenti deriválhatóság}

\begin{de}[$e$ irány szerinti iránymenti derivált]
  $f\in\RnRm,\\a\in\intD_f,\ e\in\R^n $ egyésgvektor $(\norma{e}_2 = 1)$.\\
  Az $f$ fv-nek az $a\in\intD_f$ pontban az $e$  irány szerinti iránymenti deriváltja létezik, ha az
  \[F: K(a)\owns t \mapsto f(a+te) \in \R^n\]
  fv a $0$ pontban deriválható. Az $F'(0)\in\R^n$ az $e$ irányban vett iránymenti deriváltja $a$-ban.\\
  Jel: $\partial_ef(a) := F'(0)$
\end{de}

\begin{te}[Az iránymenti derivált kiszámolása]
  Ha $f\in\RnRm,\ a\in\intD_f$,\\
  Ekkor $\forall e\in\R^n$ egységvektor $(\norma{e}_2 = 1)$ esetén $\exists\partial_ef(a)$ és
\[\partial_ef(a) = f'(a)\cdot e\]
\end{te}
\begin{megj} $\partial_ef(a)\in\R^m;\ f'(a) \in\RnRm;\ e\in\R^n$ - mátrixszorzás
\end{megj}
\underline{Spec. esetek:}
\begin{enumerate}
  \item Iránymenti derivált: parciális derivált általánosítása. Ha $e=e_i$, akkor $\partial_ef(a) = \partial_if(a)$
  \item $m=1\colon \ \vfi\colon\RnR^1; e\in\R^n; \norma{e}_2=\di\sqrt{\sum_{k=1}^n{e_k}^n} = 1$
    \[\di \partial_e\vfi(a) = \skalar{\grad f(a)}{e} = \sum_{k=1}^n \partial_kf(a)e_k\]
\end{enumerate}
\begin{Megj}
\item Lényeges: $\norma{e}_2=1$
\item Totális derivált és iránymenti derivált kapcsolata:\\
  Tudjuk: $f\in\der{a}\nn\ \forall$ irányban deriválható\\
  Visszafele NEM igaz!
\end{Megj}

\subsection{Középértéktétel}
\begin{te}[Lagrange-középértéktétel]
  $n\geq1,~U\subset\R^n$ nyílt és $\forall a\in U,\,a+h\in U$.\\
  Szakasz: $[a,\,a+h] := \{\,a+th:t\in(0,1)\,\}\subset U$
  {\listazjbetu
    \begin{enumerate}
    \item Ha $\vfi\colon U\n\R,\ \vfi\in\D(U)$: $U$ minden pontjában deriválható\\
      akkor $\exists \nu\in(0,1):\vfi(a+h)-\vfi(a)=\vfi'(a+\nu h)\cdot h=\skalar{\grad\vfi(a+\nu h)}{h}$
    \item Ha $f\colon U\n\R^m,\ m\geq2,\ f\in\der{x}\ (x\in U)$, akkor
      \[ \norma{f(a+h)-f(a)}_\infty\leq \di\sup_{0\leq\nu\leq1} f'(a+\nu h)(h) \leq
      \sup_{0\leq\nu\leq1}\opnorma{f'(a+\nu h)}\cdot \norma{h}_\infty\]
    \end{enumerate}  
  }
\end{te}
\begin{biz} nem kell
\end{biz}

\subsection{Többször deriválható függvények}
\begin{de}
  $\vfi\in\RnR,\ a\in\intD_f$. A $\vfi$ kétszer deriválható az $a\in\intD_f$, ha
  {\listazjromai
    \begin{enumerate}
    \item $\exists K(a)\colon \forall x\in K(a)$-ban $\vfi$ deriválható
    \item $\forall i=1,\dotsc,n\ \partial_i\vfi$ parciális függvények deriválhatóak $a$-ban: $\partial_i\in\der{a}$
    \end{enumerate}
  }
  Jel: $\vfi \in\dern2a$
\end{de}
\begin{megj}
  $(i) \ekviv \exists\vfi'=(\partial_1\vfi,\dotsc,\partial_n\vfi)\in\R^n\n\R^n$ függvény\\
  $(ii)\ \partial_i\vfi\in\der{a}\ (i=1,2,\dotsc,n) \ekviv \vfi'\in\der{a}$\\\\
  $\vfi'' = (\vfi')'$, de $\vfi'\in\R^n\n\R^n,\ \vfi''\in\R^n\n\R^{n\times n}$.\\
  \[\vfi(a)'' = \begin{bmatrix}
  \partial_1\partial_1\vfi &\partial_2\partial_1\vfi & \cdots & \partial_n\partial_1\vfi \\
  \vdots & \vdots & \ddots & \vdots \\
  \partial_1\partial_n\vfi &\partial_2\partial_n\vfi & \cdots & \partial_n\partial_n\vfi 
  \end{bmatrix}\in\R^{n\times n}\]
  az ún. \emph{Hesse-féle mátrix}  
\end{megj}

\begin{de}
  $\vfi \in\RnR,\ a\in\intD_\vfi$. \\A $\vfi$ $s$-szer $(s\geq2)$ deriválható az $a$-ban $(\vfi \in\dern n a)$, ha
  {\listazjromai
    \begin{enumerate}
    \item $\exists K(a)\subset D_\vfi$, hogy $\vfi$ (s-1)-szer deriválható a $K(a)\ \forall$ pontjában.
    \item Az összes $\partial_{i_1}\partial_{i_2}\ldots\partial_{i_{s-1}}\vfi\quad 1\leq i_1, i_2,\dotsc,i_{s-1} \leq n$
      \quad (s-1)-edrendű parciális derivált függvény deriválható az $a$-ban.
    \end{enumerate}
  }
\end{de}
\begin{te}[Young-tétel]$\vfi\in\RnR,\ a\in\intD_\vfi$
  \[\vfi \in\dern2a\nn\forall i,j=1..n\quad \partial_j(\partial_i\vfi) = \partial_i(\partial_j\vfi) \]
\end{te}
\begin{te}\label{te:youngkov}(Következmény!!) $\vfi\in\RnR,\ a\in\intD_\vfi, \vfi\in\dern s a\ (s\geq2)$
  \[ (\partial_{i_1}\partial_{i_2}\ldots\partial_{i_s}\vfi)(a) = 
  (\partial_{\sigma_1}\partial_{\sigma_2}\ldots\partial_{\sigma_s}\vfi)(a)\]
  ahol $1\leq i_1, i_2,\dotsc,i_s \leq n$ és a $\sigma_1, \sigma_2,\dotsc,\sigma_s$ az $i_1, i_2,\dotsc,i_s$ egy
  permutációja  
\end{te}

\subsubsection{Taylor-formula}
\begin{te}(Emlékeztető)
  $f\in\R\n\R;\ m\in\N,\ f\in\D^{m+1}\left(K\left(a\right)\right);\ a,a+h\in\D_f;\\\exists \nu\in(0,1):$
  \[\di f(a+h) = f(a) + \sum_{k=1}^m\dfrac{f^{(h)}(a)}{k!}h^k + \dfrac{f^{(m+1)}(a+\nu h)}{(m+1)!}h^{m+1}\]  
\end{te}

\begin{de}[Multiindex] $n\geq 1$ rögzített, $i$ multiindex, $i:=(i_1,\dotsc,i_n),\ i_k\geq 0$ egészek.\\
  $|i| := i_1 + i_2 + \ldots + i_n$ a multiindex rendje\\
  $i!~ := i_1! \cdot i_2! \dotsm i_n!$\\
  $x=(x_1,\dotsc,x_n)\in\R^n\colon\quad x^i := x_1^{i_1}\cdot x_2^{i_2}\dotsm x_n^{i_n}$\\
  $\partial^i\vfi := \partial_1^{i_1}\partial_2^{i_2}\dotsb\partial_n^{i_n}\vfi$ vagyis az első változó szerint
  $i_1$-szer, stb.\\
  $\partial^0\vfi :=  \vfi$
\end{de}

\begin{de}[Homogén $n$ változós $m$-edfokú polinom] \ \\$n=1,2,\dotsc$; $m=0,1,2,\dotsc$; $i$ multiindex: $|i|=m$
  \[\R^n\owns x\mapsto \sum_{|i|=m}a_ix^i\quad \text{ahol }a_i\in\R\]  
\end{de}

\begin{spec}{\listazjromai\begin{enumerate}
  \item $n=1;\ m=0,1,2,\dotsc\colon\quad \R\owns x\mapsto ax^m$
  \item $n=2;\ m=1\colon\quad i\colon (0,1) \text { v. }(1,0)\colon\quad \R^2\owns(x_1,x_2)\mapsto a_1x_1+a_2x_2$
  \item $n=2;\ m=2\colon\quad i\colon (2,0),\ (1,1),\ (0,2)\colon\quad \R^2\owns(x_1,x_2)\mapsto a{x_1}^2+bx_1x_2 +
    c{x_2}^2$    
  \end{enumerate} }
\end{spec}

\begin{te}[Taylor-formula a Lagrange-maradéktaggal]Tegyük fel, hogy
  {\listazjbetu \begin{enumerate}
    \item $\vfi\colon U\n\R,\ U\subset\R^n$ nyílt halmaz
    \item $a\in U,\ h\in\R^n\colon\ [a,a+h] := \{a+th:t\in(0,1)\}\subset U$
    \item $\vfi\in\D^{m+1}([a,a+h])\quad(m=0,1,2,\dotsc\text{rögzített})$
  \end{enumerate} }
  Ekkor $\exists \nu\in(0,1)$
  \[ \di\vfi(a+h) = \vfi(a) + \underbrace{\sum_{k=1}^m\left(\sum_{|i|=k}\dfrac{\partial^i\vfi(a)}{i!}h^i\right)}
  _{\text{Taylor-polinom}} + \underbrace{\sum_{|i|=m+1}\dfrac{\partial^i\vfi(a+\nu h)}{i!}h^i}
  _{\text{Lagrange-féle maradéktag}}\]  
\end{te}

\begin{biz}Visszavezethető $\R\n\R$-re\\
  \[ F(t) := \vfi(a+th)\qquad(t\in[0,1])\]
  Az $F\in\R\n\R$ függvényre a Taylor-formula alkalmazható a $[0,1]$ intervallumon (a feltételek teljesülnek).\\
  \[\di\exists \nu\in(0,1)\colon F(1) = F(0) + \sum_{k=1}^m \dfrac{F^{(k)}(0)}{k!} (1-0)^k + 
  \dfrac{F^{(m+1)}(\nu)} {(m+1)!}\]
  
  A tétel állítása a következő lemma felhasználásával adódik.
  \begin{lemma} A fenti $F$ függvény esetén ($\vfi$ $s$-szer deriválható $[a,a+h]$-n)
    \[\di\dfrac{F^{(k)}(t)}{k!}= \sum_{|i|=k}\dfrac{\partial^i\vfi(a+th)}{i!}h^i\qquad k=0,1,2,\dotsc,s\]
  \end{lemma}
  \textbf{A lemma bizonyítása} $k$-ra vonatkozó teljes indukcióval.\\
  $k=1$ esetén $F$ definiciója és az összetett függvény deriválási szabálya alapján
  \[ F'(t) = \skalar{\grad \vfi(a+th)}{h} = \sum_{|i|=1} \partial^i \vfi(a+th)\cdot h^i\qquad(t\in[0,1])\]
  Tegyük fel, hogy $k\in\{1,\dotsc,s-1\}$ esetén igaz az állítás. Így $k+1$-re:
  \begin{align*}
    \dfrac1{(k+1)!}F^{(k+1)}(t) &= \dfrac1{(k+1)!}(F^{(k)})'(t)\\
    &=\dfrac1{k+1}\sum_{|i|=k} \dfrac1{i!}
    (\partial_1\partial^i\vfi(a+th)h^ih_1+\ldots + \partial_n\partial^i\vfi(a+th)h^ih_n) ={}\\
    &\stackrel{\text{\ref{te:youngkov} alapján}}{=}
    \sum_{|i|=k+1}\dfrac{\partial^i\vfi(a+th)}{i!}h^i\qquad(t\in[0,1])
  \end{align*}
\end{biz}

\subsection{Inverz függvények ($\RnRn$)}
\underline{Globális, $\R\n\R$-beli tétel:}\\
\[f\colon (a,b)\n\R,\ f\in\D,\ f'>0\ (a,b)$-n, ekkor $\exists f^{-1}$ inverz, ui $f\uparrow\]
Ez nem általánosítható, de a lokális változata igen.:\\\\
\underline{Lokális, $\R\n\R$-beli tétel:}\\
$f: I\n \R, I\subset\R$ intervallum\\
$f$ folytonosan deriválható\\
$a\in I$-ben $f'(a)\neq 0$\\
EKKOR $\exists U := K(a)$ és $\exists V:= K(f(a))$: $f_{|U}\colon U\n V$ bijekció $\nn \exists$ inverz\\
és az $f^{-1}$ inverz differincálható és $\left(f^{-1}\right)'(x) = \dfrac1{f'(f^{-1}(x))}\quad\forall x\in V$


\begin{te}[Inverz függvény tétel] $\Omega \subset \R^n$ nyílt, $a\in\Omega,\ f:\Omega\n\R^n$. Tfh:
  {\listazjromai \begin{enumerate}
  \item $f$ folytonosan deriválható $\Omega$-n
  \item $\det f'(a) \neq 0$
  \end{enumerate} }
  Ekkor
  { \listazjbetu \begin{enumerate}
  \item $\exists U := K(a)$ és $\exists V:= K(f(a))$ $f_{|U}\colon U\n V$ bijekció (azaz $\exists f^{-1}$ inverz)
  \item $f^{-1}$ deriválható, $(f^{-1})'(x) = \begin{bmatrix} f'(f^{-1}(x))\end{bmatrix}^{-1}\quad (\forall x\in V)$
  \end{enumerate}
  }
\end{te}
\textbf{Alkalmazás.} Nemlineáris egyenletrendszerek megoldhatósága\\
$\left.\!\begin{array}{rcl}
f_1(x_1,\dotsc,x_n) & = & b_1\\
\vdots & & \vdots\\
f_n(x_1,\dotsc,x_n) & = & b_n\end{array}\right\}\qquad \begin{array}{cl} \left.\begin{array}{c}f_i\in\RnR\\b_i\end{array}
\right\} &\text{adott}\\x_1,\dotsc,x_n&\text{ismeretlen}\end{array}$\\
\vphantom{x}\\
$f := \begin{pmatrix}f_1\\\vdots\\f_n\end{pmatrix}\quad b := \begin{pmatrix}b_1\\\vdots\\b_n\end{pmatrix}$.\\
Legyen $a=\begin{pmatrix}a_1\\\vdots\\a_n\end{pmatrix}\in\R^n\colon f(a)=b$.\\
Keresni kell egy ilyet. Ha $f'(a)$
invertálható ($\ekviv \det f'(a)\neq 0$) $\overset{\text{Inverz fv}}{\underset{\text{tétel}}{\nn}}\\
\exists K(b)\ \forall y=(y_1,\dotsc,y_n)\in K(b)\ f(x)=y$ egyenletrendszer $x$-re egyéretlműen megoldható.
\begin{megj}Létezést biztosít.\\
A fixpont-tétel segítségével közelítő megoldás adható\end{megj}


\begin{de}$f\in\R^2\n\R$ adott. Ha $\exists I\subset \R$ intervallum és $\exists \vfi: I\n\R$, hogy\\ $f(x,\vfi(x))=0
  (x\in I)$, akkor a $\vfi$ az $f(x,y) =0$ \emph{implicit egyenlet megoldása}.\\ (a $\vfi$ függvény az $f(x,y)=0$
  implicit alakban van megadva)
\end{de}

\newpage
\begin{te}[Implicit függvény-tétel, spec: 2 változós] $f\in\R^2\n\R^1,\\D_f=\Omega\subset\R^2$ nyílt. Tfh:
  {\listazjromai \begin{enumerate}
  \item $f$ folytonosan differenciálható $\Omega$-n
  \item $\exists (a,b) \in \Omega:\ f(a,b)=0$ és $\partial_2f(a,b)\neq 0$
  \end{enumerate}}
  Ekkor
  {\listazjbetu
    \begin{enumerate}
    \item $\exists K(a) =: U\colon\ \forall x\in K(a)$-hoz $\exists \vfi(x)\in \R$, melyre $f(x,\vfi(x)) = 0\quad(\forall
      x\in U)$
    \item A $\vfi\colon U\n\R$ fv folytonosan deriválható és
      \[\vfi'(x) = - \dfrac{\partial_1f(x,\vfi(x))}{\partial_2f(x,\vfi(x))}\quad \forall x\in U \tag{$*$}\label{eq:*}\]
    \end{enumerate}
  }
\end{te}

\begin{Megj}
\item $($\ref{eq:*}$)$-ról: Ha $\vfi$ deriválható: $f(x,\vfi(x))=0\quad(x\in I) \stackrel{\text{láncszabály}}{\nn}\\
  \partial_1f(x,\vfi(x))\cdot1 + \partial_2f(x,\vfi(x))\cdot\vfi'(x) = 0$ 
\item Általánosítás: $n_1,n_2\in\N;\ \Omega_1\subset\R^{n_1};\ \Omega_2\subset\R^{n_2}$ nyíltak. $a\in\Omega_1,\
  b\in\Omega_2,\\ f\colon \Omega_1\times\Omega_2\n\R^{n_2}$, ahol $\Omega_1 \times \Omega_2$ az első, illetve a második
  változócsoportot jelöli.\\
  Legyen $\partial_1f(a,b) := (\Omega_1\owns x\mapsto f(x,b)\in\R^{n_2})'_{x=a}$ az első változócsoport szerinti
  derivált, illetve\\$\partial_2f(a,b) := (\Omega_2\owns y\mapsto f(a,y)\in\R^{n_2})'_{y=b}$ az második
  változócsoport szerinti derivált
\end{Megj}

\begin{te}[Implicit függvény-tétel, ált] \ldots $f\in\Omega_1\times \Omega_2\n\R^{n_2}$. Tfh:
  \begin{enumerate}[\quad(i)]
  \item $f$ folytonosan differenciálható
  \item $\exists a \in \Omega_1,\ b\in\Omega_2:\ f(a,b)=0$ és $\det(\partial_2f(a,b))\neq 0$
  \end{enumerate}
  \noindent Ekkor
  {\listazjbetu
    \begin{enumerate}
    \item $\exists U_1:= K(a)\subset R^{n_1}$ és $\exists U_2:=K(b)\subset \R^{n_2}$:\\
      $\forall x\in U_1$-hoz $\exists! \vfi(x)\in U_2$, melyre $f(x,\vfi(x)) = 0$
    \item A $\vfi\colon U_1\n U_2$ fv folytonosan deriválható és
      \[\vfi'(x) = - [\partial_2f(x,\vfi(x))]^{-1}\cdot \partial_1f(x,\vfi(x))\quad\forall x\in U_1 \]
    \end{enumerate}
      }
\end{te}

\subsection{Szélsőértékek ($\R^n\n \R^1$)}
\begin{te}[Elsőrendű szükséges feltétel lokális szélsőértékre]
Tfh: \\$\vfi\in U\n\R,\ U\subset \R^n$ nyílt,
\begin{enumerate}
\item $\vfi \in \der a\quad a\in U$ (belső pont!!!)
\item $\vfi$-nek lokális szélsőértéke van $a$-ban
\end{enumerate}
Ekkor  \[\vfi'(a) = (\partial_1\vfi(a),\dotsc,\partial_n\vfi(a))=0\tag{$**$}\label{eq:**}\]

\end{te}
\begin{biz}Trivi, ui: $t\mapsto f(a+te_i)\ (\in\R\n\R)$ parciális függvénynek is lokális szélsőértéke van $t=0$-ban.
\end{biz}

\newpage
\begin{Megj}
\item Szükséges, de nem elégséges: $(n=1\colon f(x) := x^3)$
\item $($\ref{eq:**}$)\ekviv \left.\begin{array}{c}\partial_1\vfi(a)=0\\\partial_n\vfi(a)=0\end{array} \right\}$
  $n$ db egyenlet, $n$ db ismeretlen: $(a_1,\dotsc,a_n)$\\
  Itt lehet csak szélsőérték
\end{Megj}

\begin{de}
  $\vfi: U\n\R,\ U\subset \R^n$ nyílt, $a\in U$. A $\vfi$-nek az $a$-ban lokális minimuma [maximuma] van, ha $\exists
  K(a) (\subset U)\colon \vfi(a) \leqq \vfi(x) \ [\vfi(a) \geqq \vfi(x)]  \quad(x\in K(a))$
\end{de}
\begin{megj}
  Lokális szélsőérték $\ekviv$ lokális minimum vagy lokális maximum
\end{megj}

\begin{de}[Kvadratikus alak]\ 
  Az $A=[a_{ij}]\in\R^{n\times n}$ szimmetrikus mátrix,\\ $h=(h_1,h_2,\dotsc,h_n)\in\R^n$. A  $Q\colon \RnR$
  \[ Q(h) := \skalar{Ah}h = \sum_{i=1}^n a_{ij}h_ih_j\]
  fv-t az \emph{$A$ mátrix által meghatározott kvadratikus formának} nevezzük 
\end{de}

\begin{megj}
  $\di Q(h) = \sum_{|i|=2}a_ih^i\quad i=(i_1,\dotsc,i_n)$ multiindex\\
  Ez egy homogén n-változós másodfokú polinom.
\end{megj}
\begin{de}
  $A=[a_{ij}]\in\R^{n\times n}$ szimmetrikus.\\
  A $Q(h)$ kvadratikus forma (vagy az $A$ mátrix)
  \begin{itemize}[\quad]
  \item \underline{pozitív definit}, ha $Q(h)>0\quad\forall h\in\R^n\setminus\{0\}$
  \item \underline{negatív definit}, ha $Q(h)<0\quad\forall h\in\R^n\setminus\{0\}$
  \item \underline{pozitív szemidefinit}, ha $Q(h)\geq0\quad\forall h\in\R^n$
  \item \underline{negatív szemidefinit}, ha $Q(h)\leq0\quad\forall h\in\R^n$
  \end{itemize}
\end{de}


\begin{te}[Sylvester-kritérium]$Q(h) = \skalar{Ah}h$ kvadratikus alak,\\$A=[a_{ij}]\in\R^{n\times n}$ szimmetrikus
  mátrix
  
  \[A=\begin{bmatrix}a_{11} & a_{12} & \cdots & a_{1n}\\
  a_{21} & a_{22} & \cdots & a_{2n}\\ \vdots & \vdots & \ddots & \vdots \\
  a_{n1} & a_{n2} & \cdots & a_{nn}\end{bmatrix};\qquad \Delta_k = \det\begin{bmatrix}a_{11}&\cdots&a_{1k}\\
  \vdots &\ddots& \vdots\\a_{k_1} & \cdots & a_{kk}\end{bmatrix} \text{sarok-aldeterminánsok}\]
Ekkor
\begin{enumerate}
\item $Q$ pozitív definit $\ekviv\Delta_1>0,\,\Delta_2>0,\dotsc,$ azaz $\sgn\Delta_k=1\quad k=1,2,\dotsc,n$
\item $Q$ negatív definit $\ekviv\Delta_1<0,\,\Delta_2>0,\dotsc,$ azaz $\sgn\Delta_k=(-1)^k \quad k=1,2,\dotsc,n$
\end{enumerate}
\end{te}

\begin{te}
  Ha $Q$ kvadratikus forma $\nn\\\exists m,M\in\R\colon m\norma{h}^2\leq Q(h)\leq M\norma{h}^2\quad(h\in\R^n)$
\end{te}
\begin{biz}
  $Q\colon \RnR$ folytonos függvény, $H:=\{x\in\R^n,\norma x = 1\}$ kompakt $\stackrel{\text{Weierstrass}}{\nn}$\\
  \[\exists M:=\max\{Q(h) : \norma h = 1\},\ \exists m:=\min\{Q(h) : \norma h = 1\}\]
  DE!\\
  \[Q(h) =  Q\left(\norma{h}\cdot\dfrac{h}{\norma{h}}\right) = \norma{h}^2Q\left(\dfrac{h}{\norma{h}}\right)\ \ \nn\ 
  \ Q(h)\leq \norma{h}^2 M,\ Q(h)\geq m\norma{h}^2\]
\end{biz}

\begin{kov}
  $Q(h)$ kvadratikus forma,\\
  $Q$ pozitív definit $\ekviv \exists c_1>0\colon Q(h)\geq c_1\norma{h}^2$\\
  $Q$ negatív definit $\ekviv \exists c_2<0\colon Q(h)\leq c_2\norma{h}^2$\\
  Az előző tételből adódik
\end{kov}

\begin{te}[Másodrendű elégséges feltétel, lokális szélsőértékre]\ \\
  Tfh $\vfi\colon U\subset\R,\ U\n\R^n$ nyílt, $a\in U$ belső pont!!!
  {\listazjromai
    \begin{enumerate}
    \item $\vfi$ kétszer folytonosan deriválható
    \item $\vfi'(a)=0$
    \item $\vfi''(a)$ Hesse-féle mátrix által generált kvadratikus alak pozitív [negatív] definit.
    \end{enumerate}
}
Ekkor $\vfi$-nek $a$-ban lokális minimuma [maximuma] van.
\end{te}

\begin{megj} $\vfi''(a)=\ldots$ + Sylvester
\end{megj}
\begin{biz}$a,a+h\in K(a)$, $f\in\dern2x\ \forall x\in K(a)$\\
  Taylor-formula alapján $\exists \nu\in(0,1)$:
\begin{gather*}
  f(a+h)-f(a)=\sum_{i=1}^n\partial_if(a)\cdot h_i+\dfrac12\cdot\sum_{i,j=1}^n\partial_i\partial_jf(a+\nu h)h_ih_j=
  \\ = \sum_{|i|=1}\dfrac{\partial^i f(a)}{i!} + \sum_{|i|=2}\dfrac{\partial^i f(a+\nu h)}{i!} =\\
  =\dfrac12\sum_{i,j=1}^n\partial_i\partial_jf(a+\nu h)h_ih_j,\text{ ui }f'(a)=0\text{, így }\partial_if(a)=0.\\
  \epsilon_ij(h):=\partial_i\partial_jf(a+\nu h)-\partial_i\partial_jf(a)\quad(i,j=1,\dotsc,n)\\
  \text{Az első feltétel alapján} \lim_0\epsilon_{ij}=0\\
  f(a+h)-f(a)=\dfrac12(Q(h)+R(h))\\
  \intertext{ahol}
  R(h):=\sum_{i,j=1}^n\epsilon_{ij}(h)h_ih_j\\
  a+h\in K(a),\, h\ne 0\ \nn\ \vert R(h)\vert=\norma{h}^2\left\vert\sum_{i,j=1}^n\epsilon_{ij}(h)\dfrac{h_i}{ 
    \norma{h}}\dfrac{h_j}{\norma{h}}\right\vert\leq\norma{h}^2\cdot\sum_{i,j=1}^n|\epsilon_{ij}(h)|\\
  \nn \exists \delta>0\colon \forall h\in\R^n\ a+h\in K(a),\  \norma h<\delta:\\
  |R(h)|\leq \dfrac m2\norma{h}^2\quad m:=\min\{Q(h)\in\R:\norma h=1\}\\
  f(a+h)-f(a)\geq \frac m2\norma{h}^2-\frac m4\norma{h}^2=\frac m4\norma{h}^2
\end{gather*}
vagyis $f$-nek az $a$ pontban lokális minimuma van.
\end{biz}

\begin{te}
  $\vfi\colon U\n\R,\ U\subset \R^n$ nyílt, $a\in U$
{\listazjromai
\begin{enumerate}
\item $\vfi$ kétszer folytonosan deriválható
\item $\vfi'(a)=0$ és a $\vfi''(a)$ álatal generált kvadratikus forma indefinit
\end{enumerate}
}
Ekkor $a$-ban $\vfi$-nek nincs lokális szélsőértéke
\end{te}
\begin{biz}
  Ha indefinit: nem szemidefinit $\nn$ szükséges feltétel alapján nincs szélsőérték
\end{biz}

\subsubsection{Feltételes szélsőérték}
\begin{PlSS}
  Adott: $x+y-2=0$ egyenletű egyenes. Melyik rajta lévő $P$ pont esetén lesz $\overline{OP}$, azaz az origótól való
  távolság minimális? Azaz:\\
  $f(x,y) := x^2 + y^2\ (x,y)\in\R^2$\\
  $H:=\{(x,y)\in\R^2| x+y-2=0\}$\\
  $f_{|H}\n\min$
\end{PlSS}
\begin{PlSS}\label{plss:fsz2}
  Adott körbe maximális területű téglalapot kell tenni.\\
  Egyszerűsítés: elég az első síknegyed, mert szimmetrikus.
  $T(x,y) := 4xy$\\
  $H:=\{(x,y)\in\R^2| x^2 + y^2 - R^2 =0\}$\\
  $f_{|H}\n\max$
\end{PlSS}
Adott $m,n\in\N,\ U\subset\R^n$ nyílt,\\
$f\colon U\n\R$ és\\
$g_i\colon U\n\R\quad i=1..m$\\
$H:=\{ z\in\R^n\,|\,g_i(z)=0,\ i=1,\dotsc,m\}$ feltételek.\\
Határozzuk meg $f_{|H}$ lokális szélsőértékeit.

\begin{de}
  Az $f\colon U\n\R$ fv-nek  a $c\in U$-ban a $g_i(z)=0\ \,(i=1..m)$ feltételekre vonatkozó \emph{lokális feltételes
    minimuma van}, ha
  \[\exists K(c): f(x) \geq f(c)\quad \forall x\in K(c)\cap H\]
\end{de}

\begin{te}[Lokális feltételre vonatkozó szükséges feltétel]
  Tfh
{\listazjbetu
  \begin{enumerate}
  \item $n,m\in\N,\ U\in\R^n$ nyílt,\\$f\colon U\n\R,\ g_i: U\n\R\ \ (i=1,\dotsc,m)$ folytonosan deriválhatóak.
  \item $f$-nek a $c\in U$-ban a $g_i(c)=0\ \,(i=1,\dotsc,m)$ feltételekre vonatkozó lokális szélső
  \item $g_i'(c)\ (i=1,\dotsc,m)$ lineárisan független vektorok
  \end{enumerate}
}
  Ekkor $\exists\lambda_1,\dotsc,\lambda_m\in\R\colon \di F:= f+\sum_{i=1}^m \lambda_ig_i$ fv-nek $c$-ben  $F'(c) = 0$
\end{te}
\begin{megj}Alkalmazáshoz:
  \[ \left.\begin{array}{l}
    \left.\begin{array}{c}\partial_1F(c)=0\\\partial_2F(c)=0\\\vdots\\\partial_nF(c)=0\end{array}\right\} \text{ n db}\\
      \left.\begin{array}{c}g_1(c)=0\\\vdots\\g_m(c)=0\end{array}\right\} \text{ m db}\\
  \end{array}\right\} \text{ n+m db egyenlet a $\lambda_1,\dotsc,\lambda_m$ és $c_1,\dotsc,c_n$ ismeretlenekre} 
  \tag{$\sharp$}\label{eqs:lagrange-mult}
  \]
  Lokális szélsőérték csak ilyen $c$-ben lehet	
\end{megj}


\textbf{Alkalmazás}\\
\Aref{plss:fsz2} példa:
\begin{gather*}
  f(x,y) := 4xy\quad (\,(x,y)\in\R^2\,)\\
  g(x,y) := x^2 + y^2 -R^2\\
  F(x,y) := 4xy + \lambda(x^2 + y^2 - R^2)\\
  \left.
  \begin{gathered}
    \partial_1 F(x,y) = 4y + 2\lambda x = 0\\
    \partial_2 F(x,y) = 4x + 2\lambda y = 0  
  \end{gathered}\,
  \right\}\quad \oplus\colon 2(x+y)(\lambda+2) = 0\\
  x^2+y^2 - R^2 = 0 
  \intertext{Innen:}
  \lambda = -2\\
  x=y=\dfrac{R}{\sqrt{2}}
\end{gather*}
Azaz $\left(\dfrac{R}{\sqrt{2}},\,\dfrac{R}{\sqrt{2}}\right)$-ben lehet szélsőérték.

\begin{te}[Elégséges feltétel a lokális szélsőértékre]\ 
{\listazjbetu
  \begin{enumerate}
  \item $f,g_i\colon U\n\R$, $U\subset\R^n$ nyílt, $i=1,\dotsc,m$ kétszer folytonosan differenciálható
  \item $c=(c1,\dotsc,c_n)$, $\lambda_1,\dotsc,\lambda_m$ kielégíti \aref{eqs:lagrange-mult}-t
  \item $F:= f + \di\sum_{i=1}^m\lambda_i g_i$-nek $c$-ben lokális szélsőértéke van (feltétel nélküli: a teljes
    értelemezési tartományt figyelembe véve).  
  \end{enumerate}
}
Ekkor $f$-nek $c$-ben a $g_1=\dotsb=g_m=0$ feltételekre vonatkozó feltételes lokális szélsőértéke van.
\end{te}
\ref{plss:fsz2}: $\lambda=2$; $F(x,y) = 4xy - 2(x^2 +y^2 - R^2) = 2R^2 - 2(x+y)^2$.



% Local Variables:
% fill-column: 120
% TeX-master: t
% End:

%  \newpage
%  \section{Paraméteres integrál}
\begin{te}
  $I=[a,\,b],\ U\subset \R^n$ nyílt és $f\colon U\times[a,b]\n\R$ folytonos,\\
  $\vfi(x) := \di\Int_a^bf(x,t) \diff t\quad(x\in U)\quad$ az $f$ paraméteres integrálja.\\
Ekkor:
{\listazjbetu
\begin{enumerate}
\item  $\vfi\colon U\n\R$ is folytonos
\item Ha $\exists \partial_if$ és $\partial_if\in \Folyt\quad (i=1,\dotsc,n)$, akkor $\vfi$ is deriválható és
\begin{gather*} \di \partial_i\vfi(x) = \Int_a^b\partial_if(x,t)\diff t\quad x\in U\end{gather*}
\end{enumerate}
}
\end{te}
\begin{pl}
  \begin{gather*}
    \vfi(x) := \Int_0^1 ln(t^2+x^2) \diff t\quad(x>0).\\
    \vfi \in\Der \text{ és } \vfi'(x) = \Int_0^1\dfrac{\partial}{\partial x} ln(t^2+x^2) \diff t =
    \Int_0^1 \dfrac{2x}{x^2+t^2} \diff t= \dfrac{2x}{x^2}\Int_0^1 \dfrac1{1+\frac t x} \diff x = {}\\
	{}=\dfrac2x[x \cdot \arctg\dfrac t x]^1_{t=0} = 2 \arctg \dfrac1x
  \end{gather*}
\end{pl}
\newpage
\section{Vonalintegrál ($\R^n\to\R^n$ függvényekre)}
\subsection{Sima utak, görbék}
\begin{de}[Sima út]$n\in\N\ \vfi\colon[a,\,b]\to\R^n$ folytonosan deriválható függvényt\\\emph{$\R^n$-beli sima út}nak
    nevezzük.\\  Az $R_\vfi = \Gamma\subset\R^n$ halmaz \emph{sima görbe}, $\vfi$ a $\Gamma$ görbe egy paraméterezése.
\end{de}

\begin{de}[Szakaszonként sima út]$a,b\in\R;\ a\leq b$. A $\vfi\colon[a,b]\to \R^n$ függvény  \emph{$\R^n$-beli
    szakaszonként sima út}, ha 
{\listazjromai
  \begin{enumerate}
  \item $\vfi\in\Folyt$
  \item $\exists a=t_0<t,\ 1<\dotsb<t_m=b$: $\vfi_{|[t_i,\,t_i+1]}\ \ i=1,\dotsc,m-1$ sima út.
\end{enumerate}
}
\end{de}

\begin{Pl}
\item Szakasz: $a,b\in \R^n\ \vfi(t) := a+t(b-a)\quad(t\in[0,\,1])$
\item Töröttvonal - szakaszonként sima út
\item Kör: $\vfi(t) := (\sin t,\cos t)\quad t\in[0,2\pi]\\
  R_\vfi=\Gamma$
\end{Pl}


\begin{de}[Szakaszonként sima utak egyesítése]
  $\vfi\colon [a,a+h]\to\R^n$\\$\psi\colon[b,b+k]\to\R^n$ szakaszonként sima utak, és tfh: $\vfi(a+h)=\psi(b)$, azaz
  $\vfi$ végpontja megegyezik $\psi$ kezdőpontjával. \\
  A $\vfi$ és $\psi$ egyesítése $(\vfi\cup\psi)$:
\[\Phi(t) = \begin{cases}\vfi(t) & t\in[a,a+h]\\\psi(t-a-h+b) & t\in[a+h,a+h+k]\end{cases}\]
\end{de}

\begin{de}[$\vfi$ ellentettje] $\widetilde{\vfi} := \vfi(2a+h-t)\qquad(t\in[a,\,a+h])$\\
  az út $a+h\to a$ irányú lett.
\end{de}

\subsection{Vonalintegrál definíciója}
\begin{te}Legyen $U\subset \R^n$ nyílt.\\
  $U$ összegüggő $\ekviv \forall x,y\in  U$ összeköthető $U$-beli szakaszonként
  sima úttal.
\end{te}

\begin{de}[Tartomány]Az $U\subset \R^n$ halmaz \emph{tartomány}, ha
{\listazjromai
  \begin{enumerate}
    \item $U$ nyílt $\R^n$-ben
    \item $U$ összefüggő
  \end{enumerate}
}
\end{de}
\begin{de}[Úton vett vonalintegrál]
  Legyen $U\subset \R^n$ tartomány, $f\colon U\to\R^n$ \underline{folytonos}, $\vfi\colon [a,b]\to\R^n$ szakaszonként
  sima. Ekkor
\[\Int_a^b\skalar{f\circ\vfi}{\vfi'} = \Int_a^b\skalar{f(\vfi(t))}{\vfi'(t)\,}\diff t =: \Int_\vfi f\]
szám az $f$ függvény $\vfi$ útra vett vonalintegrálja.
\end{de}
\newpage
\begin{Megj}
  \item $f$ folytonos $\nn$ az integrandus folytonos $\nn$ az integrál létezik.
\item $n=1,\ \vfi(t) := t\quad t\in[a,b]$
\[\Int_\vfi f \text{ az } \Int_a^bf(t)\diff t \text{ Riemann-integrálja}\]
\end{Megj}

\begin{te}[A vonalintegrál egyszerű tulajdonságai]
  $U\subset \R^n$ tartomány,\\$\vfi\colon [a,a+h]\to\R^n$ és $\psi\colon [b,b+k]\to\R^n$ szakaszonként sima utak,
  $\vfi(a+h) = \psi(b)$.\\$f,g\colon U\to\R^n$ folytonos. Ekkor
  \begin{enumerate}
  \item $\di\Int_\vfi(\lambda_1 f +\lambda_2g) = \lambda_1\Int_\vfi f+ \lambda_2\Int_\vfi g$
  \item $\di\Int_\vfi f = -\Int_{\widetilde{\vfi}}$\qquad (ellentett út)
  \item $\di\Int_{\vfi\cup\psi}\!\!\! f = \Int_\vfi f + \Int_\psi f$
  \item $\di\Big\vert \Int_\vfi f\Big\vert \leqq M\cdot l(\vfi)$, ahol $l(\vfi)$ a $\vfi$ (vagy a $\Gamma$ görbe)
  hossza és $M:= \max \{\,\norma{f(x)}_2:x\in R_\vfi\}$
  \end{enumerate}
\end{te}

\subsection{Primitív függvények}
\begin{de}[Primitív függvény]$U\subset\R^n$ tartomány, $f\colon U\to\R^n$.\\
  Az $F\colon U\to\R^n$ függvény az $f$ primitív függvénye, ha
  {\listazjromai
    \begin{enumerate}
    \item $F\in\Der$
    \item $F'(x) = f(x)\quad (\forall x\in U)$
    \end{enumerate}
  }
\end{de}

\begin{megj}
  Ha $F\in\Der$: $F'=(\partial_1F,\dotsc\partial_nF) =(f_1,\dotsc,f_n)=f$ 
\end{megj}

\begin{te}\ 
  \begin{enumzjromai}
  \item Ha $F\colon U\to\R$ az $f$ primitív függvénye $\nn \forall c\in\R\colon F+c$ is az
  \item Ha $F_1,\,F_2\colon U\to\R$ az $f$ primitív függvényei $\nn \exists c\in\R\colon F_1(x)-F_2(x) = c \quad \forall
  x\in U$
  \end{enumzjromai}
\end{te}
\begin{te}[Newton-Leibniz]
  Tfh:
\begin{enumzjromai}
  \item $U\subset \R^n$ tartomány
  \item $f\colon U\to \R^n$ folytonos
  \item $\vfi\colon [a,b]\to U$ szakaszonként sima út
  \item $f$-nek $\exists F$: a primitív fv-e
\end{enumzjromai}
Ekkor $\di\Int_\vfi f = F(\vfi(b))-F(\vfi(a))$
\end{te}
\begin{biz}
\begin{gather*}\text{(ii)}\nn a=t_0<t_1<\dotsb<t_m=b\ (m\in\N).\ \ F\circ\vfi\in\Der[t_{i-1},t_i]\ (i=1,\dotsc,m).\\
  (F\circ\vfi)'(t)=\skalar{F'(\vfi(t))}{\vfi'(t)}=\skalar{(f\circ\vfi)(t)}{\vfi'(t)}\quad(t\in[t_{i-1},\,t_i],\ 
  i=1,\dotsc,m)\\
  \intertext{Ezekre az intervallumokra alaklmazva az egyváltozós Newton-Leibniz formulát}
  \Int_\vfi f=\sum_{i=1}^m\Int_{t_{i-1}}^{t_i}\skalar{(f\circ\vfi)(t)}{\vfi'(t)}\diff t = \sum_{i=1}^m
  \left(F(\vfi(t_i)-F(\vfi(t_{i-1}))\right) = F(\vfi(b))-F(\vfi(a))
\end{gather*}
\end{biz}

\begin{pl}
  Primitív fv. meghatározása\\
  $f(x,y) = \left(\dfrac{y}{x^2},\,-\dfrac1x\right)\qquad(x>0,y>0)$\\
  Létezik-e $F\colon \R^2\to\R$, hogy $F'=f$.\\
  $\left.\!\begin{gathered}\partial_xF(x,y) = \dfrac{y}{x^2}\\\partial_yF(x,y)=-\dfrac1x+h'(y)\end{gathered}
  \right\}\nn F(x,y) = -\dfrac y x+h(y)$\\
  $h'(y) = 0\nn h\equiv$ állandó\\
  így $F(x,y) = -\dfrac y x +c\quad (x,y)>0$
\end{pl}

\begin{te}
  $U\subset \R^n$ taromány, $f\colon U\to\R$ folytonos.\\
  $f$-nek létezik primitív függvénye $\ekviv \left\{\begin{array}{l}\forall U\text{-ban haladó szakaszonként sima és
  zárt }\vfi\text{ útra:}\\\di\Int_\vfi f= 0\end{array}\right.$
\end{te}

\begin{megj}Jelölés: $\oint$: zárt útra vett integrál, körintegrál.
\end{megj}

\begin{biz}
  $\underline{\Rightarrow:}$ trivi. Newton-Leibniz: $\di\oint f=F(\vfi(b)) - F(\vfi(a))$, de $\vfi$ zárt:
  $\vfi(a)=\vfi(b)$
  $\underline{\Leftarrow:}$ Több lépésben
  \begin{enumzjbetu}
    \item Ha $\di \Oint_\vfi f=0\nn \forall x,a\in U$ és $\forall \vfi_1,\vfi_2$, ami $x$-et, $a$-t összeköti:
    $\di \Oint_{\vfi_1} f = \Oint_{\vfi_2} f$: a vonalintegrál független a két pontot összekötő útttól, ugyanis\\
      $\vfi_1\cup\widetilde{\vfi}_2$ zárt görbe, $\di 0=\Int_{\vfi_1\cup\widetilde{\vfi}_2}\!\!\!\!f = \Int_{\vfi_1} f+
    \Int_{\widetilde{\vfi}_2}\!f\quad\nn\quad \Int_{\vfi_1}\!f - \Int_{\vfi_2}\!f = 0$
  \item Ha $\di\forall \Oint_\vfi f=0$, akkor definiálhatjuk a következő függvényt:
    \[\di a\in U \text{ rögzített; }\Phi(x) := \Int_{\overline{ax}}f(x)\quad\forall x\in U\]
    ahol $\overline{ax}$ az $a$-t $x$-szel összektő szakaszonként sima út. Ez a függvény az $f$-nek $a$-ban eltűnő
    integrálfüggénye.
  \item Ha $\di\forall\Oint_\vfi f=0\,\nn$ a fenti $\Phi$ integrálfüggvény deriválható és $\Phi'=f$, azaza a $\Phi$
    integrálfüggvény az $f$ egy primitív függvénye (az integrálfüggvény deriválhatóságára vonatkozó tétel alapján)
    \begin{gather*}
       \Phi(x+h)-\Phi(x)=\Int_0^1\skalar{f(x+th}h\diff t
       \intertext{$f$ folytonos, ezért}
       \epsilon(h):=\sup{\norma{f(x+th)-f(x)}:0\leq t\leq 1}\to 0\quad(h\to0).
       \intertext{Továbbá}
       \left\vert\Phi(x+h)-\Phi(x)-\skalar{f(x)}h\right\vert = \left\vert\Int_0^1\skalar{f(x+th)-f(x)}h\diff t
       \right\vert \leq\epsilon(h)\cdot\norma h.
    \end{gather*}
    Vagyis $\Phi$ differenciálható és $\Phi'=f$.
  \end{enumzjbetu}
\end{biz}
\begin{te}[Szükéges feltétel a primitív függvény létezésére]
  $U\subset\R^n$ tartomány,\\$f\colon U\to\R^n$
  \begin{enumzjr}
    \item $f$ \underline{deriválható}
    \item $f$-nek létezik primitív függvénye
  \end{enumzjr}
  Ekkor $f'$ deriváltmátrix szimmetrikus, azaz $\partial_if_j=\partial_jf_i\ (\forall 1\leq i,j\leq n)$ és
  $f=(f_1,\dotsc,f_n)$
\end{te}

\begin{biz}(ii) $\nn \exists F\colon U\to\R,\ F\in\Der$ és $F'=f$\\
  (i)$\nn F\in\Der^2\overset{\text{Young-t.}}{\nn} \partial_i(\underbrace{\partial_jF}_{f_j}) =
  \partial_j(\underbrace{\partial_iF}_{f_i})$\\
$F'=(\partial_1,\dotsc,\partial_n)=(f_1,\dotsc,f_n)$
\end{biz}

\begin{Megj}
\item $\R\to\R$ esetén $\forall$ folytonos függvénynek létezik primitív függvény\\
  Ha $n\geq 2$, akkor $\exists f$ deriválható, melynek nincs primitív függvénye.
\item Csillagtartományon ez a szükséges felétel elégséges is
\end{Megj}
\begin{de}[Csillagtartomány]
  $U\subset \R^n$ az $a\in U$ pontra nézve csillagtartomány, ha $\forall x\in U: [a,x]\subset U$
\end{de}

\begin{te}[Elégséges feltétel a primitív függvény létezésére]
  Tfh:
  \begin{enumzjr}
  \item $U\subset \R^n$ az $a\in U$-ra csillagtartomány
  \item $f\colon U\to\R^n$ folytonosan deriválható
  \item $f'$ deriváltmátrix szimmetrikus
  \end{enumzjr}
  Ekkor $F$-nek $\exists$ primitív függvénye, az
  \[\di U\owns x\mapsto\!\!\Int_{[a,x]}\!\!\!f\]
  az  $f$ egy $a$-ban eltűnő primitív függvénye
\end{te}

\begin{biz}
  Megmutatjuk, hogy
\begin{gather*}
  U\owns x\mapsto F(x):=\Int_a^xf=\Int_0^1\skalar{f(a+t(x-a))}{x-a}\diff t
  \intertext{függvény differenciálható és $F'=f$.}
  \partial_iF(x) = \Int_0^1\left(\sum_{j=1}^nt\cdot\partial_if_j(a+t(x-a))\cdot (x_j-a_j)+f_i(a+t(x-a))\right)\diff t
  \intertext{$f'$ szimmetrikus, így}
  \partial_iF(x) = \Int_0^1\left(t\cdot \sum_{j=1}^n\partial_jf_i(a+t(x-a))(x_j-a_j)+f_i(a+t(x-a))\right)\diff t = \\
  \hspace*{2em}= \Int_0^1\left(t\cdot\dfrac{\partial}{\partial t}f_i(a+t(x-a))+f_i(a+t(x-a))\right)\diff t =\\
  \hspace*{2em}=f_i(x)-\Int_0^1f_i(a+t(x-a))\diff t + \Int_0^1f_i(a+t(x-a))\diff t =f_i
\end{gather*}  
\end{biz}

\newpage
\section{Többszörös integrál}
\subsection{A többszörös integrál fogalma}
Két lépésben: $N$-dimeziós intervallum, majd tetszőleges $H\subset\R^N$ korlátos halmaz

\subsubsection{$N$-dimeziós intervallum és felosztása}
\begin{de}
  $I^j:=[a^j,\,b^j]\subset \R\quad \forall j=1,\dotsc,N$\\
  $I := I^1\times I^2\times\dotsb\times I^N\quad\R^N$-beli kompakt intervallum\\
  $\mu(I) := (b^1-a^1)\cdot (b^2-a^2)\dotsm(b^N-a^N)$ az $I$ mértéke.
\end{de}
\begin{de}[Felosztás - egydimenziós]
  $[a,b]\subset \R,\\\tau := \{\,[x_{r-1},\,x_r]:r=1,2,\dotsc,m\}\in\mathcal{F}([a,b])$
\end{de}
\begin{de}[Felosztás]
  Legyen $\tau_j\in\mathcal{F}(I^j)\quad (j=1,\dotsc,N)$.\\
  $\tau := \tau_1\times\tau^2\times\dotsb\times\tau^N=\{I^1_{r_1}\times I_{r_2}^2\times I^N_{r_N}:1\leq r_j\leq m\}$ az
  $I$ intervallum egy felosztása\\
  Jel: $\mathcal{F}(I)$ a felosztás egy halmaza
\end{de}

\begin{de}
  $f\colon I\to\R,\ I\subset \R^N$ kompakt intervallum, $f$ korlátos, $\tau\in\mathcal{F}(I)$.
  \begin{gather*}
    s(f,\tau):=\sum_{J\in\tau}(\inf f_{|J})\cdot \mu(J)\\
    S(f,\tau) :=\sum_{J\in\tau}(\sup f_{|J})\cdot \mu(J)
  \end{gather*}
  az $f$ függvény $\tau$ felosztáshoz tartozó alsó- illetve felső közelítő összege.
\end{de}
\begin{megj}
$\{\,s(f,\tau):\tau\in\mathcal{F}(I)\}$ és $\{\,S(f,\tau):\tau\in\mathcal{F}(I)\}$ korlátosak
\end{megj}
\begin{de}
  $f\colon I,\to \R,\ I\subset \R^N$ kompakt intervallum, $f$ korlátos függvény.
  \begin{gather*}
    \sup\, \{\,s(f,\tau):\tau\in\mathcal{F}(I)\,\} =: I_*f\\
    \inf\, \{\,S(f,\tau):\tau\in\mathcal{F}(I)\,\} =: I^*f
  \end{gather*}
  az $f$ Darboux-féle alsó-, ill felső integrálja.
\end{de}

\begin{de}Az $f$ Riemann-integrálható, jel: $f\in\Rint(I)$, ha $I_*f=I^*f=\Int_If$.\\
  $\di\Int_If$ az $f$ Riemann integrálja $I$-n.
\end{de}

\begin{megj}$I_*f$, $I^*f$ létezik minden ilyen függvényre
\[I_*f\leq I^*f,\text{ ui }\forall\tau,\sigma\in\mathcal{F}(I)\colon s(f,\tau)\leq S(f,\sigma)\]
\end{megj}

\subsection{A többszörös integrál alapvető tulajdonságai}
\begin{te}
  $I\subset\R^N$ kompakt, $f,g\colon I\to\R$ korlátos függvények, $f,g\in\Rint(I)$. Ekkor
\begin{enumzjb}
  \item $f+g\in\R(I)$, és $\di\Int_I (f+g) = \Int_I f + \Int_I g$
  \item $\forall \lambda\in\R\colon \lambda f\in\Rint(I)$ és $\di\Int_I(\lambda f)=\lambda\Int_I f$
\end{enumzjb}
\end{te}

\begin{pl}Nem integrálható a következő függvény, ahol $(x,y)\in [0,1]\times[0,1]=:I$.\\
  $f(x,y):=\begin{cases}0 & x,y \text{ racionális }\\1 &\text{ különben}\end{cases}$\\
  $f\not\in\Rint(I)$, ui $I_*f=0\neq I^*f=1$
\end{pl}
\begin{te}[Egyenlőtlenség]$I\subset\R^N$ kompakt intervallum, $f,g\colon I\to\R$ és tfh:\\$f,g\in\Rint(I)$. Ekkor
  \begin{enumzjb}
    \item $f\leq g\ I$-n $\di\nn \Int_If\leq \Int_Ig$
    \item $\vert f\vert\in\Rint(I)$ és $\di\left\vert\Int_If\right\vert\leq\Int_I\vert f\vert$
    \item \textbf{első középértéktétel} Ha $g\geq 0\ I$-n:
      \begin{gather*}
	m\Int_Ig\leq \Int_I(fg)\leq M\Int_Ig\\
	M:=\sup_If\\
	m:=\inf_If
      \end{gather*}
    \item \textbf{második középértéktétel} Ha  még $f\in\Folyt(I)$ is:
      \[\exists \xi \in I\quad \Int_I fg = f(\xi)\Int_Ig\quad g\geq 0\ I\text{-n}\]
  \end{enumzjb}
\end{te}
\begin{biz}Mint $\R\to\R$\end{biz}

\begin{te}$f\colon I\to\R,\ I\subset\R^N$ kompakt intervallum\\
  $f$ folytonos $\nn$ $f$ integrálható $I$-n.
\end{te}
\begin{biz}Lásd $\R\to\R$\end{biz}

\subsection{Az integrál értelmezése tetszőleges $H\subset \R^N$ korlátos tartományon}
\begin{de}Legyen $H\subset\R^N$ \underline{korlátos} halmaz, $f\colon H\to\R$. Ekkor $\exists I\subset\R^N$ kompakt
  intervallum, hogy $H\subset I$. Az $f$ függvény Riemann-integrálható a $H$-n, ha az
\[ \tilde{f}(x) := \begin{cases}f(x)& x\in H\\0& x\in I\setminus H \end{cases}\]
függvény integrálható az $I$-n. Ekkor
\[ \Int_Hf := \Int_I\tilde{f}\]
\end{de}

\begin{Megj}
  \item A definíció nem függ az $I$ megválasztásától
    \item A $H$-n vett integrálokra is érvényesek az alapvető tulajdonságokra kimondott állítások.
\end{Megj}

\subsection{Az integrál számítása}
$N=2$-re, $N\geq 2$ esetén hasonlóan.\\
Három eset:
\begin{enumerate}
\item $D_f$ kompakt intervallum (szukcesszív: egymás utáni integrálással)
\item $D_f$ ún. normáltartomány (szukcesszív integrálással)
\item $D_f$ egyéb tartomány (integrál-transzformációval)
\end{enumerate}

\subsubsection{Intervallumon}
$N=2,\ D_f=[a,b]\times[c,d]=:I$, $f\colon I\to\R$ korlátos függvény. Vesszük a tengelyekkel párhuzamos megtszetgörbéket:
$\forall x\in[a,b]\colon\vfi_x(y):=f(x,y)$, ahol $y\in[c,d]$. $\vfi_x\colon[c,d]\to\R$


\begin{te}Ha $f\in\Rint(I)$, akkor az
\[M(x):=I^*\vfi_x\text{ és }m(x):=I_*\vfi_x\quad x\in[a,b]\]
függvények integrálhatók $[a,b]$-n és
\[\Int_If=\Int_a^bM(x)\diff x=\Int_a^bm(x)\diff x\]
\end{te}

\begin{ko}
  Ha $f\in\Rint(I)$ és még tfh: $\forall x\in[a,b]\colon \vfi_x\in\Rint[c,d]$, azaz $I^*\vfi_x=I_*\vfi_x$, akkor
  \[\Int_If=\Int_a^b\left(\Int_c^df(x,y)\diff y\right)\diff x\]
  Jel: $\di\Int_If=\Int_a^b\!\!\Int_c^df(x,y)\diff x\diff y$
\end{ko}
\begin{ko}  Ha
  \begin{enumzjr}
  \item $f\in\Rint(I)$
  \item $\forall x\in[c,d]\ \vfi_x\in\Rint[c,d]$
  \item $\forall y\in[c,d]\ \vfi_y(x):=f(x,y)\quad x\in[a,b]$ függvény integrálható $[a,b]$-n
  \end{enumzjr} Ekkor
  \[\Int_I f=\Int_a^b\left(\Int_c^df(x,y)\diff y\right)\diff x=\Int_c^d\left(\Int_a^bf(x,y)\diff x\right)\diff y\]
  azaz a változók felcserélhetók
\begin{pl}
\begin{gather*}
  f(x,y):= x^2y,\ (x,y)\in[0,1]\times[0,1]\\
  \Int_0^1\!\!\Int_0^1f(x,y)\diff x\diff y=\Int_0^1\left(\Int_0^1f(x,y)\diff y\right)\diff x =\Int_0^1\left[
    \dfrac{x^2y^2}2 \right]_{y=0}^{y=1}\diff x=\Int_0^1\dfrac{x^2}2\diff x=\\\hspace{1em}=
  \left[\dfrac{x^3}6\right]_0^1=\dfrac16
  \intertext{illetve}
  \Int_0^1\!\!\Int_0^1f(x,y)\diff x\diff y=\Int_0^1\!\left(\Int_0^1f(x,y)\diff x\right)\diff y =\Int_0^1\left[
    \dfrac{x^3y}3 \right]_{x=0}^{x=1}\diff y=\Int_0^1\dfrac{y}3\diff y=\left[\dfrac{y^2}6\right]_0^1=\dfrac16
\end{gather*}
\end{pl}
\end{ko}

\subsubsection{Normáltartományon}
\begin{de}\ 
  \begin{enumzjb}
  \item $x$-re nézve normáltartomány: $\vfi_1(x)\leq\vfi_2(x)\quad(x\in[a,b])$
    \[H:=\{(x,y)\in\R^2:a\leq x\leq b;\  \vfi_1(x)\leq y\leq\vfi_2(x)\}\]
  \item $y$-ra nézve normáltartomány: $\psi_1(y)\leq\psi_2(y)\quad(y\in[c,d])$
    \[K:=\{(x,y)\in\R^2:c\leq y\leq d;\  \psi_1(y)\leq x\leq\psi_2(y)\}\]
  \end{enumzjb}
\end{de}

\begin{te}\ 
  \begin{enumzjb}
    \item $H\subset \R^2\ x$-re nézve normáltartomány, $f\colon H\to\R$  folytonos. Ekkor $f\in\Rint(H)$ és
      \[\Int_Hf=\Int_a^b\left(\Int_{\vfi_1(x)}^{\vfi_2(x)}\!\!f(x,y)\diff y\right)\diff x\]
    \item $K\subset \R^2\ y$-ra nézve normáltartomány, $f\colon K\to\R$  folytonos. Ekkor $f\in\Rint(K)$ és
      \[\Int_Kf=\Int_c^d\left(\Int_{\psi_1(y)}^{\psi_2(y)}\!\!f(x,y)\diff x\right)\diff y\]
  \end{enumzjb}
\end{te}

\subsection{Integrál-transzformációval}
Bármilyen egyéb tartományon integrál-transzformációval téglalapra vett integrálásra vezetjük vissza a kiuszámítását
(helyettesítéses integrálás).

\begin{pl}
\begin{gather*}
  x=r\cos\vfi=\Phi_1(r,\vfi)\\
  y=r\sin\vfi=\Phi_2(r,\vfi)\\
  \R^2\to\R^2\owns\Phi:=(\Phi_1,\Phi_2)\colon U\to V\text{ folytonosan deriválható bijektív leképezés}\\
  \det\Phi'(r,\vfi)=\det\begin{bmatrix}\cos\vfi & -r\sin\vfi\\\sin\vfi & r\cos\vfi\end{bmatrix} = r \ne 0\\
  \intertext{Ezért $\exists\Phi^{-1}\nn \Phi$  bijektív}
  \intertext{Igazolható:}
  \iint\limits_Vf(x,y)\diff x\diff y=\Int_V f=\iint\limits_U f(r\cos\vfi,r\sin\vfi)\cdot\underbrace{
  \vert\det\Phi'(r,\vfi)\vert }_{\underset{r}{\scriptsize{\Vert}}}\diff r\diff \vfi  
\end{gather*}
\end{pl}
\begin{te}[Integrál-transzformáció]
  Legyen $\Phi:=(\Phi_1,\,\Phi_2)\colon U\to V\quad (U,V\in\R^2)$ folytonosan deriválható és $\det\Phi' \ne0\ \,U$-n
  (vagyis $\Phi$ bijekció).\\
  Ha  $f\colon V\to\R$ folytonos $\nn f\in\Rint(V)$ és
  \[\Int_Vf=\Int_Uf\circ\Phi\cdot\vert\det\Phi'\vert\]
\end{te}

% Local Variables:
% fill-column: 120
% TeX-master: t
% End:

\end{document}

% Local Variables:
% fill-column: 120
% TeX-master: t
% End:
